%!TEX root = ../../common/main.tex

\chapter{Conclusion and outlook}
\label{ch:conclusion}

During the first \acs{LHC} run period from 2010 to 2012, the \LHCb experiment
showed an outstanding performance and high quality data was recorded. The
discovery of the Higgs boson \cite{Aad:2015zhl} endorses the theoretical
framework and is a huge success for particle physics. Still, the necessity of
\BSM physics is not deniable. Missing direct observations of \BSM effects like
heavy super-symmetric particles or a suitable dark matter candidate dash the
hope of new findings at the \si{TeV}-scale. \RunTwo of the \LHC will be crucial
for the future of the particle physics community and the design of
next-generation experiments.

While direct searches for new heavy particles are constrained by the available
collision energies, indirect searches are sensitive to \BSM effects through
higher order contributions, even with collision energies far below the threshold
for direct production of heavy particles. The \LHCb experiment was therefore
designed to perform high precision tests of the \SM in decays of \B and \D
mesons and to identify possible deviations from the \SM. Especially the
measurement of \CP violation in the decay of \BdToJpsiKS acts as a prime example
with small theoretical uncertainties, easy to reconstruct final states, and high
event yields.

The measurement of the \CP parameters \SJpsiKS and \CJpsiKS presented in this
thesis is realised on a dataset corresponding to an integrated luminosity of
$\SI{3.0}{\per\femto\barn}$ recorded by the \LHCb experiment in
\acl{protonproton} collisions at centre-of-mass energies of $\num{7}$ and
$\SI{8}{\TeV}$. The sample contains $\num{41500}$ reconstructed $\BdToJpsiKS$
candidates with a flavour tagging decision assigned by the \acl{SSpi} or the
\acl{OS} tagging algorithm. Using an \acl{uEML} fit, the \CP parameters
\SJpsiKS and \CJpsiKS are measured to be
%
\begin{equation*}
  \begin{split}
    \SJpsiKS &= \phantom{-}\num{0.731} \pm \num{0.035} \statp \pm \num{0.020} \systp \eqcm\eqand \\
    \CJpsiKS &=           \num{-0.038} \pm \num{0.032} \statp \pm \num{0.005} \systp \eqcm
  \end{split}
\end{equation*}
%
with a statistical correlation coefficient of $\rho(\SJpsiKS,\CJpsiKS) =
\num{0.483}$. With the parameter \CJpsiKS fixed to zero, the measurement yields
%
\begin{equation*}
  \SJpsiKS = \sintwobeta = \num{0.746 +- 0.030}\statp \eqpd
\end{equation*}

The measurement improves the previous \LHCb result \cite{Aaij:1497268} by
including a larger dataset, additional trigger lines, an optimised candidate
selection and by incorporating the \acl{SSpi} tagger decisions. It is the most
precise measurement of \CP violation at a hadron collider and is in excellent
agreement with the current world average.

The obtained precision of the measurement outperforms expectations based on
the result from the previous measurement, which predicts a sensitivity on
\SJpsiKS of $\num{0.04}$, only considering the larger data sample while assuming
all efficiencies to be unchanged. Implying the improved tagging performance
into these predictions, still results in an expected sensitivity on \SJpsiKS of
$\num{0.037}$, thus the choice of including additional trigger lines and
optimising the selection proves itself as beneficial.

Including this measurement, the updated world average on \sintwobeta
\cite{Amhis:2014hma} is
%
\begin{equation*}
  \sintwobeta = \num{0.691 +- 0.017} \eqcm
\end{equation*}
%
reducing the tension to the fit of all other \CKM matrix parameters slightly to
$\Delta\chisq = \num{1.66}$ \cite{Charles:2004jd}. 
\Cref{fig:conclusion:ckm_fitter_15} shows the ${(\dquark,\bquark)}$ unitarity
triangle in the ${(\ovE{\rho},\ovE{\eta})}$-plane from a global fit
incorporating all measured \CKM parameters \cite{Charles:2004jd} except from the
one reported here that is additionally shown to allow for a better comparison.
%
\begin{figure}[ht]
\centering
\includegraphics[width=1\textwidth]{private/content/conclusions/figs/ckmfitter_summer15.pdf}
\caption{Constraints on the ${(\dquark,\bquark)}$ unitarity triangle in the
${(\ovE{\rho},\ovE{\eta})}$-plane from a global fit incorporating all measured
\CKM parameters except the result presented in this thesis, which is shown
separately as a comparison in light blue. Regions outside the coloured areas
have $1-p > \SI{95.45}{\percent}$. The red hashed region of the global
combination corresponds to $\SI{68}{\percent}$ \acp{CL}. \cite{Charles:2004jd}}
\label{fig:conclusion:ckm_fitter_15}
\end{figure}

During \RunTwo of the \LHC, starting this year at a centre-of-mass energy of
$\SI{13}{\TeV}$, \LHCb is expected to collect a large dataset corresponding to
an integrated luminosity of around $\SI{5}{\per\femtobarn}$ until the next long
shut-down scheduled in 2018. With an estimated sensitivity on \SJpsiKS of
$\num{0.018}$ \cite{Moedden:2015}, this additional data will allow the \LHCb
collaboration to perform the world's best single measurement of \sintwobeta. As
this sensitivity falls below the current systematic uncertainty, a better
understanding of the background tagging asymmetry and a reduction of the flavour
tagging uncertainties are necessary. With more data being available, a
reassessment of the magnitude of the background tagging asymmetry is possible,
leading either to a model to describe the effect inside the likelihood fit or a
confirmation that no background asymmetry is present in data. The uncertainties
on the flavour tagging parameters will shrink with more data available in the
calibration and cross-check channels, as well as a better understanding of the
flavour tagging algorithms and new developments in the calibration procedure.
Furthermore, the systematic uncertainty due to neglecting the decay width
difference \DGd will become closer to the statistical uncertainty in \RunTwo,
such that the handling of \DGd has to be revisited
\cite{Moedden:2015}.

To further improve the sensitivity on the measurement of \sintwobeta, additional
decay modes will be explored. Based on the \RunOne dataset, the decay of the \Bd
into the $\Jpsi (\to\elel) \KS$ final state is expected to contribute with a
sensitivity of $\num{0.1}$ \cite{bdtojpsieeks:ramon}. Beyond that, higher
charmonium resonances as in $\BdToPsiTwoSKS$ will add sensitivity to the
combined result. Preparatory studies have shown an expected sensitivity on
\sintwobeta of $\num{0.09}$ in the combination of the $\psitwos\to\Jpsi\pi\pi$
and the $\psitwos\to\mumu$ final states for \RunOne \cite{Mueller:2014}.

The biggest \LHCb competitor will be the \BelleTwo experiment. The collaboration
plans to start data taking in 2018 with an expected instantaneous luminosity
$\num{50}$ times larger than its predecessor experiment \Belle. With an
assumption of $\num{100}$ days of efficient data taking per year, an integrated
luminosity of $\SI{8}{\per\attobarn}$ will be recorded per year. If these
expectations hold true, \BelleTwo will be able to reduce the total uncertainty
on \sintwobeta to $\num{0.010}$ on a short time scale with the uncertainty
mainly being dominated by irreducible systematic uncertainties
\cite{Aushev:2010bq}.

During the second long shut down of the \LHC in 2018 the \LHCb detector will be
upgraded to cope with a higher instantaneous luminosity of $L =
\SI{2e33}{\lumi}$ and an enhanced $\bquark\bquarkbar$-pair production rate of
$\num{e6}$ per second \cite{Bediaga:1443882}. The upgrade involves a new hybrid
pixel-sensor \acl{VELO} \cite{TDRVELO}, new tracking stations based on silicon
micro-strip and scintillating-fibre technology \cite{TDRTracking}, enhancements
of the \acl{RICH} systems, the calorimeters and the muon system \cite{TDRPID},
and a new software trigger concept capable of reading out the full detector at
the nominal bunch crossing frequency of $\SI{40}{\mega\hertz}$
\cite{TDRTrigger}. This will result in a large dataset corresponding to an
expected integrated luminosity of $\SI{50}{\per\fb}$ with the upgraded detector.
Assuming unchanged data taking efficiencies and similar performance of the
trigger, stripping, and flavour tagging, \acl{ToyMC} studies estimate a
sensitivity on \sintwobeta of around $\num{0.007}$ \cite{Moedden:2015} in the
decay \BdToJpsiKS.

The presented measurement is competitive with the results from the \BFactories and
\LHCb will be able to provide the world's most accurate measurement of
\sintwobeta with more data available by the end of \RunTwo. For measurements
beyond that, contributions from higher-order loop processes will become
important, such that the measurement of \CP violation in \BsToJpsiKS will be an
essential ingredient to further improve the precision on \sintwobeta and to test
the \SM expectations.


