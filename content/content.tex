%!TEX root = ../common/main.tex

% \chapter{Tests}
% %!TEX root = ../../common/main.tex

\chapter{Tests}

% ---------------------------------------
% Fonts
% ---------------------------------------
\section{Fonts}

\subsection{Blindtext}
\blindtext[3]
\blindlist{itemize}[3]
\blindmathtrue
\blindtext
\blindlist{enumerate}[3]
\blindmathpaper

% ---------------------------------------
% todonotes
% ---------------------------------------
\section{To-do notes}

\blindtext
\info{This is a very informative note}
\blindtext
\todo{A standard to-do note}
\blindtext
\addref{Cite this, cite that}
\blindtext
\missing{Add section about interesting study}
\blindtext
\important{Very urgent stuff!}
The quick brown fox jumps over the lazy dog. Jackdaws love my big
\replace{Jinx}{Sphinx} of Quartz. Pack my box with five dozen liquor jugs.
\blindtext
\redo{Recreate this ugly looking plot and make it nice!}

% ---------------------------------------
% Feynman Diagrams
% ---------------------------------------

\section{Feynman Diagrams}

\begin{tikzpicture}[line width=1.5 pt, scale=1.3]
  \everymath{\displaystyle}
%
  \node[draw,circle] (id1) at (0,1.7em) {$d$};
  \node[draw,circle] (id2) at (0,0em) {$u$};
  \node[draw,circle] (iu1) at (0,-1.7em) {$d$};
%
  \draw[dashed, black!30] (0,0) ellipse (1.8em and 3em);  
%
  \node[draw,circle] (od1) at (8em,1.7em) {$d$};
  \node[draw,circle] (ou1) at (8em,0em) {$u$};
  \node[draw,circle] (ou2) at (8em,-1.7em) {$u$};
%
  \draw[dashed, black!30] (8em,0) ellipse (1.8em and 3em);
%   
  \draw[fermion] (id1) -- (od1);
  \draw[fermion] (id2) -- (ou1);
  \draw[fermion] (iu1) -- (3em,-1.7em);
  \draw[fermion] (3em,-1.7em) -- (ou2);
    \draw[snake=snake, line before snake=.5em] (3em,-1.7em) -- (5em, -4.5em); 
  \draw[fermion] (5em, -4.5em) -- (8em, -4em);
  \draw[fermionbar] (5em, -4.5em) -- (8em, -5.5em);
  % NOTE: snake = snake is important
  \node at (3em,-4em) {$W^+$};
  \node at (8.6em,-4em) {$\nu_e$};
  \node at (8.5em,-5.5em) {$e$};
\end{tikzpicture}


\vspace{1em}

\begin{center}
  \begin{tikzpicture}[line width=1.5 pt, scale=1.3]
    \draw[fermion] (15:1.5)--(0,0);
    \draw[fermionbar] (180:1)--(0,0);
    \draw[vector] (-40:1)--(0,0);
    \node at (15:1.7) {$c$};
    \node at (180:1.2) {$b$};
    \node at (0.75,-.2) {$\Wp$};
    \coordinate (uin) at (-1,-.3);
    \coordinate (umid) at (-80:.5);
  % \begin{scope}[shift={(0,.3)}]
  %   \draw[fermion] (180:1) to [out=0,in=220] (40:1); 
  %   \node at (40:1.4) {$d$};
  %   \node at (180:1.2) {$d$};
  % \end{scope}
  \begin{scope}[shift={(-40:1)}]
    \draw[fermion] (-20:1)--(0,0);
    \draw[fermionbar] (40:1)--(0,0);
    \node at (40:1.2) {$c$};
    \node at (-20:1.2) {$s$};
    \coordinate (uend) at (-40:1.2);
    \node at (-40:1.4) {$d$};
  \end{scope}
  \draw[fermion] (uin) to [out=0, in=140] (umid) to [out=-35, in=160] (uend);
  \node at (-1.2,-.3) {$d$};
  \draw [gray,decorate,decoration={brace,amplitude=5pt},xshift=-4pt]
     (-1.2,-.5)  -- (-1.2,.2) 
     node [black,midway,left=4pt,xshift=-2pt] {$\Bd$};
  \draw [gray,decorate,decoration={brace,amplitude=5pt},xshift=-4pt]
     (2,.6)  -- (2,-.1) 
     node [black,midway,right=4pt] {$\Jpsi$};
  \draw [gray,decorate,decoration={brace,amplitude=5pt},xshift=-4pt]
     (2.2,-.8)  -- (2.2,-1.7) 
     node [black,midway,right=4pt] {$\KS$};
  \end{tikzpicture}
  %
  %
  \qquad\qquad
  %
  %
  \begin{tikzpicture}[line width=1.5 pt, scale=1.3]
    \draw[fermion] (15:1.5)--(0,0);
    \draw[fermionbar] (180:1)--(0,0);
    \draw[vector] (-40:1)--(0,0);
    \node at (15:1.7) {$u$};
    \node at (180:1.2) {$b$};
    % \node at (0.2,-.5) {$W$};
    \coordinate (uin) at (-1,-.3);
    \coordinate (umid) at (-80:.5);
  % \begin{scope}[shift={(0,.3)}]
  %   \draw[fermion] (180:1) to [out=0,in=220] (40:1); 
  %   \node at (40:1.4) {$d$};
  %   \node at (180:1.2) {$d$};
  % \end{scope}
  \begin{scope}[shift={(-40:1)}]
    \draw[fermion] (-20:1)--(0,0);
    \draw[fermionbar] (40:1)--(0,0);
    \node at (40:1.2) {$c$};
    \node at (-20:1.2) {$s$};
    \coordinate (uend) at (-40:1.2);
    \node at (-40:1.4) {$u$};
  \end{scope}
  \draw[fermion] (uin) to [out=0, in=140] (umid) to [out=-35, in=160] (uend);
  \node at (-1.2,-.3) {$u$};
  \draw [gray,decorate,decoration={brace,amplitude=5pt},xshift=-4pt]
     (-1.2,-.5)  -- (-1.2,.2) 
     node [black,midway,left=4pt,xshift=-2pt] {$B^+$};
  \draw [gray,decorate,decoration={brace,amplitude=5pt},xshift=-4pt]
     (2,.6)  -- (2,-.1) 
     node [black,midway,right=4pt] {$D$};
  \draw [gray,decorate,decoration={brace,amplitude=5pt},xshift=-4pt]
     (2.2,-.8)  -- (2.2,-1.7) 
     node [black,midway,right=4pt] {$K^+$};
  \end{tikzpicture}
\end{center}

\vspace{1em}

\begin{center}
  \begin{tikzpicture}[line width=1.5 pt, scale=1.3]
    \draw[fermion] (40:1.2)--(0,0);
    \draw[fermionbar] (180:1)--(0,0);
    \draw[vector] (-40:1)--(0,0);
    \node at (40:1.4) {$u$};
    \node at (180:1.2) {$b$};
    \node at (0.2,-.5) {$W$};
  \begin{scope}[shift={(0,.3)}]
    \draw[fermion] (180:1) to [out=0,in=220] (40:1); %****  
    \node at (40:1.4) {$d$};
    \node at (180:1.2) {$d$};
  \end{scope}
  \begin{scope}[shift={(-40:1)}]
    \draw[fermion] (-20:1)--(0,0);
    \draw[fermionbar] (20:1)--(0,0);
    \node at (-20:1.2) {$s$};
    \node at (20:1.2) {$u$};
  \end{scope}
  \end{tikzpicture}
\end{center}

% ---------------------------------------
% Introduction
% ---------------------------------------
%!TEX root = ../../common/main.tex

\chapter{Introduction}
\label{ch:introduction}



Discrete global symmetries play an outstanding role in the \SM. In particular,
the charge conjugation $\CSym$, the parity $\PSym$, and the time reversal
$\TSym$ transformations. The \CPT theorem \cite{set:cpt} states that the \CPT
symmetry holds for a quantum field theory, whereat the violation of a single of
the discrete symmetries $\CSym$, $\PSym$, or $\TSym$ is still possible.

Lee and Yang where the first to question the general assumption that all
physical interactions are invariant even under a single transformation by
stating a possible $\PSym$ violation in weak interaction \cite{Lee:1956qn}. This
statement was backed soon after by the measurement of $\PSym$ violation in the
$\beta^{-}$ decay of ${}^{60}\text{Co}$ \cite{Wu:1957my}. The experiment
revealed that the weak interaction only couples to fermions with left-handed
chirality or anti-fermions with right-handed chirality, thus also violating
symmetry under $\CSym$ transformation. Still, the interaction seems to behave
invariant under combined $\CP$ transformation. Not until 1964 the symmetry
violation under $\CP$ transformation was observed in the neutral kaon system
\cite{Christenson:1964fg} and consequently established in the \SM.

Writing the source of \CP violation into the \SM still was a challenge. The
triumphant idea came from Makoto Kobayashi and Toshihide Maskawa
\cite{Kobayashi:1973fv} proposing a third quark generation as a possibility to
explain \CP violation in the framework of the electroweak theory. The discovery
of the \bquark quark just 4 years later \cite{Herb:1977ek} provided enough
confidence in their theory to start planning the two \BFactories \Babar and
\Belle. Located at the two $\elel$ colliders PEP-II (Stanford, CA, US) and KEKB
(Tsukuba, JP) electrons and positrons were brought to collisions at asymmetric
energies. Both machines operated at a centre-of-mass energy of exactly the
\YFourS bottonium resonance at $\SI{10.58}{\GeV}$, producing $\B\Bbar$ meson
pairs at high rates of $\num{10e6}$ a day, therefore their nomenclature
\cite{Bevan:2014iga}.


\begin{itemize}
  \item symmetries in nature
  \item noether theorem?
  \item broken symmetries
  \item CPV in the SM, CKM mechanism
  \item CPV in lepton sector
  \item golden channel, sin2beta
  \item Measurement of sin2beta at B factories
  \item CKM unitarity
  \item matter-antimatter asymmetry
  \item ckm unitarity tests
  \item BSM physics
  \item indirect searches for BSM physics
  \item CPV at LHCb
  \item BdToJpsiKS (Vub tension)
  \item short summary of measurement
\end{itemize}

% References to people involved in the analysis
% \par\noindent\newline
\subsubsection*{Collaboration}
The work presented in this document was only possible to achieve in close
collaboration with colleagues from the \acs{LHCb} collaboration, most notably
Frank Meier and Julian Wishahi from the local working group, they are---together
with the author---the contact authors of this analysis published as:
%
\begin{quotation}
  \fullcite{Aaij:2015vza}.
\end{quotation}

Mirco Dorigo and Ulrich Eitschberger from the flavour tagging group provided the
flavour tagging calibration measurements (\cf
\cref{sec:flavour_tagging:calibration:os,sec:flavour_tagging:calibration:ss}).
The calculation of the kaon regeneration (\cf
\cref{sec:measurement_of_sin2beta:cpv_measurement:kaon_regeneration}) was
performed by Jeroen van Tilburg from the \B to charmonium working group.

% Structure of the document
% \par\noindent\newline
% The document is structured as follows:
\subsubsection*{Outline}

In \cref{ch:cpv_theory} the \SM is shortly recapped with an emphasis on flavour
physics and \CP violation in the quark sector. The nature of Yukawa
interactions and the appearance of the \CKM matrix through spontaneous symmetry
breaking are summarised. Then, the time evolution of \Bmesons including decay
and flavour oscillations is outlined. Finally, an overview of the measurement of
\CP violation in the decay of \BdToJpsiKS is given.

The \LHCb detector and its subsystems are briefly described in
\cref{ch:lhcb_experiment} with particular attention to the track reconstruction
and \acl{PID} techniques. The \LHCb trigger system is sketched and the software
stack is depicted. At this point, the data taking in the first successful years
of running is summarised also.

As it plays an important role in this analysis the flavour tagging algorithms
utilised at \LHCb are discussed in more detail in \cref{ch:flavour_tagging}.
After a general overview and a comparison to developments at the \BFactories the
flavour tagging algorithms are explained. Afterwards, the calibration of the
flavour tagging output is motivated and described. The performance of the
methods is then described and an outlook on recent developments is given.

The main part (\cref{ch:measurement_of_sin2beta}) contains the details of the
performed measurement. The data taking and preparation is explained as well as
studies of potential background contributions. The influence of decay time
acceptance and resolution effects is briefly summarised. Before the results are
presented the likelihood function is presented. Finally, all studies of
systematic effects are collected.

This thesis concludes with a summary and puts the results in context to the
global \CKM picture. Furthermore, \RunTwo and \LHCb upgrade prospects are
presented.


\chapter{CP violation in the $\Bz$ meson system}
Roundup of B physics: observation of b quarks, B oscillations, CKM, CPV
\section{Flavour physics}
\section{Oscillations of neutral B mesons}
\begin{itemize}
  \item Difference between mass and weak eigenstates, Wiger-Weisskopf approximation, mass difference, decay-time width difference, behaviour under C/P transformations, quark mixing phase definitions.
  \item Definition of states for the B0 system (and maybe the K system), 
  \item Experimental status of B0-B0bar oscillations, the LHCb measurement of deltam in Bd2JpsiKstar and Bd2Dpi.
\end{itemize}

\section{Charge-Parity violation in the SM}
\begin{itemize}
  \item Discrete symmetries, CPT, weak and strong phases, CPV observables, classification of CPV, CP eigenstates
  \item CPV in SM: gauge symmetries, fermions in the SM, CKM matrix, the unitarity triangle, CKM parametrization
\end{itemize}
\subsection{CP violation in the decay Bd2JpsiKS}
\begin{itemize}
  \item CPV in this unique decay, master formulas, etc.
\end{itemize}

\section{Decay-time dependent measurement of sin2beta}
\begin{itemize}
  \item Description of the measurement technique: decay-time dependence, flavour tagging
  \item Experimental status: measurements at the B factories, results, world average
\end{itemize}

\chapter{The LHCb experiment}
lumi leveling, angular acceptance, bbbar cross section, incoherent production, boost along zaxis, number of B decays in acceptance, production of bbar pairs
\section{Components}
\begin{itemize}
  \item Purpose
  \item Technology
  \item Design
  \item Performance(?)
\end{itemize}
\subsection{VeLo}
\begin{itemize}
  \item purpose: displaced secondary vertices, decay length, decay time, IP
  \item silicon modules in r and phi, geometrical dimensions, mechanical accuracy, closest approach to beam
  \item acceptance: 1.6 < eta < 4.9; |z| < 10.6cm
  \item requirement of at least three hits in the VELO stations
  \item pile-up veto system
  \item retractable, RF foil, in vacuum (primary and secondary)
  \item hardware interlock system
  \item performance(?)
\end{itemize}
\subsection{BCM ;)}
\subsection{TT \& IT}
\begin{itemize}
  \item silicon microstrip sensors
  \item position, geometrical dimensions, active area, x-u-v-x alignment
  \item TT pros: spatial resolution, hit occupancy, signal shaping time, single-hit efficiency, radiation damage, material budget, number of readout channels
  \item IT parts: ...
\end{itemize}
\subsection{RICH}
\begin{itemize}
  \item general: cherenkov light detectors, spherical and flat mirrors, hybrid photon detectors
  \item RICH1: low momentum charged particles, 1-60GeV/c, aerogel and C4F10 radiators
  \item RICH2: CF4 gas radiator, 15-100GeV/c, reduced polar angle acceptance
  \item HPDs: 
\end{itemize}
\subsection{OT}
\begin{itemize}
  \item drift-time detector, tracking of charged particles
  \item array of gas-tight straw-tube modules, each module two layers of drift-tubes
  \item drift-time < 50ns, drift-coordinate resolution 200mum
  \item three stations, each of four layers in x-u-v-x alignment, gas admixture, acceptance
\end{itemize}
\subsection{Calo}
\begin{itemize}
  \item trigger tasks: transverse energy hadron/electron/photon candidates for L0
  \item PID for electrons, photons, hadrons
  \item energy and position measurement
  \item design: ECAL (+SPD and PS), HCAL
  \item technical details: create scintillation light, then transmit to a photo-multiplier using fibres
\end{itemize}
\subsection{Muon}
\begin{itemize}
  \item five stations of multi-wire proportional chambers
  \item muon ID important for charmonium final states and rare decays
  \item high pT trigger for L0, and muon ID for HLT
\end{itemize}
\section{Trigger}
two level trigger system: L0 and HLT, L0 from custom made electronics, real-time with bunch crossing-frequency, HLT on software in processor farm
\subsection{L0 trigger}
from 40MHz to 1MHz, trigger on large transverse momentum pT and energy ET; the hightest ET hadron, electron, photon clusters in the calo; the two highest pT muons in the muon chambers; pile-up system in velo calculates the number of PVs; the caos calculate the total observed energy and estimate the number of tracks based on the number of hits in the SPD; using these global event cuts to reject events which would otherwise be triggered due to large combinatorics
\subsection{High level trigger}
reduced the !MHz L0 output rate to 5kHz; software based and flexible, again composed of two parts: HLT1 and HLT2; 
\section{Online system}
\DAQ, \TFC, and \ECS
\begin{itemize}
  \item \DAQ: transport of bunch-crossing data from the detector fond-end electornics to permanent storage
  \item Front-end detector electronics -> TELL1 board : receiver cards -> \acp{FPGA} (processing, zero-suppresion, data compression) -> SyncLink (another \FPGA) collect and send raw IP-packet by GbEthernet mezzanine cards -> DAQ
  \item TELL1 has a credit-card size PC connected that interfaces to the \ECS
  \item Clock and synchronisation signals (\eg triggers) are transmitted trough the on-board \TTC interface
\end{itemize}
\subsection{Front-end electronics}
\section{Software stack}
\section{Computing grid}

\chapter{The measurement of sin2beta}
\section{Data preparation}
\begin{itemize}
  \item Year of data taking, collected integrated luminosity
  \item Data processing: Reco version, DaVinci, etc.
  \item general remarks on the data preparation
  \item DTF explanation including constraints used in the fit
  \item Observables
\end{itemize}

\subsection{Trigger}
L0, HLT1, HLT2 requirements, TCKs?, efficiencies

\subsection{Stripping}
Stripping details, cuts, ...

\subsection{Offline selection}
Selection cuts, FoM, efficiencies

\subsection{Multiple candidates}
Why do we see multiple candidates, difference between multiple PVs and multiple B candidates, how to remove them

\subsection{Simulated data}
Data sets used, software stack and versions, simulation conditions, DEC file, generation parameters

\section{Flavour tagging}
\begin{itemize}
  \item General introduction to FT at LHCb
  \item influence on measuring sin2beta, Dilution
  \item tagging efficiency, mistag fraction, tagging power
\end{itemize}

\subsection{OS FT}
OS algorithms: kaon, muon, electron, vertex charge, charm (+ MVA versions)

\subsection{SS FT}
SS algorithms: kaon/pion/proton

\subsection{Calibration}
\begin{itemize}
  \item calibration method
  \item tagging correlations
  \item OS calibration using Bu2JpsiK: procedure, transferability/validity to Bd2JpsiKS
  \item SS calibration using Bd2JpsiKst: procedure, transferability/validity to Bd2JpsiKS
  \item determination of systematic uncertainties
\end{itemize}

\subsection{Performance}
Performance of the FT algorithms, effect on dilution of asymmetry

\section{Backgrounds}
\section{Decay time resolution and acceptance}
\subsection{Resolution}
\subsection{Acceptance}
\section{Likelihood fit}
\subsection{Fitter validation}
\section{CPV measurement}
\subsection{...}
\subsection{Kaon regeneration}
\section{Study of systematic effects}
\subsection{Cross-checks}
\subsection{Systematics}
\subsection{Summary of systematic effects}

\chapter{Conclusions and Outlook}
\section{Results}
\section{The global picture}
\section{Run II prospects}
\section{Summary}
