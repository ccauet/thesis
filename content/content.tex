%!TEX root = ../common/main.tex

% \chapter{Tests}
% %!TEX root = ../../common/main.tex

\chapter{Tests}

\section{Subsection}
This is a first test of the common/private layout


% ---------------------------------------
% Introduction
% ---------------------------------------
%!TEX root = ../../common/main.tex

\chapter{Introduction \& Motivation}
Start with a first short introduction of the SM, B physics phenomenology, shortly introducing CPV, and sin2beta


\chapter{CP violation in the $\Bz$ meson system}
Roundup of B physics: observation of b quarks, B oscillations, CKM, CPV
\info{Write some introduction stuff}
\section{Flavour physics}
\section{Oscillations of neutral B mesons}
\begin{itemize}
  \item Difference between mass and weak eigenstates, Wiger-Weisskopf approximation, mass difference, decay-time width difference, behaviour under C/P transformations, quark mixing phase definitions.
  \item Definition of states for the B0 system (and maybe the K system), 
  \item Experimental status of B0-B0bar oscillations, the LHCb measurement of deltam in Bd2JpsiKstar and Bd2Dpi.
\end{itemize}

\section{Charge-Parity violation in the SM}
\begin{itemize}
  \item Discrete symmetries, CPT, weak and strong phases, CPV observables, classification of CPV, CP eigenstates
  \item CPV in SM: gauge symmetries, fermions in the SM, CKM matrix, the unitarity triangle, CKM parametrization
\end{itemize}
\subsection{CP violation in the decay Bd2JpsiKS}
\begin{itemize}
  \item CPV in this unique decay, master formulas, etc.
\end{itemize}

\section{Decay-time dependent measurement of sin2beta}
\begin{itemize}
  \item Description of the measurement technique: decay-time dependence, flavour tagging
  \item Experimental status: measurements at \replace{the B factories}{Babar and Belle}, results, world average
\end{itemize}
\addref{PDG}

\chapter{The LHCb experiment}
lumi leveling, angular acceptance, bbbar cross section, incoherent production, boost along zaxis, number of B decays in acceptance, production of bbar pairs
\section{...}
\section{Track reconstruction}
\subsection{VeLo}
\info{This is the velo}
\begin{itemize}
  \item purpose: displaced secondary vertices, decay length, decay time, IP
  \item silicon modules in r and phi, geometrical dimensions, mechanical accuracy, closest approach to beam
  \item acceptance: 1.6 < eta < 4.9; |z| < 10.6cm
  \item requirement of at least three hits in the VELO stations
  \item pile-up veto system
  \item retractable, RF foil, in vacuum (primary and secondary)
  \item hardware interlock system
  \item performance(?)
\end{itemize}
\subsection{TT \& IT}
\begin{itemize}
  \item silicon microstrip sensors
  \item position, geometrical dimensions, active area, x-u-v-x alignment
  \item TT pros: spatial resolution, hit occupancy, signal shaping time, single-hit efficiency, radiation damage, material budget, number of readout channels
  \item IT parts: ...
\end{itemize}
\subsection{OT}
\begin{itemize}
  \item drift-time detector, tracking of charged particles
  \item array of gas-tight straw-tube modules, each module two layers of drift-tubes
  \item drift-time < 50ns, drift-coordinate resolution 200mum
  \item three stations, each of four layers in x-u-v-x alignment, gas admixture, acceptance
\end{itemize}
\subsection{Muon}
\todo{Muon ftw}
\begin{itemize}
  \item five stations of multi-wire proportional chambers
  \item muon ID important for charmonium final states and rare decays
  \item high pT trigger for L0, and muon ID for HLT
\end{itemize}
\subsection{Track reconstruction technique and performance}
Results are form Bd2JpsiKS!
\begin{itemize}
  \item hits from VELO, TT, IT and OT are combined to form trajectories form the VELO to the calorimeters. 
  \item list different track types: long, upstream, downstream, VELO, T
  \item track seeds, Kalman filter, pattern recognition, ghosts
  \item definition of: reconstructible, successfully reconstructed
  \item performance numbers for long tracks and downstream tracks
\end{itemize}

\section{Particle identification}
\subsection{RICH}
\begin{itemize}
  \item general: cherenkov light detectors, spherical and flat mirrors, hybrid photon detectors
  \item RICH1: low momentum charged particles, 1-60GeV/c, aerogel and C4F10 radiators
  \item RICH2: CF4 gas radiator, 15-100GeV/c, reduced polar angle acceptance
  \item HPDs: 
\end{itemize}
\subsection{Calo}
\begin{itemize}
  \item trigger tasks: transverse energy hadron/electron/photon candidates for L0
  \item PID for electrons, photons, hadrons
  \item energy and position measurement
  \item design: ECAL (+SPD and PS), HCAL
  \item technical details: create scintillation light, then transmit to a photo-multiplier using fibres
\end{itemize}
\subsection{Particle identification technique and performance}
\begin{itemize}
  \item combine info from RICHs, calo, and muon system to identify: e,mu,pi,K,p also gamma, pi0
  \item hadron PID: likelihood approach, RICH pattern mached to all possible tracks under assumption of partice hypotheses. -> global pattern-recongnition
  \item RICH efficiency
  \item muon PID: extrapolating reconstructed tracks into the muon stations; considered a muon cand if a minimum number of stations have hits in their field of interest FOI
  \item muon PID efficiency
  \item lectron PID: balance of track momentum and energy of the charged cluster in the ECAL matched with extrapolated tracks; matching bremsstrahlung photons to electron tracks before the magnet
  \item gamma PID: ECAL cluster w/o associated track
  \item pi0 PID: 
  \item global performance: PV resolution of 10mum transverse to the beam and 60mum along the beam axis; invariant mass resolution between 12 and 25 MeV; lifetime resolution of around 40fs
\end{itemize}

\section{Trigger}
two level trigger system: L0 and HLT, L0 from custom made electronics, real-time with bunch crossing-frequency, HLT on software in processor farm
\subsection{L0 trigger}
from 40MHz to 1MHz, trigger on large transverse momentum pT and energy ET; the hightest ET hadron, electron, photon clusters in the calo; the two highest pT muons in the muon chambers; pile-up system in velo calculates the number of PVs; the caos calculate the total observed energy and estimate the number of tracks based on the number of hits in the SPD; using these global event cuts to reject events which would otherwise be triggered due to large combinatorics
\subsection{High level trigger}
reduced the 1MHz L0 output rate to 5kHz; software based and flexible, again composed of two parts: HLT1 and HLT2; 
\section{Online system}
\DAQ, \TFC, and \ECS
\begin{itemize}
  \item \DAQ: transport of bunch-crossing data from the detector fond-end electornics to permanent storage
  \item Front-end detector electronics -> TELL1 board : receiver cards -> \acp{FPGA} (processing, zero-suppresion, data compression) -> SyncLink (another \FPGA) collect and send raw IP-packet by GbEthernet mezzanine cards -> DAQ
  \item TELL1 has a credit-card size PC connected that interfaces to the \ECS
  \item Clock and synchronisation signals (\eg triggers) are transmitted trough the on-board \TTC interface
\end{itemize}

\section{Software stack}
\begin{itemize}
  \item \Gaudi: LHCb software framework (used by ATLAS as well)
  \item \Brunel: reconstruction
  \item \Brunel: trigger
  \item \DaVinci: end-user analysis software
  \item Simulation: \Gauss, \Boole, \Pythia, \EvtGen, \Photos, \Herwigpp, \Sherpa, \GeantFour
\end{itemize}
\section{Computing grid}
...

\chapter{The measurement of sin2beta}
\section{Data preparation}
\begin{itemize}
  \item Year of data taking, collected integrated luminosity
  \item Data processing: Reco version, DaVinci, etc.
  \item general remarks on the data preparation
  \item DTF explanation including constraints used in the fit
  \item Observables
\end{itemize}

\subsection{Trigger}
L0, HLT1, HLT2 requirements, TCKs?, efficiencies

\subsection{Stripping}
Stripping details, cuts, ...

\subsection{Offline selection}
Selection cuts, FoM, efficiencies

\subsection{Multiple candidates}
Why do we see multiple candidates, difference between multiple PVs and multiple B candidates, how to remove them

\subsection{Simulated data}
Data sets used, software stack and versions, simulation conditions, DEC file, generation parameters

\section{Flavour tagging}
\begin{itemize}
  \item General introduction to FT at LHCb
  \item influence on measuring sin2beta, Dilution
  \item tagging efficiency, mistag fraction, tagging power
\end{itemize}

\subsection{OS FT}
OS algorithms: kaon, muon, electron, vertex charge, charm (+ MVA versions)

\subsection{SS FT}
SS algorithms: kaon/pion/proton

\subsection{Calibration}
\begin{itemize}
  \item calibration method
  \item tagging correlations
  \item OS calibration using Bu2JpsiK: procedure, transferability/validity to Bd2JpsiKS
  \item SS calibration using Bd2JpsiKst: procedure, transferability/validity to Bd2JpsiKS
  \item determination of systematic uncertainties
\end{itemize}

\subsection{Performance}
Performance of the FT algorithms, effect on dilution of asymmetry

\section{Backgrounds}
\section{Decay time resolution and acceptance}
\subsection{Resolution}
\subsection{Acceptance}
\section{Likelihood fit}
\subsection{Fitter validation}
\important{Validate the fitter asap!}
\section{CPV measurement}
\subsection{...}
\subsection{Kaon regeneration}
\section{Study of systematic effects}
\subsection{Cross-checks}
\subsection{Systematics}
\subsection{Summary of systematic effects}

\chapter{Conclusions and Outlook}
\todo{Write nice conclusions}
\section{Results}
\section{The global picture}
\section{Run II prospects}
\section{Summary}
