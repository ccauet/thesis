%!TEX root = ../../common/main.tex

\section{Flavour physics}
\label{sec:cpv_theory:flavour_physics}

The field of flavour physics includes the description of the weak interaction of
quarks and leptons, mixing of neutral meson systems, and in general the time
evolution of those states. Inferred by the nature of the Yukawa interactions \CP
violation is possible and present in nature. In this thesis the focus is laid on
the quark sector.

\subsection{The \acs{CKM} quark mixing matrix}
\label{sec:cpv_theory:flavour_physics:ckm_matrix}

To ensure local $\group{SU}{2}\otimes\group{U}{1}$ gauge invariance of the
electroweak Lagrangian while at the same time preserve renormalisability of the
theory, the mechanism of spontaneous symmetry breaking is exploited. While the
mechanism also implies new massive vector boson fields as well as the appearance
of a massive scalar particle---the Higgs boson---the focus will be on the quark
sector.

By introducing a complex scalar Higgs field $\phi \equiv
\left(\begin{smallmatrix} \phi ^{+} \\ \phi^{0}\end{smallmatrix}\right)$ a
$\group{SU}{2}\otimes\group{U}{1}$ invariant Yukawa Lagrangian
$\Lagrangian{Y}{}$ for the quark sector can be written down,
%
\begin{equation}\label{sec:cpv_theory:flavour_physics:ckm_matrix:yukawa_lagrangian}
  \Lagrangian{Y}{\quark} = -Y_{ij}^{\dquark} \ovE{Q_{Li}} \phi \dquark_{Rj}^{\prime} - Y_{ij}^{\uquark} \ovE{Q_{Li}} \phi^{\ast} \uquark_{Rj}^{\prime} + \hc \eqcm
\end{equation}
%
with $Y_{ij}^{\quark}$ denoting the Yukawa couplings, $Q_{Li}$ being the
left-handed quark doublets, and $\quark_{Rj}$ the right-handed singlets. 

Being massless prior to this, the charged leptons, quarks, and the $\Wpm$ and
$\Zboson$ bosons obtain masses after spontaneously breaking the symmetry and
substituting the Higgs field by its \VEV
%
\begin{equation}\label{sec:cpv_theory:flavour_physics:ckm_matrix:higgs_vev}
  \langle\phi\rangle = \frac{1}{\sqrt{2}} \begin{pmatrix}
    0 \\
    v \\
  \end{pmatrix}\eqpd
\end{equation}
%
The Yukawa couplings $Y_{ij}$ are not constraint by the theory thus are
completely arbitrary, leading to most of the free \SM parameters. The coupling
matrices---also known as mass matrices---are a priori non-diagonal in the weak
interaction basis, but in general can be diagonalised by bi-unitary
transformations. As a consequence of the diagonalisation the quark fields are
transformed as well into the mass eigenstates basis allowing for
flavour-changing currents through $\Wpm$ boson exchange 
%
\begin{equation}
  \frac{-g}{\sqrt{2}} \left( \uquarkbar, \cquarkbar, \tquarkbar \right)_L \gamma^\mu \Wp_\mu \VCKM 
  \begin{pmatrix}
    \dquark \\
    \squark \\
    \bquark \\
  \end{pmatrix}_L \eqpd
\end{equation}
%
with the weak charge $g$ and the transformation matrix denoted $\VCKM$. The \CKM
matrix elements $\VCKM^{ij}$ connect an up-type quark $i$ to a down-type quark
$j$ with a probability proportional to $\vert\VCKM^{ij}\vert^2$. By definition, the
$3 \times 3$ \CKM matrix is unitary, $\VCKM\VCKM^\dagger=1$, and complex, with
$\num{18}$ free parameters. Following the unitarity conditions
%
\begin{equation}\label{sec:cpv_theory:flavour_physics:ckm_matrix:unitarity_relations}
  \sum_i V_{ij}^{\phantom{\ast}} V_{ik}^{\ast} = \delta_{jk} \eqand
  \sum_j V_{ij}^{\phantom{\ast}} V_{kj}^{\ast} = \delta_{ik} \eqcm
\end{equation}
%
and a redefinition of arbitrary quark phases, reduces the number of parameters
to three angles and one complex phase. The matrix elements are denoted
%
\begin{equation}
  \begin{pmatrix}
    \dquark^\prime \\
    \squark^\prime \\
    \bquark^\prime \\
  \end{pmatrix}_L
  = 
  \begin{pmatrix}
    \Vud & \Vus & \Vub \\
    \Vcd & \Vcs & \Vcb \\
    \Vtd & \Vts & \Vtb \\
  \end{pmatrix}
  \begin{pmatrix}
    \dquark \\
    \squark \\
    \bquark \\
  \end{pmatrix}_L
  = \VCKM
  \begin{pmatrix}
    \dquark \\
    \squark \\
    \bquark \\
  \end{pmatrix}_L
\end{equation}
%
and can be expressed as a complex rotation matrix with the three angles
$\theta_{12}, \theta_{12}, \theta_{12} \in {[0, \pi/2]}$ and a phase $\delta \in
{(-\pi, \pi]}$. With the definition of $s_{ij} \equiv \sin \theta_{ij}$ and
$c_{ij} \equiv \cos \theta_{ij}$, an exact representation of the \CKM matrix can
be written as
%
\begin{multline}
  \VCKM = \\
  \begin{pmatrix}
    c_{12} c_{13}                                                   & s_{12} c_{13}                                                 & s_{13} \exponential{-i\delta} \\
    -s_{12} c_{23} - c_{12} s_{23} s_{13} \exponential{i\delta}     & c_{12} c_{23} - s_{12} s_{23} s_{13} \exponential{i\delta}    & s_{23} c_{13}                 \\
    s_{12} s_{23} - c_{12} c_{23} s_{13} \exponential{i\delta}      & -c_{12} s_{23} - s_{12} c_{23} s_{13} \exponential{i\delta}   & c_{23} c_{13}                 \\
  \end{pmatrix}\eqpd
\end{multline}
%
where the phase $\delta$ is the only source of \CP violation in the \SM. A
popular approximation by Wolfenstein takes advantage of the measured hierarchy
of the matrix elements, where the diagonal elements are of $\order{1}$, while
the off-diagonal elements follow $\vert\Vub\vert^2 \ll \vert\Vcb\vert^2 \ll
\vert\Vus\vert^2 \ll 1$ or in terms of the rotation angles $s_{12} \ll s_{23}
\ll s_{12} \ll 1$. Exploiting this hierarchy in terms of an expansion with leads
to the Wolfenstein parametrisation
%
\begin{equation}  
  \VCKM = \begin{pmatrix}
    1 - \sfrac{\lambda^2}{2}        & \lambda                     & A \lambda^3 (\rho - i\eta)  \\
    -\lambda                        & 1 - \sfrac{\lambda^2}{2}    & A \lambda  2                \\
    A \lambda^3 (1 - \rho - i\eta)  & - A \lambda^2               & 1                           \\
  \end{pmatrix}
  + \order{\lambda^4}
\end{equation}
%
written in terms of
%
\begin{equation}
  s_{12} = \lambda, \eqspace s_{23} = A \lambda^2, \eqand s_{13}\exponential{i\delta} = A \lambda^3 (\rho + i \eta) \eqpd
\end{equation}
%
Given the unitarity relations in
\cref{sec:cpv_theory:flavour_physics:ckm_matrix:unitarity_relations} the six
vanishing combinations can be represented in a triangle in the complex plane.
The most prominent unitarity triangle arises from
%
\begin{equation}
  \Vud^{\phantom{\ast}}\Vub^{\ast} + \Vcd^{\phantom{\ast}}\Vcb^{\ast} + \Vtd^{\phantom{\ast}}\Vtb^{\ast} = 0 
\end{equation}
%
where normalising each side of the triangle by
$\Vcd^{\phantom{\ast}}\Vcb^{\ast}$ yields vertices at ${(0,0)}$ and ${(1,0)}$
and an apex at ${(\ovE{\rho},\ovE{\eta})}$. The area of all triangles is equal
and can be expressed as half of the Jarlskog invariant $J = \pm \Im
V_{ik}^{\phantom{\ast}} V_{jl}^{\phantom{\ast}} V_{il}^{\ast} V_{jk}^{\ast}$
with $i \neq j$ and $l \neq k$. It is a measure of \CP violation in the \SM and
can be determined to be $\vert J \vert = \lambda^6 A^2 \eta \approx \num{3e-5}$.

The angles and the sides of the triangle can be expressed as
%
\begin{equation}
  \alpha = \arg\left(-\frac{\Vtd^{\phantom{\ast}}\Vtb^{\ast}}{\Vud^{\phantom{\ast}}\Vub^{\ast}}\right),\eqspace 
  \beta =  \arg\left(-\frac{\Vcd^{\phantom{\ast}}\Vcb^{\ast}}{\Vtd^{\phantom{\ast}}\Vtb^{\ast}}\right),\eqspace 
  \gamma = \arg\left(-\frac{\Vud^{\phantom{\ast}}\Vub^{\ast}}{\Vcd^{\phantom{\ast}}\Vcb^{\ast}}\right)
\end{equation}
%
and
%
\begin{equation}
  R_t = \left| \frac{\Vtd^{\phantom{\ast}}\Vtb^{\ast}}{\Vcd^{\phantom{\ast}}\Vcb^{\ast}} \right|,\eqspace
  R_u = \left| \frac{\Vud^{\phantom{\ast}}\Vub^{\ast}}{\Vcd^{\phantom{\ast}}\Vcb^{\ast}} \right|,\eqspace
  R_c = \left| \frac{\Vcd^{\phantom{\ast}}\Vcb^{\ast}}{\Vcd^{\phantom{\ast}}\Vcb^{\ast}} \right|
\end{equation}
%
which simplifies to
%
\begin{equation}
  R_t \exponential{-i\beta} + R_u \exponential{-i\gamma} = 1 \eqpd
\end{equation}
%

\missing{UTFit, CKMfitter, global CKM picture}

\subsection{B meson decay}
\label{sec:cpv_theory:flavour_physics:bdecays}

decay amplitudes to final state f or CP conjugate final state fbar
%
\begin{equation}
  \begin{alignedat}{2}
    & A_f         = \matrixelement{f}{\Hamiltonian{}{}}{\Meson},        \eqspace && \ovE{A}_f         = \matrixelement{f}{\Hamiltonian{}{}}{\Mesonbar}, \\
    & A_{\ovE{f}} = \matrixelement{\ovE{f}}{\Hamiltonian{}{}}{\Meson},  \eqspace && \ovE{A}_{\ovE{f}} = \matrixelement{\ovE{f}}{\Hamiltonian{}{}}{\Mesonbar}
  \end{alignedat}
\end{equation}
%
transformation under CP for initial and final state with two new phases zetaM and zetaf
%
\begin{equation}
  \begin{alignedat}{2}
    & \CP\ket{\Meson}    = \exponential{+i\zeta_{M}} \ket{\Mesonbar},\eqspace && \CP \ket{f}        = \exponential{+i\zeta_{f}} \ket{\ovE{f}}, \\
    & \CP\ket{\Mesonbar} = \exponential{-i\zeta_{M}} \ket{\Meson}   ,\eqspace && \CP \ket{\ovE{f}}  = \exponential{-i\zeta_{f}} \ket{f}
  \end{alignedat}
\end{equation}
%
such that CP squared gives unity. if CP conserved $\ovE{A}_{\ovE{f}} = \exponential{i(\zeta_f - \zeta_A)} A_f$

\subsection{Oscillations of neutral B mesons}
\label{sec:cpv_theory:flavour_physics:bmixing}

Time evolution of meson state, initial state at t=0
%
\begin{equation}
  \ket{\Psi(t)} = \psi_1(t) \ket{\Meson} + \psi_2(t) \ket{\Mesonbar}
\end{equation}
%
Weisskopf-Wigner approx time evolution, Hamiltonian
%
\begin{equation}
  \mathbf{H} = \mathbf{M} - \frac{i}{2} \mathbf{\Gamma}
\end{equation}
%
given Schrödinger DGL
%
\begin{equation}
  i \frac{\dif}{\dif t} 
  \begin{pmatrix}
    \psi_1 \\
    \psi_2
  \end{pmatrix}
  =
  \mathbf{H}
  \begin{pmatrix}
    \psi_1 \\
    \psi_2
  \end{pmatrix}
  =
  \begin{pmatrix}
    m_{11} - \frac{i}{2} \Gamma_{11}    & m_{12} - \frac{i}{2} \Gamma_{12} \\
    m_{21} - \frac{i}{2} \Gamma_{21}    & m_{22} - \frac{i}{2} \Gamma_{22} \\
  \end{pmatrix}
  \begin{pmatrix}
    \psi_1 \\
    \psi_2
  \end{pmatrix}
\end{equation}
%
CPT symmetry $m_{11} = m_{22} = m$, $m_{21} = m_{12}^{\ast}$, $\Gamma_{11} = \Gamma_{22} = \Gamma$, $\Gamma_{21} = \Gamma_{12}^{\ast}$
%
then diagonalising the hamiltonian yields mass eigenstates heavy and light
%
\begin{equation}
  \begin{split}
    \ket{\Meson_L} &= p \ket{\Meson} + q \ket{\Mesonbar} \\
    \ket{\Meson_H} &= p \ket{\Meson} - q \ket{\Mesonbar}
  \end{split}
\end{equation}
%
with $\abs{q}^2 + \abs{p}^2 = 1$ and eigenvalues
%
\begin{equation}
  \begin{split}
    \mu_H &= m_H - \frac{i}{2} \Gamma_H \\
    \mu_L &= m_L - \frac{i}{2} \Gamma_L
  \end{split}
\end{equation}
%
with the definitions 
%
\begin{equation}
  m = \frac{m_H + m_L}{2}, \eqspace \Gamma = \frac{\Gamma_H + \Gamma_L}{2}
\end{equation}
%
and
%
\begin{equation}
  \DM = m_H - m_L, \eqspace \DG = \Gamma_H - \Gamma_L
\end{equation}
%
and
%
\begin{equation}
  \left(\frac{q}{p}\right)^2 = \frac{m_{12}^{\ast} - \frac{i}{2} \Gamma_{12}^{\ast}}{m_{12} - \frac{i}{2} \Gamma_{12}}
\end{equation}
%
DM is by definition positive, the sign of DG has to be measured

\subsection{Time evolution of meson states}
\label{sec:cpv_theory:flavour_physics:time_evolution}

time evolution of meson states
%
\begin{equation}
  \begin{split}
    \ket{\Meson (t)}    &= g_{+}(t) \ket{\Meson}    - \frac{q}{p} g_{-}(t) \ket{\Mesonbar} \\
    \ket{\Mesonbar (t)} &= g_{+}(t) \ket{\Mesonbar} - \frac{p}{q} g_{-}(t) \ket{\Meson}
  \end{split}
\end{equation}
%
with 
%
\begin{equation}
  g_{\pm}(t) \equiv 
  \frac{1}{2} \left[
    \exponential{-i \mu_H t} \pm \exponential{-i \mu_L t}
  \right]
  =
  \frac{1}{2} \left[
    \exponential{-i m_H t} \exponential{-\frac{i}{2} \Gamma_H t} \pm \exponential{-i m_L t} \exponential{-\frac{i}{2} \Gamma_L t}
  \right]
\end{equation}
%
and the definitions of lambda
%
\begin{equation}
  \lambda_f               \equiv \frac{1}{\ovE{\lambda}_f}   = \frac{q}{p} \frac{\ovE{A}_f}{A_f} \eqand 
  \ovE{\lambda}_{\ovE{f}} \equiv \frac{1}{\lambda_{\ovE{f}}} = \frac{p}{q} \frac{A_{\ovE{f}}}{\ovE{A}_{\ovE{f}}}
\end{equation}
%
the decay rate of a meson M produced at t=0 to final state f at time t is given by
%
\begin{equation}
  \Gamma \left(\Meson(t)\to f\right)= \bigl\vert \matrixelement{f}{T}{\Meson} \bigr\vert^2
\end{equation}
%
then
%
\begin{equation}
  \begin{split}
    \frac{\Gamma \bigl(\Meson   (t) \to       f \bigr)}{\exponential{-\Gamma t}} &= 
      \frac{1}{2} \abs{A_f}^2 \left( 1 + \abs{\lambda_f}^2 \right) \\
        \Biggl[ \cosh&\left(\frac{\DG t}{2}\right) + D_f \sinh\left(\frac{\DG t}{2}\right) + C_f \cos(\DM t) - S_f \sin(\DM t) \Biggr] \\
    \frac{\Gamma \bigl(\Meson   (t) \to \ovE{f} \bigr)}{\exponential{-\Gamma t}} &= 
      \frac{1}{2} \abs{\ovE{A}_{\ovE{f}}}^2 \left( 1 + \abs{\ovE{\lambda}_{\ovE{f}}}^2 \right) \abs{\frac{q}{p}}^2 \\
        \Biggl[ \cosh&\left(\frac{\DG t}{2}\right) + D_f \sinh\left(\frac{\DG t}{2}\right) + C_f \cos(\DM t) - S_f \sin(\DM t) \Biggr] \\
    \frac{\Gamma \bigl(\Mesonbar(t) \to       f \bigr)}{\exponential{-\Gamma t}} &= 
      \frac{1}{2} \abs{A_f}^2 \left( 1 + \abs{\lambda_f}^2 \right) \abs{\frac{p}{q}}^2 \\
        \Biggl[ \cosh&\left(\frac{\DG t}{2}\right) + D_f \sinh\left(\frac{\DG t}{2}\right) + C_f \cos(\DM t) - S_f \sin(\DM t) \Biggr] \\
    \frac{\Gamma \bigl(\Mesonbar(t) \to \ovE{f} \bigr)}{\exponential{-\Gamma t}} &= 
      \frac{1}{2} \abs{\ovE{A}_{\ovE{f}}}^2 \left( 1 + \abs{\ovE{\lambda}_{\ovE{f}}}^2 \right) \\
        \Biggl[ \cosh&\left(\frac{\DG t}{2}\right) + D_f \sinh\left(\frac{\DG t}{2}\right) + C_f \cos(\DM t) - S_f \sin(\DM t) \Biggr] \\
  \end{split}
\end{equation}
%
where D, C, and S are the CP coefficients defined as
%
\begin{equation}
  \begin{alignedat}{2}
    &D_f = \frac{2 \Re \lambda_f}{1+\abs{\lambda_f}^2}, \eqspace 
    &C_f = \frac{1 - \abs{\lambda_f}^2}{1 + \abs{\lambda_f}^2}, \eqspace 
    &S_f = \frac{2 \Im \lambda_f}{1+\abs{\lambda_f}^2} \\
    &D_{\ovE{f}} = \frac{2 \Re \ovE{\lambda}_{\ovE{f}}}{1+\abs{\ovE{\lambda}_{\ovE{f}}}^2}, \eqspace 
    &C_{\ovE{f}} = \frac{1 - \abs{\ovE{\lambda}_{\ovE{f}}}^2}{1 + \abs{\ovE{\lambda}_{\ovE{f}}}^2}, \eqspace 
    &S_{\ovE{f}} = \frac{2 \Im \ovE{\lambda}_{\ovE{f}}}{1+\abs{\ovE{\lambda}_{\ovE{f}}}^2} \\
  \end{alignedat}
\end{equation}
%
which satisfy $D^2+C^2+S^2 = 1$

\subsection[Classification of \CP violating effects]{Classification of \CPbfsf violating effects}
\label{sec:cpv_theory:flavour_physics:cpv_classification}

\subsubsection[Direct \CP violation]{Direct \CPbfsf violation}
\label{sec:cpv_theory:flavour_physics:cpv_classification:direct}

definition of CPV in decay aka direct CPV
%
\begin{equation}
  \frac{\ovE{A}_{\ovE{f}}}{A_f} \neq 1
\end{equation}
%
only possible source of CPV in charged mesons and all baryons, where mixing is absent

\subsubsection[\CP violation in mixing]{\CPbfsf violation in mixing}
\label{sec:cpv_theory:flavour_physics:cpv_classification:mixing}

CPV in mixing is defined by
%
\begin{equation}
  \abs{\frac{q}{p}} \neq 1
\end{equation}

\subsubsection[\CP violation in the interference between decays with and without mixing]{\CPbfsf violation in the interference between decays with and without mixing}
\label{sec:cpv_theory:flavour_physics:cpv_classification:interference}

CPV in the interference of the amplitudes of the decay and the decay after mixing
%
\begin{equation}
  \Im \lambda_f \neq 0   
\end{equation}
%
definition of the CP asymmetry of a \Meson into a common \CP final state
%
\begin{equation}
  \CPAsymmetry(t) = \frac{\Gamma \bigl(\Mesonbar (t) \to f_{\text{\CP}}\bigr) - \Gamma \bigl(\Meson (t) \to f_{\text{\CP}}\bigr)}
                         {\Gamma \bigl(\Mesonbar (t) \to f_{\text{\CP}}\bigr) + \Gamma \bigl(\Meson (t) \to f_{\text{\CP}}\bigr)}
\end{equation}


