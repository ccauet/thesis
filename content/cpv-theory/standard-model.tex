%!TEX root = ../../common/main.tex

\section{The \acl*{SM}}
\label{sec:cpv_theory:standard_model}

The \acf{SM} describes the constituents of matter and the fundamental forces.
The theoretical model is rigorously tested by experiments and up to date any
attempts to falsify even a part of it were unsuccessful. From the beginning of
the 1960s until the discovery of the Higgs boson \cite{Aad:2015zhl} all
experimental evidence found, is in support of the \SM. Nevertheless, the \SM
must be part of some bigger universal theory. The absence of a suitable dark
matter candidate, a missing predictive model explaining the quark and lepton
mass hierarchy and mixing, and the lack of an explanation of the Higgs hierarchy
problem, are just a selection of questions that are not answered by the \SM.

The theoretical framework of the \SM is built as a quantum field theory
describing the dynamics and kinematics of the theory. Based on the gauge group
$\group{SU}{3}\otimes\group{SU}{2}\otimes\group{U}{1}$ the \SM Lagrangian
$\mathcal{L}_{\text{\acs*{SM}}}$ can describe all\footnote{The \ac{SM} does not
describe gravity, which makes in not less attractive from the perspective of
high energy particle physics.} known fundamental forces and their bosonic force
mediators, the fermionic particles forming matter, as well as the scalar field
that lends masses to the bosons and fermions of the \SM trough the mechanism of
spontaneous symmetry breaking.

The fermions split into quarks and leptons each composed into three generations
that can be listed by their left-handed doublets
%
\begin{gather*}\label{eq:cpv_theory:standard_model:quarks_and_leptons}
  \begin{pmatrix}
    \electron \\
    \nuel
  \end{pmatrix}_{L}\eqspace
  \begin{pmatrix}
    \muon \\
    \numu
  \end{pmatrix}_{L}\eqspace
  \begin{pmatrix}
    \tauon \\
    \nutau
  \end{pmatrix}_{L} \\
  \begin{pmatrix}
    \uquark \\
    \dquark
  \end{pmatrix}_{L}\eqspace
  \begin{pmatrix}
    \cquark \\
    \squark
  \end{pmatrix}_{L}\eqspace
  \begin{pmatrix}
    \tquark \\
    \bquark
  \end{pmatrix}_{L}
\end{gather*}
%
as well as their right-handed singlet counterparts
%
\begin{equation*}\label{eq:cpv_theory:standard_model:right_handed}
  \electron_{R},\eqthinspace \muon_{R},\eqthinspace \tauon_{R},\eqthinspace \uquark_{R},\eqthinspace \dquark_{R},\eqthinspace \cquark_{R},\eqthinspace \squark_{R},\eqthinspace \tquark_{R},\eqand \bquark_{R}\eqpd
\end{equation*}
%
The $\photon$ as the carrier of the electro-magnetic force couples to all
particles with a non-zero electro-magnetic charge $q$, \ie the charged leptons
$\electron$, $\muon$, and $\tauon$ and the quarks. The weak force's intermediate bosons

