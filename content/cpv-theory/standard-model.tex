%!TEX root = ../../common/main.tex

\section{The \acl*{SM}}
\label{sec:cpv_theory:standard_model}

The \acf{SM} tries to describe the constituents of matter and anti-matter as
well as their interactions. The theoretical model has been rigorously tested by
experiments and any attempts to falsify even a part of it have been unsuccessful
up to date. From the beginning of the 1960s until the discovery of the Higgs
boson \cite{Aad:2015zhl} all experimental evidence has been in support of the
\SM. Nevertheless, the \SM must be part of some bigger universal theory: The
absence of a suitable dark matter candidate, a missing predictive model
explaining the quark and lepton mass hierarchy and mixing, and the absence of
anti-matter/the abundance of matter in the visible universe, are just a
selection of questions that are not answered by the \SM.

The theoretical framework of the \SM is built as a Lorentz invariant quantum
field theory describing the dynamics and kinematics of the theory. Based on the
gauge group $\group{SU}{3}\otimes\group{SU}{2}\otimes\group{U}{1}$ the \SM
Lagrangian $\mathcal{L}_{\text{\acs*{SM}}}$ describes all\footnote{The \ac{SM}
does not describes gravity, which makes it not less attractive from the
perspective of high energy particle physics as the strength of the gravitational
couplings is several orders of magnitude smaller compared to the other forces.}
known fundamental forces and their bosonic force mediators, the fermionic
particles forming matter, as well as the scalar field that lends masses to the
bosons and fermions of the \SM through the mechanism of spontaneous symmetry
breaking.

The fermionic fields can be expressed by their left-handed doublets of quarks 
%
\begin{equation*}\label{eq:cpv_theory:standard_model:quarks}
  \begin{pmatrix}
    \uquark \\
    \dquark^\prime
  \end{pmatrix}_{\text{L}}\eqspace
  \begin{pmatrix}
    \cquark \\
    \squark^\prime
  \end{pmatrix}_{\text{L}}\eqspace
  \begin{pmatrix}
    \tquark \\
    \bquark^\prime
  \end{pmatrix}_{\text{L}}
\end{equation*}
%
and leptons
%
\begin{equation*}\label{eq:cpv_theory:standard_model:leptons}
  \begin{pmatrix}
    \nuel \\
    \electron
  \end{pmatrix}_{\text{L}}\eqspace
  \begin{pmatrix}
    \numu \\
    \muon
  \end{pmatrix}_{\text{L}}\eqspace
  \begin{pmatrix}
    \nutau \\
    \tauon
  \end{pmatrix}_{\text{L}}
\end{equation*}
%
as well as their right-handed singlet counterparts
%
\begin{equation*}\label{eq:cpv_theory:standard_model:right_handed}
  \electron_{\text{R}},\eqthinspace \muon_{\text{R}},\eqthinspace \tauon_{\text{R}},\eqthinspace \uquark_{\text{R}},\eqthinspace \dquark_{\text{R}},\eqthinspace \cquark_{\text{R}},\eqthinspace \squark_{\text{R}},\eqthinspace \tquark_{\text{R}},\eqand \bquark_{\text{R}}\eqpd
\end{equation*}
%
The \SM does not include right-handed neutrinos, yet they might be necessary to
explain the observed non-vanishing neutrino masses. To each listed particle, an
anti-particle with oppositely signed charge quantum numbers exist, usually
denoted with an overbar, \eg $\quarkbar$.

The fundamental interactions are represented by the \EM force, the weak force,
and the strong force. The \EM and weak interactions can be unified into the so
called electro-weak interaction \cite{set:gws}, represented by an
$\group{SU}{2}\otimes\group{U}{1}$ gauge group. Four corresponding massless
gauge bosons are present that can be associated with the $\group{SU}{2}$ and
$\group{U}{1}$ group. After the spontaneous breaking of the symmetry at the
electro-weak energy scale the massless photon $\photon$ and the massive $\Wpm$
and $\Zboson$ bosons emerge.

The photon couples to particles with non-zero \EM charge, \ie the charged
leptons $\electron$, $\muon$, and $\tauon$ and the quarks. The likewise
uncharged \Zboson additionally couples to the uncharged neutrinos. The bosons
$\Wpm$ act as mediators between the upper and lower fields of the fermionic
doublets enabling \eg the radioactive $\beta$ decay with a
$\dquark\to\uquark\electron\nuelbar$ transition through the exchange of a $\Wm$.

Finally, the strong force couples to the colour charge carried solely by quarks
and is mediated by gluons. Quarks cannot be observed in unbound states, thus
always form composite particles (\emph{hadrons}) as \eg the proton consists of
two $\uquark$-quarks and one $\dquark$-quark. In general, particles composed of
three quarks ($\quark\quark\quark$) or three anti-quarks
($\quarkbar\quarkbar\quarkbar$) are called \emph{baryons}, particles with a
quark and an anti-quark as constituents ($\quark\quarkbar$) are called
\emph{mesons}. Besides these two prevalent bound states, exotic
\emph{pentaquark} states composed of four quarks and one anti-quark were
experimentally confirmed \cite{Aaij:2015tga} recently.

All visible matter consists entirely of first generation quarks ($\uquark$ and
$\dquark$) and leptons ($\electron$), thus no other fermionic particles are
mandatory to explain most physical phenomena and moreover it seems non-essential
that they exist at all. To shed light into this puzzle the structure and
hierarchy of the quark and lepton sector is studied in the field of
\emph{flavour physics}.
