%!TEX root = ../../common/main.tex

\section{The \acl*{SM}}
\label{sec:cpv_theory:standard_model}

The \acf{SM} describes the constituents of matter and the fundamental forces.
The theoretical model is rigorously tested by experiments and any attempts to
falsify even a part of it were unsuccessful up to date. From the beginning of
the 1960s until the discovery of the Higgs boson \cite{Aad:2015zhl} all
experimental evidence found is in support of the \SM. Nevertheless, the \SM must
be part of some bigger universal theory. The absence of a suitable dark matter
candidate, a missing predictive model explaining the quark and lepton mass
hierarchy and mixing, and the absence of anti-matter/abundance of matter in the
visible universe, are just a selection of questions that are not answered by the
\SM.

The theoretical framework of the \SM is built as a quantum field theory
describing the dynamics and kinematics of the theory. Based on the gauge group
$\group{SU}{3}\otimes\group{SU}{2}\otimes\group{U}{1}$ the \SM Lagrangian
$\mathcal{L}_{\text{\acs*{SM}}}$ can describe all\footnote{The \ac{SM} does not
describes gravity, which makes in not less attractive from the perspective of
high energy particle physics.} known fundamental forces and their bosonic force
mediators, the fermionic particles forming matter, as well as the scalar field
that lends masses to the bosons and fermions of the \SM trough the mechanism of
spontaneous symmetry breaking.

The fermionic fields can be expressed by their left-handed doublets of quarks 
%
\begin{equation*}\label{eq:cpv_theory:standard_model:quarks}
  \begin{pmatrix}
    \uquark \\
    \dquark^\prime
  \end{pmatrix}_{L}\eqspace
  \begin{pmatrix}
    \cquark \\
    \squark^\prime
  \end{pmatrix}_{L}\eqspace
  \begin{pmatrix}
    \tquark \\
    \bquark^\prime
  \end{pmatrix}_{L}
\end{equation*}
%
and leptons
%
\begin{equation*}\label{eq:cpv_theory:standard_model:leptons}
  \begin{pmatrix}
    \nuel \\
    \electron
  \end{pmatrix}_{L}\eqspace
  \begin{pmatrix}
    \numu \\
    \muon
  \end{pmatrix}_{L}\eqspace
  \begin{pmatrix}
    \nutau \\
    \tauon
  \end{pmatrix}_{L}
\end{equation*}
%
as well as their right-handed singlet counterparts
%
\begin{equation*}\label{eq:cpv_theory:standard_model:right_handed}
  \electron_{R},\eqthinspace \muon_{R},\eqthinspace \tauon_{R},\eqthinspace \uquark_{R},\eqthinspace \dquark_{R},\eqthinspace \cquark_{R},\eqthinspace \squark_{R},\eqthinspace \tquark_{R},\eqand \bquark_{R}\eqpd
\end{equation*}
%
The \SM does not comprises right-handed neutrinos, yet these might be necessary
to explain the observed non-vanishing neutrino masses. To each listed particle,
an anti-particle with opposite signed charge quantum numbers exist, usually
denoted with an overbar, \eg $\quarkbar$.

The fundamental interactions are represented by the \EM force, the weak force,
and the strong force. The \EM and weak interactions can be unified in the so
called electro-weak interaction \cite{set:gws}, represented by an
$\group{SU}{2}\otimes\group{U}{1}$ gauge group. Four corresponding massless
gauge bosons are present that can be associated with the $\group{SU}{2}$ and
$\group{U}{1}$ group. After the spontaneous breaking of the symmetry at the
electro-weak energy scale the massless photon $\photon$ and the massive $\Wpm$
and $\Zboson$ bosons emerge. The photon couples to particles with non-zero \EM
charge, \ie the charged leptons $\electron$, $\muon$, and $\tauon$ and the
quarks. The likewise uncharged \Zboson additionally couples to the uncharged
neutrinos. The bosons $\Wpm$ act as mediator between the upper and lower parts
of the fermionic doublets enabling \eg the radioactive $\beta$ decay with a
$\dquark\to\uquark\electron\nuelbar$ transition trough the exchange of a $\Wm$.
Finally, the strong force couples to the so called colour charge carried solely
by quarks and is mediated by gluons. Quarks cannot be observed in unbound
states, thus always form composite particles (\emph{hadrons}) as \eg the proton
consists of two $\uquark$-quarks and one $\dquark$-quark. In general, particles
composed of three quarks ($\quark\quark\quark$) or three anti-quarks
($\quarkbar\quarkbar\quarkbar$) are called \emph{baryons}, particles with a
quark and an anti-quark as constituents ($\quark\quarkbar$) are called
\emph{mesons}.



\missing{Discrete C,P, and T symmetries}
\info{Free parameters in SM?}

The visible matter consists entirely of first generation quarks ($\uquark$ and
$\dquark$) and leptons ($\electron$), thus no other particles are mandatory to
explain most physical phenomena and moreover it seems non-essential that they
exist at all. To shed light into this puzzle the structure and hierarchy of the
quark and lepton sector is studied in the field of \emph{flavour physics}.
