%!TEX root = ../../common/main.tex

\chapter{Flavour Tagging}
\label{ch:flavour_tagging}

The time-dependent measurement of the $\CP$ asymmetry $\CPAsymmetry$ in the
decay of $\Bd$ and $\Bdbar$ mesons into the $\CP$ eigenstate $\Jpsi\KS$ requires
the knowledge of the $\B$ meson flavour at production, \ie weather it contained
a $\bquark$ or a $\bquarkbar$. The method and algorithms used to infer this
information from all available event properties are named \enquote{flavour
tagging}.

Each tagging algorithm provides a tag $\tagdecision$ and a probability estimate
$\mistagestimate$ that the assigned tag is wrong, also called \enquote{mistag}
estimate. The tag is $\tagdecision=+1$ for an initial $\Bd$, $\tagdecision=-1$
for an initial $\Bdbar$, and $\tagdecision=0$ if the tagging algorithms were not
able to determine a decision. The mistag estimate interval is given by
$[0,0.5[$, where mistags $\mistagestimate^{\prime}>0.5$ are evaluated as
$\mistagestimate = 1 - \mistagestimate^{\prime}$ and the corresponding tag's
sign is reversed. If the algorithm is not able to determine a decision, the
mistag is set $\mistagestimate=0.5$.

The performance of the flavour tagging algorithms can be assigned using control
samples of $\B$ mesons whose final state determines the $\B$ flavour at decay
time (\ie are \enquote{flavour specific}), \eg $\BuToJpsiK$ where the charge of
the kaon allows to infer the charge of the $\B$.

Given the numbers of all right tagged $\B$ candidates $\NRtagged$, all wrong
tagged candidates $\NWtagged$, and all \enquote{untagged} candidates
$\NUtagged$, the \enquote{tagging efficiency} $\tageff$ can be assigned
%
\begin{equation}\label{eq:flavour_tagging:tageff}
  \tageff = \frac{\NRtagged + \NWtagged}{\NRtagged + \NWtagged + \NUtagged}\eqpd
\end{equation}
%
In addition to the mistag estimate, the true underlying fraction of wrong tagged
candidates $\mistag$ is given by
%
\begin{equation}\label{eq:flavour_tagging:mistag}
  \mistag = \frac{\NWtagged}{\NRtagged + \NWtagged}\eqpd
\end{equation}
%
The statistically relevant \acl{FoM} for time-dependent measurements of $\CP$
violation\addref{for $\efftageff$ for CPV measurements}, the \enquote{effective
tagging efficiency}
%
\begin{equation}\label{eq:flavour_tagging:efftageff}
  \efftageff = \tageff(1 - 2 \mistag)^2 = \tageff \tagdilution^2 \eqcm
\end{equation}
%
represents the fraction of candidates necessary to reach the same statistical
power if the tagging would be perfect, and hence should be maximized to achieve
the best performance. The quantity $\tagdilution$ is called \enquote{dilution},
taking a value of $1$ in case of perfect tagging, and $0$ in case of random
tagging. As described in \cref{missing} the dilution can be directly translated
to the amplitude of the measured $\CP$ asymmetry. \missing{Effect of dilution on CPV asymmetry}

In the following sections more details of the flavour tagging are provided. A
short introduction of the utilized algorithms and a comparison to the flavour
tagging used at lepton colliders is given in \cref{sec:flavour_tagging:lhcb}.
\Cref{sec:flavour_tagging:os,sec:flavour_tagging:ss} describe the two different
classes of flavour tagging algorithms used at \LHCb, while the calibration of
the tagging algorithms outputs is described in
\cref{sec:flavour_tagging:calibration}. The combination of different algorithm
responses into a single decision is outlined in
\cref{sec:flavour_tagging:combination}. Performance numbers are provided in
\cref{sec:flavour_tagging:performance}, recent developments are discussed in
\cref{sec:flavour_tagging:developments}, and finally the experimental impact of
the flavour tagging on the measurement of $\CP$ violation in $\BdToJpsiKS$
decays is provided in \cref{sec:flavour_tagging:sin2beta}.

More details on the flavour tagging used by \LHCb can be found in
\cite{Aaij:2012mu}.

\section{Tagging at LHCb}
\label{sec:flavour_tagging:lhcb}

Several tagging algorithms, each specialized on different characteristics of the
underlying event, are used in order to determine the $\B$ meson flavour at
production. The algorithms can be classified into two types: the \SS tagging and
\OS tagging algorithms, also called \SS/\OS taggers. The \SS taggers infer the
production flavour of the signal $\B$ meson by identifying charged candidates
that have a high chance of being remnants of its hadronisation process. On the
other hand, the \OS taggers exploit the dominant production of $\B$ mesons
through $\bbbar$ quark pair production, allowing to partially reconstruct the
$\bquark$ hadron produced together with each reconstructed signal $\B$ meson and
thereby infer its initial flavour. \Cref{fig:flavour_tagging:lhcb:schematics}
gives an overview of the tagging algorithms and
\cref{sec:flavour_tagging:os,sec:flavour_tagging:ss} contain a more detailed
discussion of the \OS and \SS tagging algorithms.

\begin{figure}
\centering
%!TEX root = ../../../common/main.tex

% Colours 
\definecolor{fcdOrnA}{HTML}{331605}
\definecolor{fcdOrnB}{HTML}{662C0A}
\definecolor{fcdOrnC}{HTML}{99420F}
\definecolor{fcdOrnD}{HTML}{CC5814}
\definecolor{fcdOrnE}{HTML}{FF6E19}
\definecolor{fcdOrnF}{HTML}{FF975B}
\definecolor{fcdOrnG}{HTML}{FFAC7C}
\definecolor{fcdOrnH}{HTML}{FFC19C}
\definecolor{fcdOrnI}{HTML}{FFD6BD}
\definecolor{fcdOrnJ}{HTML}{FFEADE}
\definecolor{fcdBluA}{HTML}{052A33}
\definecolor{fcdBluB}{HTML}{0A5466}
\definecolor{fcdBluC}{HTML}{0F7E99}
\definecolor{fcdBluD}{HTML}{14A8CC}
\definecolor{fcdBluE}{HTML}{19D2FF}
\definecolor{fcdBluF}{HTML}{5BDFFF}
\definecolor{fcdBluG}{HTML}{7CE5FF}
\definecolor{fcdBluH}{HTML}{9CECFF}
\definecolor{fcdBluI}{HTML}{9CECFF}
\definecolor{fcdBluJ}{HTML}{DEF9FF}
\definecolor{fcdGrnA}{HTML}{243304}
\definecolor{fcdGrnB}{HTML}{476608}
\definecolor{fcdGrnC}{HTML}{6B990D}
\definecolor{fcdGrnD}{HTML}{A0E02D}
\definecolor{fcdGrnE}{HTML}{B2FF15}
\definecolor{fcdGrnF}{HTML}{C8FF58}
\definecolor{fcdGrnG}{HTML}{D3FF79}
\definecolor{fcdGrnH}{HTML}{DEFF9B}
\definecolor{fcdGrnI}{HTML}{E9FFBC}
\definecolor{fcdGrnJ}{HTML}{F4FFDE}
\definecolor{fcdVltA}{HTML}{310433}
\definecolor{fcdVltB}{HTML}{620866}
\definecolor{fcdVltC}{HTML}{930D99}
\definecolor{fcdVltD}{HTML}{C411CC}
\definecolor{fcdVltE}{HTML}{F514FF}
\definecolor{fcdVltF}{HTML}{F858FF}
\definecolor{fcdVltG}{HTML}{F979FF}
\definecolor{fcdVltH}{HTML}{FB9BFF}
\definecolor{fcdVltI}{HTML}{FCBCFF}
\definecolor{fcdVltJ}{HTML}{FEDDFF}

\definecolor{fcdGrayA}{HTML}{111111}
\definecolor{fcdGrayB}{HTML}{222222}
\definecolor{fcdGrayC}{HTML}{333333}
\definecolor{fcdGrayD}{HTML}{444444}
\definecolor{fcdGrayE}{HTML}{555555}
\definecolor{fcdGrayF}{HTML}{666666}
\definecolor{fcdGrayG}{HTML}{777777}
\definecolor{fcdGrayH}{HTML}{888888}
\definecolor{fcdGrayI}{HTML}{999999}
\definecolor{fcdGrayJ}{HTML}{AAAAAA}
\definecolor{fcdGrayK}{HTML}{BBBBBB}
\definecolor{fcdGrayL}{HTML}{CCCCCC}
\definecolor{fcdGrayM}{HTML}{DDDDDD}
\definecolor{fcdGrayN}{HTML}{EEEEEE}

\definecolor{fcdTropiteal}    {HTML}{00A8C6}
\definecolor{fcdTealDrop}     {HTML}{40C0CB}
\definecolor{fcdWhiteTrash}   {HTML}{F9F2E7}
\definecolor{fcdAtomicBikini} {HTML}{AEE239}
\definecolor{fcdFeebleWeek}   {HTML}{8FBE00}





\colorlet{ClrTxt}{black}
\colorlet{ClrTxtVeryDarkGray}{fcdGrayE}
\colorlet{ClrTxtDarkGray}{fcdGrayJ}
\colorlet{ClrVtxGray}{fcdGrayM}

\colorlet{ClrSigQuark}{fcdTealDrop}
\colorlet{ClrSigMeson}{fcdTropiteal}
\colorlet{ClrSigArrow}{fcdTropiteal}

\colorlet{ClrTagQuark}{fcdAtomicBikini}
\colorlet{ClrTagMeson}{fcdFeebleWeek}
\colorlet{ClrTagArrow}{fcdFeebleWeek}

\begin{tikzpicture}[
  scale=1, 
  >=stealth',
  font=\small,
  quark_sig/.style={
    align=center, 
    minimum size=3ex,
    circle,
    color=ClrSigQuark,
    fill=ClrSigQuark,
    text=ClrTxt,
    draw, 
    thick,
    inner sep=0pt,
    outer sep=0pt,
    node distance=0ex
  },
  quark_tag/.style={
    align=center, 
    minimum size=3ex,
    circle,
    color=ClrTagQuark,
    fill=ClrTagQuark,
    text=ClrTxt,
    draw, 
    thick,
    inner sep=0pt,
    outer sep=0pt,
    node distance=0ex
  },
  meson_sig/.style={
    draw, 
    align=center, 
    minimum size=4.5ex,
    circle,
    color=ClrSigMeson,
    fill=ClrSigMeson,
    text=ClrTxt,
    thick,
    inner sep=0pt,
    outer sep=1pt,
    node distance=0ex
  },
  meson_tag/.style={
    draw, 
    align=center, 
    minimum size=4.5ex,
    circle,
    color=ClrTagMeson,
    fill=ClrTagMeson,
    text=ClrTxt,
    thick,
    inner sep=0pt,
    outer sep=1pt,
    node distance=0ex
  },
  meson_comb/.style={
    shape=ellipse, 
    draw,
    fill,
    very thick,
    inner sep=0pt,
    outer sep=0pt,
    minimum width=8ex,
    minimum height=5ex},
  vertex/.style={
    shape=ellipse,
    draw,
    inner sep=1ex,
    outer sep=0pt,
    minimum width=10ex,
    minimum height=10ex,
    color=ClrVtxGray,
    fill=ClrVtxGray,
    text=ClrTxt,
    node distance=1ex
  },
  vertex_label/.style={
    inner sep=0pt,
    outer sep=0pt,
    text=ClrTxtDarkGray,
    node distance=1ex,
    font=\sffamily\small
  },
  tagger_label/.style={
    inner sep=0pt,
    outer sep=0pt,
    text=ClrTxtVeryDarkGray,
    node distance=1ex,
    font=\sffamily\small    
  },
  arrow_sig/.style={
    ->,
    very thick,
    color=ClrSigArrow
  },
  arrow_tag/.style={
    ->,
    very thick,
    color=ClrTagArrow
  }
]

%\draw[help lines] (-1,-5) grid (11,5);

\draw[dashed,color=fcdGrayE] (0,0) -- (12,0);
\node[text width=2.5cm,text=ClrTxtDarkGray,font=\sffamily\small] (SST) at (10.5,+0.3) {\hfill same side};
\node[text width=2.5cm,text=ClrTxtDarkGray,font=\sffamily\small] (OST) at (10.5,-0.3) {\hfill opposite side};


\node[draw,circle,fill,color=ClrSigQuark,inner sep=0pt,minimum size=8pt] 
  (coll) at (0,0) {};

\draw[<-,very thick] (coll) -- (+1,0);
\draw[->,very thick] (-1,0) -- (coll);

\begin{pgfonlayer}{foreground}
  
  % bbbar
  \node[quark_sig] (qrk_bbar) at (0.2,+0.8) {$\bquarkbar$};
  \node[quark_sig] (qrk_b) at (0.2,-0.8) {$\bquark$};
  
  \path (qrk_bbar) 
        to[circle connection bar switch color=from (ClrSigQuark) to (ClrSigQuark)] 
        (coll);
  \path (qrk_b) 
        to[circle connection bar switch color=from (ClrSigQuark) to (ClrSigQuark)] 
        (coll);
  
  % Signal decay
  \node[meson_sig,color=ClrSigMeson,fill=ClrSigMeson,text=ClrTxt] (SigJpsi) at (7.5,+2.25) {$\Jpsi$}  ;
  \node[meson_sig,color=ClrSigMeson,fill=ClrSigMeson,text=ClrTxt] (SigKS) [below=of SigJpsi] {$\KS$}  ;
  
  % SS tagging
  \path (qrk_bbar) ++(15:3ex) node (qrk_d) [quark_tag] {$\dquark$};
  
  \node[quark_tag] (qrk_dbar) at (0.8,+2.2) {$\dquarkbar$};
  
  \node[draw,circle,fill,color=ClrTagQuark,inner sep=0pt,minimum size=4pt] 
    (vacuumexc) at ([xshift=-2ex] $(qrk_dbar)!0.5!(qrk_d)$) {};
  
  \path (qrk_d) 
        to[circle connection bar switch color=from (ClrTagQuark) to (ClrTagQuark)] 
        (vacuumexc);
  \path (qrk_dbar) 
        to[circle connection bar switch color=from (ClrTagQuark) to (ClrTagQuark)] 
        (vacuumexc);  
  \path (qrk_dbar) ++(165:3ex) node (qrk_u)    [quark_tag] {$\uquark$};

  \node[meson_tag] (SSpip) at (3.3,+2.7) {$\pip$}; 
  

  % OS tagging
  \path (qrk_b)    ++(-15:3ex) node (qrk_xbar) [quark_tag] {$\quarkbar$};  

  % Tagging particles 
  \node[meson_tag] (OSlepton) at (9.5,-2.25) {$\lepm$};
  \node[meson_tag] (OSkaon)   at (9.5,-1)    {$\Km$};
  
\end{pgfonlayer} 
  
  
\node[meson_comb,rotate=+15,color=ClrSigMeson,text=ClrTxt] (SigBz) at ($(qrk_bbar)!0.5!(qrk_d)$) {  }; 
\node[meson_comb,rotate=-15,color=ClrTagMeson,text=ClrTxt] (Hb) at ($(qrk_b)!0.5!(qrk_xbar)$) {}; 
\node[meson_comb,rotate=-15,color=ClrTagMeson,text=ClrTxt] (pip_SS) at ($(qrk_dbar)!0.5!(qrk_u)$)   {}; 

  
\begin{pgfonlayer}{background}
  \node[vertex,fit=(SigBz)(Hb)(pip_SS),minimum width=16ex] (PV) {};
  \node[vertex_label,align=center] (PV_label) [above=of PV] {PV};
  
  \node[vertex,fit=(SigJpsi)(SigKS)   ,minimum width=10ex] (SigSV) {};
  \node[vertex_label,align=center] (SigSV_label) [above=of SigSV] {SV};
  
  \node[vertex,minimum width=11ex,minimum height=9ex,align=left] (OS_SV) at (4.5,-1.75) {$\bquark \to \cquark$\\  $\bquark \to X \lepm$};
  \node[vertex_label,align=center] (OS_SV_label) [above=of OS_SV] {SV};
  
  
  \node[vertex,minimum width=9ex,minimum height=7ex,align=center] (OS_TV) at (7.5,-1.25) {$\cquark \to \squark$};
\end{pgfonlayer}


\begin{pgfonlayer}{foreground}
  \draw[arrow_sig] (SigBz)  -- node (Bz_label) [above] {$\Bd$} ([xshift=+4pt] SigSV.west);
  \draw[arrow_sig] let \p1 =(SigJpsi.east) in (\x1-1,\y1+2)  -- (\x1+50,\y1+5);
  \draw[arrow_sig] let \p1 =(SigJpsi.east) in (\x1-1,\y1-2)  -- (\x1+50,\y1-5);
  \draw[arrow_sig] let \p1 =(SigKS.east)   in (\x1-1,\y1+2)  -- (\x1+50,\y1+5);
  \draw[arrow_sig] let \p1 =(SigKS.east)   in (\x1-1,\y1-2)  -- (\x1+50,\y1-5);
    
  \draw[arrow_tag] (Hb)     -- node (Hb_label) [above] {$\hb$}  ([xshift=+4pt] OS_SV.west);
  \draw[arrow_tag] (OS_SV)  -- ([xshift=+4pt] OS_TV.west);
  \draw[arrow_tag] (OS_SV)  -- ([xshift=+0pt] OSlepton.west);
  \draw[arrow_tag] (OS_TV)  -- ([xshift=+0pt] OSkaon.west);
  \draw[arrow_tag] (pip_SS) -- ([xshift=+0pt] SSpip.west);


  \node[tagger_label,align=left] (SSpip_label) [right=of SSpip] {SS pion};
  \node[tagger_label,align=left] (OSlepton_label) [right=of OSlepton] {OS muon\\ OS electron};
  \node[tagger_label,align=left] (OSkaon_label)   [right=of OSkaon] {OS kaon};
  \node[tagger_label,align=center] (OSVtcCh_label) [below=of OS_SV] {OS vertex charge};

\end{pgfonlayer}

\end{tikzpicture}

\caption{Schematic overview of the used \acs*{OS} and \acs*{SS} tagging
algorithms. \cite{wishahi:2013jt}}
\label{fig:flavour_tagging:lhcb:schematics}
\end{figure}

\subsection*{Flavour tagging at \Babar and \Belle}
\label{sec:flavour_tagging:lhcb:b_factories}
Caused by the different nature of the experimental setup the tagging method used
at \LHCb differs from methods used at the \BFactories. As described in
\cref{sec:lhcb_experiment:detector} the $\bquark$ hadron production is dominated
by gluon-gluon fusion over a large range of $q^2$ in contrast to the production
at the $\YFourS$ $\bbbar$ resonance as it is the case at the \BFactories. The
following section gives a résumé of the flavour tagging methods employed by the
\Babar and \Belle collaborations outlined in \Ref~\cite[][Ch. 8]{Bevan:2014iga}.
%
\begin{figure}
\centering
\includegraphics[width=\textwidth]{private/content/flavour-tagging/figs/b_factory_basic_principles}
\caption{Illustration of the measurement of \sintwobeta as conducted by the
\BFactories. \cite{Bevan:2014iga}}
\label{fig:flavour_tagging:lhcb:b_factory_basic_principles}
\end{figure}

\Cref{fig:flavour_tagging:lhcb:b_factory_basic_principles} illustrates the basic
principles of a \CP violation measurement at the \BFactories. The produced
$\Bd\Bdbar$ pair's wave function is in a $P$-wave entangled state, until one of
the mesons decays. From this point in space time ($t_1$) the second $\B$ mesons
propagates further through the detector, mixes to its antimatter state, and
finally decays at $t_2$. The \CP asymmetry $\CPAsymmetry$ will therefore be a
function of the decay time difference $\Delta t$. This has two implications:
First, the decay time difference will always be computed relative to the
\enquote{tagging} $\B$ meson decay and thus might be negative if the signal $\B$
meson decays first. Secondly, if the tagging $\B$ meson decays into a
flavour-specific final state, this specifies the signal $\B$ mesons flavour at
$\Delta t=0$.

As a result of the production mechanism, the fully reconstructed signal decay
leaves all remaining tracks to originate from the tagging $\B$. This allows to
classify the decay of the tagging meson according to the signature of the final
state particles. Specialised taggers then identify decays based on their
specific signature. In a second stage, all results provided by the single
taggers are combined into a final tagging decision. 

The lepton taggers deduce a tag from the charge of electrons and muons from
$\bquark \to \cquark \lepm \neutrinobar$ transitions in semileptonic $\B$
decays. Likewise, second order transitions from $\bquark \to \Wm \cquark ( \to
\squark \lepp \neutrino )$ can be used, but has to be handled differently as
their charge is opposite compared to primary leptons. Charged kaon candidates
from the $\bquark \to \cquark \to \squark$ decay chain reflect the charge of the
tagging meson. Here as well, kaons from second order transitions carry the
opposite charge. Charged pions from charm decays work as tagging particles
either directly or in events with both a pion and a charged kaon the correlation
of the particles can be utilised to improve the tag decision. Selecting high
momentum particles, \eg fast pions from $\Bdbar \to \Dstarp \pim$, is on its own
another source of tagging information and might additionally be correlated with
slow particles to enhance the tagging result. Finally the flavour of $\Lambda$
baryons from decays if the tagging $\B$ meson can be exploited.

The \Babar experiment uses \acp{ANN} for each single tagger to provide several
intermediate tagging decisions that are subsequently combined using another \ANN
to a joined tagging decision. The effective tagging efficiency of the final
\Babar tagging algorithm is estimated on data to be $\efftageff = \SI{33.1 +-
0.3}{\percent}$. In \Belle analyses the flavour tagging approach is similar.
Instead of \acp{ANN} multi-dimensional look-up tables are used. At first,
information from charged tracks are looked-up to sort them into the signature
categories. Next, the received results are used on an event-level to provide the
final tagging decision. Using this approach an effective tagging efficiency of
$\efftageff = \SI{30.1 +- 0.4}{\percent}$ is achieved.

\section{OS FT}
\label{sec:flavour_tagging:os}
OS algorithms: kaon, muon, electron, vertex charge, charm (+ MVA versions)

\section{SS FT}
\label{sec:flavour_tagging:ss}
SS algorithms: kaon/pion/proton

\section{Calibration}
\label{sec:flavour_tagging:calibration}
\begin{itemize}
  \item calibration method
  \item tagging correlations
  \item OS calibration using Bu2JpsiK: procedure, transferability/validity to Bd2JpsiKS
  \item SS calibration using Bd2JpsiKst: procedure, transferability/validity to Bd2JpsiKS
  \item determination of systematic uncertainties
\end{itemize}
\subsection{OS calibration using Bu2JpsiK}
\label{sec:flavour_tagging:calibration:os}
\subsection{SS calibration using Bd2JpsiKst}
\label{sec:flavour_tagging:calibration:ss}

\section{Combination}
\label{sec:flavour_tagging:combination}
Combination of OS and SS tagging, FT correlations

\section{Performance}
\label{sec:flavour_tagging:performance}
Performance of the FT algorithms, effect on dilution of asymmetry
influence on measuring sin2beta, Dilution

\section{Recent developments}
\label{sec:flavour_tagging:developments}

\section{Influence on the measurement of sin2beta}
\label{sec:flavour_tagging:sin2beta}
