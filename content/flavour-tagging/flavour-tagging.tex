%!TEX root = ../../common/main.tex

\chapter{Flavour tagging}
\label{ch:flavour_tagging}

The time-dependent measurement of the $\CP$ asymmetry $\CPAsymmetry(\obsTime)$
requires the knowledge of the reconstructed \Bmeson's flavour at production, \ie
whether it contained a $\bquark$ or a $\bquarkbar$ quark. The method and
algorithms used to infer this information from all available event properties
are named flavour tagging.

Each tagging algorithm provides a tag $\tagdecision$ and a probability estimate
$\mistagestimate$ that the assigned tag is wrong, also called mistag estimate.
The tag is $\tagdecision=+1$ for an initial $\Bd$, $\tagdecision=-1$ for an
initial $\Bdbar$, and $\tagdecision=0$ if the tagging algorithms were not able
to determine a decision. The mistag estimate interval is given by ${[0,0.5[}$,
where mistags $\mistagestimate^{\prime}>0.5$ are evaluated as $\mistagestimate =
1 - \mistagestimate^{\prime}$ and the corresponding tag's sign is reversed. If
the algorithm is not able to determine a decision, the mistag is set
$\mistagestimate=0.5$.

The performance of the flavour tagging algorithms can be assigned using control
samples of \Bmesons whose final state determines the $\B$ flavour at decay time
(\ie are flavour-specific), \eg $\BuToJpsiK$ where the charge of the kaon allows
to infer the charge of the initial $\B$.

Given the numbers of all rightly tagged $\B$ candidates $\NRtagged$, all wrongly
tagged candidates $\NWtagged$, and all untagged candidates $\NUtagged$, the
tagging efficiency $\tageff$ can be assigned
%
\begin{equation}\label{eq:flavour_tagging:tageff}
  \tageff = \frac{\NRtagged + \NWtagged}{\NRtagged + \NWtagged + \NUtagged}\eqpd
\end{equation}
%
In addition to the mistag estimate, the true underlying fraction of wrongly
tagged candidates $\mistag$ is given by
%
\begin{equation}\label{eq:flavour_tagging:mistag}
  \mistag = \frac{\NWtagged}{\NRtagged + \NWtagged}\eqpd
\end{equation}
%
The statistically relevant \acl{FoM} for time-dependent measurements of $\CP$
violation, the \enquote{effective tagging efficiency}
%
\begin{equation}\label{eq:flavour_tagging:efftageff}
  \efftageff = \tageff(1 - 2 \mistag)^2 = \tageff \tagdilution^2 \eqcm
\end{equation}
%
represents the fraction of candidates necessary to reach the same statistical
power if the tagging would be perfect, and hence should be maximized to achieve
the best performance. The quantity $\tagdilution$ is called dilution, taking a
value of $1$ in case of perfect tagging, and $0$ in case of random tagging. As
described \eg in \Ref~\cite[][Ch. 10.3]{Bevan:2014iga} the dilution can be
directly translated to the amplitude of the measured $\CP$ asymmetry.

In the following sections more details on flavour tagging are provided. A short
introduction of the utilized algorithms and a comparison to the flavour tagging
used at lepton colliders is given in \cref{sec:flavour_tagging:b_factories}. In
\Cref{sec:flavour_tagging:lhcb} the two different classes of flavour tagging
algorithms used at \LHCb are presented, while the calibration of the tagging
algorithm outputs is described in
\cref{sec:flavour_tagging:calibration}. The combination of the different
algorithms' responses into a single decision is outlined in
\cref{sec:flavour_tagging:combination}. Recent developments are discussed in
\cref{sec:flavour_tagging:developments}.

More details on the flavour tagging employed by the \LHCb collaboration can be
found in \cite{Aaij:2012mu,FT:RunI}.

% %%%%%%%%%%%%%%%%%%%%%%%%%%%%%%%%%%%%%%%%%%%%%%%%%%%%%%%%%%%%%%%%%%%%%%%%%%%%%%
\section{Flavour tagging at \Babar and \Belle}
\label{sec:flavour_tagging:b_factories}

Due to the different nature of the experimental setup, the tagging method used
at \LHCb differs from methods used at the \BFactories. As described in
\cref{sec:lhcb_experiment:detector}, the \bhadron production is dominated
by gluon-gluon fusion over a large range of $q^2$ in contrast to the production
at the $\YFourS$ $\bbbar$ resonance as it was the case at the \BFactories. The
following section gives a résumé of the flavour tagging methods employed by the
\Babar and \Belle collaborations outlined in \Ref~\cite[][Ch. 8]{Bevan:2014iga}.
%
\begin{figure}[h]
\centering
\includegraphics[width=\textwidth]{private/content/flavour-tagging/figs/bfactory_tagging}
\caption{Illustration of the measurement of \sintwobeta as conducted by the \BFactories.
}
\label{fig:flavour_tagging:lhcb:b_factory_basic_principles}
\end{figure}

\Cref{fig:flavour_tagging:lhcb:b_factory_basic_principles} illustrates the basic
principles of a \CP violation measurement at the \BFactories. The produced
$\Bd\Bdbar$ pair's wave function is in a $P$-wave entangled state, until one of
the mesons decays. From this point in time ($t_1$) the second $\B$ meson
propagates further through the detector, oscillates to its antimatter state, and
finally decays at $t_2$. The \CP asymmetry $\CPAsymmetry(\obsTime)$ will
therefore be a function of the decay time difference $\Delta t$. This has two
implications: First, the decay time difference will always be computed relative
to the \emph{tagging} \Bmeson decay and thus might be negative if the signal
$\B$ meson decays first. Secondly, if the tagging \Bmeson decays into a
flavour-specific final state, this specifies the signal \Bmeson's flavour at
$\Delta t=0$.

As a result of the production mechanism, the fully reconstructed signal decay
leaves all remaining tracks to originate from the tagging $\B$. This allows to
classify the decay of the tagging meson according to the signature of the final
state particles. Specialised taggers then identify decays based on their
specific signature. In a second stage, all results provided by the single
taggers are combined into a final tagging decision. 

The lepton taggers deduce a tag from the charge of electrons and muons from
$\bquark \to \cquark \lepm \neutrinobar$ transitions in semileptonic $\B$
decays. Likewise, second order transitions from $\bquark \to \Wm \cquark ( \to
\squark \lepp \neutrino )$ can be used, but have to be handled differently as
their charge is opposite compared to primary leptons. Charged kaon candidates
from the $\bquark \to \cquark \to \squark$ decay chain reflect the charge of the
tagging meson. Here as well, kaons from second order transitions carry the
opposite charge. Charged pions from charm decays work as tagging particles
either directly or in events with both a pion and a charged kaon, the
correlation of the particles can be utilised to improve the tag decision.
Selecting high momentum particles, \eg fast pions from $\Bdbar \to \Dstarp
\pim$, is on its own another source of tagging information and might
additionally be correlated with slow particles to enhance the tagging result.
Finally the flavour of $\Lambdap$ baryons from decays of the tagging $\B$ meson
can be exploited.

The \Babar experiment uses \acp{ANN} for each single tagger to provide several
intermediate tagging decisions that are subsequently combined using another \ANN
to gain a joined tagging decision. The effective tagging efficiency of the final
\Babar tagging algorithm is estimated on data to be $\efftageff = \SI{33.1 +-
0.3}{\percent}$. In \Belle analyses the flavour tagging approach is similar.
Instead of \acp{ANN} multi-dimensional look-up tables are used. At first,
information from charged tracks are looked-up to sort them into the signature
categories. Next, the results are used on an per-event level to provide the
final tagging decision. Using this approach an effective tagging efficiency of
$\efftageff = \SI{30.1 +- 0.4}{\percent}$ is achieved.

% %%%%%%%%%%%%%%%%%%%%%%%%%%%%%%%%%%%%%%%%%%%%%%%%%%%%%%%%%%%%%%%%%%%%%%%%%%%%%%
\section{Flavour tagging algorithms}
\label{sec:flavour_tagging:lhcb}

At \LHCb several tagging algorithms, each specialized on different
characteristics of the underlying event, are used in order to determine the
\Bmeson flavour at production. Alongside a tag decision $\tagdecision$, each
algorithm provides an estimated mistag probability $\mistagestimate$ based on an
\acp{ANN} response. The algorithms can be classified into two types: the
\acf{SS} tagging and \acf{OS} tagging algorithms, also called \ac{SS}/\ac{OS}
taggers. The \ac{SS} taggers infer the production flavour of the signal \Bmeson
by identifying charged candidates that have a high chance of being remnants of
its hadronisation process. The \ac{OS} taggers exploit the dominant production
of \Bmesons through $\bbbar$ quark pair production, allowing to partially
reconstruct the \bhadron produced together with each reconstructed signal
\Bmeson and thereby infer its initial flavour.
\Cref{fig:flavour_tagging:lhcb:schematics} gives an overview of the tagging
algorithms used on the \acl{OS} and \acl{SS}.

The tagging algorithms are developed using simulated $\BuToJpsiK$ and
$\BdToDstarmunu$ decays. In an iterative procedure the selection criteria were
optimized in order to maximise the effective tagging efficiency $\efftageff$.
The algorithms only consider charged tracks with a good quality of the track fit
and momenta above $\SI{2}{\GeVc}$. Tracks with a polar angle of less than
$\SI{12}{\mrad}$ with respect to the beamline are declined. Further on,
particles originating from the signal candidate are suppressed by rejecting all
tracks that lie inside a cone of $\SI{5}{\mrad}$ around any signal $\B$ meson
daughter. Tracks from other \acp{PV} are eliminated using \IP requirements.

\Cref{sec:flavour_tagging:os,sec:flavour_tagging:ss} contain detailed
the \ac{SS} tagging and \ac{OS} tagging algorithms, also called \ac{SS}/\ac{OS}
taggers. The \ac{SS} taggers infer the production flavour of the signal \Bmeson
by identifying charged candidates that have a high chance of being remnants of
its hadronisation process. On the other hand, the \ac{OS} taggers exploit the
dominant production of \Bmesons through $\bbbar$ quark pair production, allowing
to partially reconstruct the \bhadron produced together with each reconstructed
signal \Bmeson and thereby infer its initial flavour.
%
\begin{figure}
\centering
%!TEX root = ../../../common/main.tex

% Colours 
\definecolor{fcdOrnA}{HTML}{331605}
\definecolor{fcdOrnB}{HTML}{662C0A}
\definecolor{fcdOrnC}{HTML}{99420F}
\definecolor{fcdOrnD}{HTML}{CC5814}
\definecolor{fcdOrnE}{HTML}{FF6E19}
\definecolor{fcdOrnF}{HTML}{FF975B}
\definecolor{fcdOrnG}{HTML}{FFAC7C}
\definecolor{fcdOrnH}{HTML}{FFC19C}
\definecolor{fcdOrnI}{HTML}{FFD6BD}
\definecolor{fcdOrnJ}{HTML}{FFEADE}
\definecolor{fcdBluA}{HTML}{052A33}
\definecolor{fcdBluB}{HTML}{0A5466}
\definecolor{fcdBluC}{HTML}{0F7E99}
\definecolor{fcdBluD}{HTML}{14A8CC}
\definecolor{fcdBluE}{HTML}{19D2FF}
\definecolor{fcdBluF}{HTML}{5BDFFF}
\definecolor{fcdBluG}{HTML}{7CE5FF}
\definecolor{fcdBluH}{HTML}{9CECFF}
\definecolor{fcdBluI}{HTML}{9CECFF}
\definecolor{fcdBluJ}{HTML}{DEF9FF}
\definecolor{fcdGrnA}{HTML}{243304}
\definecolor{fcdGrnB}{HTML}{476608}
\definecolor{fcdGrnC}{HTML}{6B990D}
\definecolor{fcdGrnD}{HTML}{A0E02D}
\definecolor{fcdGrnE}{HTML}{B2FF15}
\definecolor{fcdGrnF}{HTML}{C8FF58}
\definecolor{fcdGrnG}{HTML}{D3FF79}
\definecolor{fcdGrnH}{HTML}{DEFF9B}
\definecolor{fcdGrnI}{HTML}{E9FFBC}
\definecolor{fcdGrnJ}{HTML}{F4FFDE}
\definecolor{fcdVltA}{HTML}{310433}
\definecolor{fcdVltB}{HTML}{620866}
\definecolor{fcdVltC}{HTML}{930D99}
\definecolor{fcdVltD}{HTML}{C411CC}
\definecolor{fcdVltE}{HTML}{F514FF}
\definecolor{fcdVltF}{HTML}{F858FF}
\definecolor{fcdVltG}{HTML}{F979FF}
\definecolor{fcdVltH}{HTML}{FB9BFF}
\definecolor{fcdVltI}{HTML}{FCBCFF}
\definecolor{fcdVltJ}{HTML}{FEDDFF}

\definecolor{fcdGrayA}{HTML}{111111}
\definecolor{fcdGrayB}{HTML}{222222}
\definecolor{fcdGrayC}{HTML}{333333}
\definecolor{fcdGrayD}{HTML}{444444}
\definecolor{fcdGrayE}{HTML}{555555}
\definecolor{fcdGrayF}{HTML}{666666}
\definecolor{fcdGrayG}{HTML}{777777}
\definecolor{fcdGrayH}{HTML}{888888}
\definecolor{fcdGrayI}{HTML}{999999}
\definecolor{fcdGrayJ}{HTML}{AAAAAA}
\definecolor{fcdGrayK}{HTML}{BBBBBB}
\definecolor{fcdGrayL}{HTML}{CCCCCC}
\definecolor{fcdGrayM}{HTML}{DDDDDD}
\definecolor{fcdGrayN}{HTML}{EEEEEE}

\definecolor{fcdTropiteal}    {HTML}{00A8C6}
\definecolor{fcdTealDrop}     {HTML}{40C0CB}
\definecolor{fcdWhiteTrash}   {HTML}{F9F2E7}
\definecolor{fcdAtomicBikini} {HTML}{AEE239}
\definecolor{fcdFeebleWeek}   {HTML}{8FBE00}





\colorlet{ClrTxt}{black}
\colorlet{ClrTxtVeryDarkGray}{fcdGrayE}
\colorlet{ClrTxtDarkGray}{fcdGrayJ}
\colorlet{ClrVtxGray}{fcdGrayM}

\colorlet{ClrSigQuark}{fcdTealDrop}
\colorlet{ClrSigMeson}{fcdTropiteal}
\colorlet{ClrSigArrow}{fcdTropiteal}

\colorlet{ClrTagQuark}{fcdAtomicBikini}
\colorlet{ClrTagMeson}{fcdFeebleWeek}
\colorlet{ClrTagArrow}{fcdFeebleWeek}

\begin{tikzpicture}[
  scale=1, 
  >=stealth',
  font=\small,
  quark_sig/.style={
    align=center, 
    minimum size=3ex,
    circle,
    color=ClrSigQuark,
    fill=ClrSigQuark,
    text=ClrTxt,
    draw, 
    thick,
    inner sep=0pt,
    outer sep=0pt,
    node distance=0ex
  },
  quark_tag/.style={
    align=center, 
    minimum size=3ex,
    circle,
    color=ClrTagQuark,
    fill=ClrTagQuark,
    text=ClrTxt,
    draw, 
    thick,
    inner sep=0pt,
    outer sep=0pt,
    node distance=0ex
  },
  meson_sig/.style={
    draw, 
    align=center, 
    minimum size=4.5ex,
    circle,
    color=ClrSigMeson,
    fill=ClrSigMeson,
    text=ClrTxt,
    thick,
    inner sep=0pt,
    outer sep=1pt,
    node distance=0ex
  },
  meson_tag/.style={
    draw, 
    align=center, 
    minimum size=4.5ex,
    circle,
    color=ClrTagMeson,
    fill=ClrTagMeson,
    text=ClrTxt,
    thick,
    inner sep=0pt,
    outer sep=1pt,
    node distance=0ex
  },
  meson_comb/.style={
    shape=ellipse, 
    draw,
    fill,
    very thick,
    inner sep=0pt,
    outer sep=0pt,
    minimum width=8ex,
    minimum height=5ex},
  vertex/.style={
    shape=ellipse,
    draw,
    inner sep=1ex,
    outer sep=0pt,
    minimum width=10ex,
    minimum height=10ex,
    color=ClrVtxGray,
    fill=ClrVtxGray,
    text=ClrTxt,
    node distance=1ex
  },
  vertex_label/.style={
    inner sep=0pt,
    outer sep=0pt,
    text=ClrTxtDarkGray,
    node distance=1ex,
    font=\sffamily\small
  },
  tagger_label/.style={
    inner sep=0pt,
    outer sep=0pt,
    text=ClrTxtVeryDarkGray,
    node distance=1ex,
    font=\sffamily\small    
  },
  arrow_sig/.style={
    ->,
    very thick,
    color=ClrSigArrow
  },
  arrow_tag/.style={
    ->,
    very thick,
    color=ClrTagArrow
  }
]

%\draw[help lines] (-1,-5) grid (11,5);

\draw[dashed,color=fcdGrayE] (0,0) -- (12,0);
\node[text width=2.5cm,text=ClrTxtDarkGray,font=\sffamily\small] (SST) at (10.5,+0.3) {\hfill same side};
\node[text width=2.5cm,text=ClrTxtDarkGray,font=\sffamily\small] (OST) at (10.5,-0.3) {\hfill opposite side};


\node[draw,circle,fill,color=ClrSigQuark,inner sep=0pt,minimum size=8pt] 
  (coll) at (0,0) {};

\draw[<-,very thick] (coll) -- (+1,0);
\draw[->,very thick] (-1,0) -- (coll);

\begin{pgfonlayer}{foreground}
  
  % bbbar
  \node[quark_sig] (qrk_bbar) at (0.2,+0.8) {$\bquarkbar$};
  \node[quark_sig] (qrk_b) at (0.2,-0.8) {$\bquark$};
  
  \path (qrk_bbar) 
        to[circle connection bar switch color=from (ClrSigQuark) to (ClrSigQuark)] 
        (coll);
  \path (qrk_b) 
        to[circle connection bar switch color=from (ClrSigQuark) to (ClrSigQuark)] 
        (coll);
  
  % Signal decay
  \node[meson_sig,color=ClrSigMeson,fill=ClrSigMeson,text=ClrTxt] (SigJpsi) at (7.5,+2.25) {$\Jpsi$}  ;
  \node[meson_sig,color=ClrSigMeson,fill=ClrSigMeson,text=ClrTxt] (SigKS) [below=of SigJpsi] {$\KS$}  ;
  
  % SS tagging
  \path (qrk_bbar) ++(15:3ex) node (qrk_d) [quark_tag] {$\dquark$};
  
  \node[quark_tag] (qrk_dbar) at (0.8,+2.2) {$\dquarkbar$};
  
  \node[draw,circle,fill,color=ClrTagQuark,inner sep=0pt,minimum size=4pt] 
    (vacuumexc) at ([xshift=-2ex] $(qrk_dbar)!0.5!(qrk_d)$) {};
  
  \path (qrk_d) 
        to[circle connection bar switch color=from (ClrTagQuark) to (ClrTagQuark)] 
        (vacuumexc);
  \path (qrk_dbar) 
        to[circle connection bar switch color=from (ClrTagQuark) to (ClrTagQuark)] 
        (vacuumexc);  
  \path (qrk_dbar) ++(165:3ex) node (qrk_u)    [quark_tag] {$\uquark$};

  \node[meson_tag] (SSpip) at (3.3,+2.7) {$\pip$}; 
  

  % OS tagging
  \path (qrk_b)    ++(-15:3ex) node (qrk_xbar) [quark_tag] {$\quarkbar$};  

  % Tagging particles 
  \node[meson_tag] (OSlepton) at (9.5,-2.25) {$\lepm$};
  \node[meson_tag] (OSkaon)   at (9.5,-1)    {$\Km$};
  
\end{pgfonlayer} 
  
  
\node[meson_comb,rotate=+15,color=ClrSigMeson,text=ClrTxt] (SigBz) at ($(qrk_bbar)!0.5!(qrk_d)$) {  }; 
\node[meson_comb,rotate=-15,color=ClrTagMeson,text=ClrTxt] (Hb) at ($(qrk_b)!0.5!(qrk_xbar)$) {}; 
\node[meson_comb,rotate=-15,color=ClrTagMeson,text=ClrTxt] (pip_SS) at ($(qrk_dbar)!0.5!(qrk_u)$)   {}; 

  
\begin{pgfonlayer}{background}
  \node[vertex,fit=(SigBz)(Hb)(pip_SS),minimum width=16ex] (PV) {};
  \node[vertex_label,align=center] (PV_label) [above=of PV] {PV};
  
  \node[vertex,fit=(SigJpsi)(SigKS)   ,minimum width=10ex] (SigSV) {};
  \node[vertex_label,align=center] (SigSV_label) [above=of SigSV] {SV};
  
  \node[vertex,minimum width=11ex,minimum height=9ex,align=left] (OS_SV) at (4.5,-1.75) {$\bquark \to \cquark$\\  $\bquark \to X \lepm$};
  \node[vertex_label,align=center] (OS_SV_label) [above=of OS_SV] {SV};
  
  
  \node[vertex,minimum width=9ex,minimum height=7ex,align=center] (OS_TV) at (7.5,-1.25) {$\cquark \to \squark$};
\end{pgfonlayer}


\begin{pgfonlayer}{foreground}
  \draw[arrow_sig] (SigBz)  -- node (Bz_label) [above] {$\Bd$} ([xshift=+4pt] SigSV.west);
  \draw[arrow_sig] let \p1 =(SigJpsi.east) in (\x1-1,\y1+2)  -- (\x1+50,\y1+5);
  \draw[arrow_sig] let \p1 =(SigJpsi.east) in (\x1-1,\y1-2)  -- (\x1+50,\y1-5);
  \draw[arrow_sig] let \p1 =(SigKS.east)   in (\x1-1,\y1+2)  -- (\x1+50,\y1+5);
  \draw[arrow_sig] let \p1 =(SigKS.east)   in (\x1-1,\y1-2)  -- (\x1+50,\y1-5);
    
  \draw[arrow_tag] (Hb)     -- node (Hb_label) [above] {$\hb$}  ([xshift=+4pt] OS_SV.west);
  \draw[arrow_tag] (OS_SV)  -- ([xshift=+4pt] OS_TV.west);
  \draw[arrow_tag] (OS_SV)  -- ([xshift=+0pt] OSlepton.west);
  \draw[arrow_tag] (OS_TV)  -- ([xshift=+0pt] OSkaon.west);
  \draw[arrow_tag] (pip_SS) -- ([xshift=+0pt] SSpip.west);


  \node[tagger_label,align=left] (SSpip_label) [right=of SSpip] {SS pion};
  \node[tagger_label,align=left] (OSlepton_label) [right=of OSlepton] {OS muon\\ OS electron};
  \node[tagger_label,align=left] (OSkaon_label)   [right=of OSkaon] {OS kaon};
  \node[tagger_label,align=center] (OSVtcCh_label) [below=of OS_SV] {OS vertex charge};

\end{pgfonlayer}

\end{tikzpicture}

\caption{Schematic overview of the used \ac{OS} and \ac{SS} tagging
algorithms. \cite{wishahi:2013jt}}
\label{fig:flavour_tagging:lhcb:schematics}
\end{figure}

% ------------------------------------------------------------------------------
\subsection{\Acl*{OS} algorithms}
\label{sec:flavour_tagging:os}

The \OS algorithms all infer the tagging decision from the quark flavour of the
\bhadron produced in association with the signal \Bmeson. If for example a $\Bu$
decays into a $\Kp X$ final state through $\bquarkbar \to \cquarkbar \to
\squarkbar$ transition, it can unambiguously be stated by looking at the kaon
charge that the \bhadron produced along with the $\Bu$ contained a $\bquark$
quark. The tag identification for events with this signature fall into the scope
of the \OSK tagger. In contrast to the \SS taggers all \OS algorithms are
sensitive to an intrinsic mistag due to flavour oscillations of neutral \Bmesons.

In total, four distinct \OS taggers, each developed for a special \OS decay
signature, provide tagging information. Besides the \OSK tagger, the \OSe and
the \OSm taggers select leptons coming from the primary $\bquark \to \cquark
\lepm \neutrinobar$ decay to use their charge as an information carrier of the
$\B$ flavour. Finally, the \OSvtx tagger performs an inclusive reconstruction of
the \OS \SV to then compute a weighted sum of all particle track charges that
originate from the \SV. In the following, the selection criteria and algorithms
of the \OS taggers are briefly described.
 
%...............................................................................
\subsubsection{The \acl*{OSK} tagger}
\label{sec:flavour_tagging:os:kaon}

The kaon tagger exploits the charge of kaons stemming from $\bquark \to
\cquark \to \squark$ decays of the opposite side \bhadron. As the charge of the
kaon is the same of the ancestor's charge, the kaon always carries the opposite
charge as the signal \Bmeson.

To reduce background contributions of prompt kaons and kaons from primary
$\bquarkbar \to \Wp (\to \cquark \squarkbar) \cquarkbar$ transitions, in which
case the kaon carries the \enquote{wrong} charge, several selection criteria are
applied. Requirements on the \acl{pT}, the \acl{IP}, the \acl{IP} fit
significance, \ac{IP}/$\sigma_\text{\IP}$, and the track fit \chisqndf are
performed. \Ac{PID} requirements on the \DLLKpi and the \DLLKp suppress
mis-identified particles (\cf
\cref{sec:lhcb_experiment:pid:techniques_and_performance}). Clone tracks are
removed using the Kullback-Liebler criterion \cite{Needham:2008zza} and kaon
candidates which originate from other \acp{PV} are rejected using
\IP/$\sigma_\text{\IP}$ requirements with respect to any \PV.
\cref{tab:flavour_tagging:os:kaon:cuts} lists all applied cuts. If more than one
tagging candidate is found, the decision obtained from the candidate with the
highest \pT is chosen. To further increase the effective tagging efficiency only
tag decisions are considered if the estimated probability of the tagger to be
correct is larger than $\num{0.54}$.
%
\begin{table}
  \centering
  \caption{Selection requirements for the \OSK tagger \cite{Grabalosa:2012qra}.}
  \label{tab:flavour_tagging:os:kaon:cuts}
  \begin{tabular}{ll}
    \toprule
    Property & Value \\
    \midrule
    \pT                                       & $>\SI{0.5}{\GeVc}$                  \\
    \IP                                       & $<\SI{1.25}{\milli\metre}$          \\
    \IP$/\sigma_\text{\IP}$                   & $>\num{3.35}$                       \\
    track fit \chisqndf                       & $<\num{2.75}$                       \\
    \DLLKpi                                   & $>\num{6}$                          \\
    \DLLKp                                    & $>\num{-4}$                         \\
    \IP$/\sigma_\text{\IP}$ \wrt any \PV      & $>\num{4.5}$                        \\
    \multicolumn{2}{l}{reject clone tracks} \\
    \bottomrule
  \end{tabular}
\end{table}

%...............................................................................
\subsubsection{The \acl*{OSe} and the \acl*{OSm} taggers}
\label{sec:flavour_tagging:os:lepton}

Leptons from $\bquark \to \cquark \Wm (\to \lepm \neutrinobar)$ transitions are
selected in order to deduce the opposite \bhadron charge. Thus positive lepton
charge indicates a $\bquark$ quark content of the signal \Bmeson and vice versa.

Cuts on the \PID and the lepton \pT are applied to reduce the mis-identification
rate and to suppress leptons from secondary decays of charm mesons. The muon
candidate purity is further enhanced by reducing fake muons, clone tracks, and
requiring a sufficient track fit quality. Electron candidate tracks must lie
inside the \HCAL acceptance and leave a substantial energy deposition in the
\ECAL compared to their momentum. To reduce backgrounds from photon conversions
close to the \protonproton interaction region, an upper limit on the maximal
ionisation charge $Q_\text{\VELO}$ deposited in the \VELO is set. If more than
one lepton candidate passes the requirements, the one with the highest \pT is
chosen. The selection requirements for the lepton taggers are listed in
\cref{tab:flavour_tagging:os:lepton:cuts}.
%
\begin{table}
  \centering
  \caption{Selection requirements for the \OSe and \OSm taggers
  \cite{Grabalosa:2012qra}, where $E$ is the particle energy measured in the
  \ac{ECAL} and $p$ the measured particle momentum. $Q_\text{\VELO}$ describes
  the ionisation charge deposit in the \VELO.}
  \label{tab:flavour_tagging:os:lepton:cuts}
  \begin{tabular}{ll}
    \toprule
    % general
    Property                                  & Value                               \\
    \midrule
    lepton \pT                                & $>\SI{1.2}{\GeVc}$                  \\
    muon \DLLmupi                             & $>\num{1}$                          \\
    muon track fit \chisqndf                  & $<\num{3.2}$                        \\
    \multicolumn{2}{l}{reject muon clone tracks and fake muons}                     \\
    electron \DLLepi                          & $>\num{4}$                          \\
    electron $E/p$                            & $>\num{0.8}$                        \\
    max. electron $Q_\text{\VELO}$            & $<\num{1.6}$                        \\
    \multicolumn{2}{l}{electron candidate track in \HCAL acceptance}                \\
    \bottomrule
  \end{tabular}
\end{table}

%...............................................................................
\subsubsection{The \acl*{OSvtx} tagger}
\label{sec:flavour_tagging:os:vertex}

Besides the single particle taggers for kaons, muons, and electrons, the
\ac{OSvtx} tagger follows an alternative approach by trying to inclusively
reconstruct the decay vertex of the opposite \bhadron. Starting with a two-track
seed, matching tracks are appended and a weighted sum of the vertex charge is
calculated. The procedure is optimized using cuts on the considered tracks, the
seed properties, and the features of the reconstructed vertex.

The initial vertex seed is found by sampling track pairs from all particle
candidates surviving the following criteria: At least one of the tracks must be
a long track with a track fit quality of $\chisqndf<\num{2.5}$. To exclude poor
track reconstruction only tracks with an \IP (\wrt the \PV) uncertainty of
$\sigma_\text{\IP}<\num{1}$ are taken into account. Tracks from prompt particles
are rejected if the \IP significance $\text{\IP}/\sigma_\text{\PV}$ \wrt the \PV
is below $\num{2.5}$ or above $\num{100}$. Additionally the track has to have a
$\pT>\SI{0.15}{\GeVc}$ and an \IP$<\SI{3}{\milli\metre}$ \wrt the \PV. The
two seed tracks must be separated by an angle of $\phi>\SI{1}{\mrad}$ and one of
the tracks has to have a $\pT>\SI{0.3}{\GeVc}$.

For tracks pairs passing all selection criteria a vertex fit is performed. If
the fit succeeds and a fit quality of at least $\chisqndf<10$ is met, the fitted
vertex is considered as a seed. The seed is required to lie inside the detector
acceptance and in the forward direction of the \PV ($z>0$). 

The invariant mass of the seed is calculated and has to be greater than
$\SI{0.2}{\GeVcc}$ and not be compatible with the known $\KS$ or $\Lambdaz$
masses. % 0.490 < m_KS < 0.505 and 1.11 < m_Lambda < 1.12
A likelihood is computed for all seeds considering the vertex fit $\chisqndf$,
the minimum \pT, the maximum \PV \IP, the minimum \PV \IP$/\sigma_\text{\IP}$,
the angle between the seed tracks, the $z$-distance to \PV, and the angle
between the seed direction \wrt the \PV. The seed with the maximum likelihood
is selected. The seed algorithm successfully finds a seed in nearly
$\SI{50}{\percent}$ of all events and has a rate of $\SI{60}{\percent}$ to find
a seed that is formed by \bhadron decay products.

Subsequently more tracks are added to the seed. Additional tracks are required
to be compatible with the reconstructed seed vertex
(\IP$<\SI{0.9}{\milli\metre}$), have a \DOCA to any track in the seed smaller
than $\SI{0.2}{\milli\metre}$, the track fit quality is at least
$\chisqndf<\num{3}$, and are unlikely to be a clone track. Additionally, the \IP
with respect to the \PV has to be larger than $\SI{0.1}{\milli\metre}$ and
$\text{\IP}/\sigma_\text{\IP}>\num{3.5}$.

After all tracks are added to the vertex, another set of selection requirements
has to be fulfilled to optimize the effective tagging efficiency. The sum of all
track momenta has to be $\sum_i p_i > \SI{10}{\GeVc}$ and $\sum_i \pT(i) >
\SI{10}{\GeVc}$, the invariant vertex mass must be greater than $m_\text{vtx} >
\SI{0.5}{\GeVcc}$, the sum of the track
$\sum_i\text{\IP}_i/\sigma_{\text{\IP}_i}$ has to be larger $\num{10}$ and the
sum of the track \DOCA has to be $\sum_i\text{\acs{DOCA}}_i <
\SI{0.5}{\milli\metre}$.

The tag decision is then deduced from the inclusively reconstructed vertex by
calculating the weighted charge for all tracks $i$ forming the vertex
%
\begin{equation}\label{eq:flavour_tagging:os:vertex:charge}
  Q_\text{vtx} = \frac{\sum_i \pT^k(i)Q_i}{\sum_i\pT^k(i)}\eqcm
\end{equation}
%
where the track \acp{pT} are used as weights and the $k$ parameter is optimised
using simulated data to be $k=\num{0.4}$. In a final step, only charges of
$\vert Q_\text{vtx} \vert > 0.25$ are considered and the mistag probability has
to be smaller than $\num{0.46}$.

% ------------------------------------------------------------------------------
\subsection{\Acl*{SS} algorithms}
\label{sec:flavour_tagging:ss}

The \acl{SS} taggers exploit charge information from pions or kaons produced in
the hadronisation process of the signal \Bmeson. As illustrated in
\cref{fig:flavour_tagging:lhcb:schematics} the $\qqbar$ quark pair produced
alongside the signal meson, hadronises to form the $\B$ meson and in the case of
a $\Bd$ ($\Bs$) an accompanying pion (kaon). Another potential production
mechanism of pions or kaons associated with the signal \Bmeson is in the decay
of excited $\B$ meson states (\cf \cite{FT:KaonNNet}).

%...............................................................................
\subsubsection{The \acl*{SSpi} tagger}
\label{sec:flavour_tagging:ss:pion}

The pions from either the \Bmeson hadronisation or the decay of excited states
carry the same charge sign as the signal \Bmeson's $\bquark$ quark and are
therefore suitable to extract tagging information. The selection criteria for
pion candidates are listed in \cref{tab:flavour_tagging:ss:pion:cuts}. If more
than one pion candidate passes the requirements, the one with the highest
\pT is chosen. The tag decision has to have a probability to be correct of at
least $\num{0.54}$ otherwise it will be refused.
%
\begin{table}
  \centering
  \caption{Selection requirements for the \SSpi tagging candidates
  \cite{Grabalosa:2012qra}, where $\Delta Q$ describes the difference between
  the $\Bd\pion$ and the $\Bd$ invariant mass.}
  \label{tab:flavour_tagging:ss:pion:cuts}
  \begin{tabular}{ll}
    \toprule
    Property                                  & Value                               \\
    \midrule
    \multicolumn{2}{l}{only long track pion candidates}                             \\
    \DLLKpi                                   & $<\num{4.5}$                        \\
    \DLLppi                                   & $<\num{15}$                         \\
    \pT                                       & $>\SI{0.5}{\GeVc}$                  \\
    $p$                                       & $>\SI{2.5}{\GeVc}$                  \\
    \PV $\text{\IP}/\sigma_\text{\IP}$        & $<\num{4}$                          \\
    $\Delta Q$                                & $<\SI{1.5}{\GeVcc}$                 \\
    \bottomrule
  \end{tabular}
\end{table}

%...............................................................................
\subsubsection{The \acl*{SSK} tagger}
\label{sec:flavour_tagging:ss:kaon}

The charge of kaons produced together with a signal $\Bs$ meson can be exploited
in order to deduce a tag decision. The selection requirements for kaon
candidates are similar to those for the \SSpi tagger. Besides \PID variables,
momenta, the quality of the \IP fit and the track fit, differences between the
$\Bs\kaon$ mass and the $\Bs$ mass, the difference in the $\phi$ angle between
the signal \Bmeson and the kaon, and the pseudo-rapidity between the signal
\Bmeson and the kaon are taken into account. If more than one candidate passes
the criteria, the one with the highest \pT is used. The \SSK tagger is not used
in this analysis as only $\Bd$ and $\Bdbar$ meson initial states are considered.

% %%%%%%%%%%%%%%%%%%%%%%%%%%%%%%%%%%%%%%%%%%%%%%%%%%%%%%%%%%%%%%%%%%%%%%%%%%%%%%
\section{Calibration of the flavour tagging output}
\label{sec:flavour_tagging:calibration}

The tagging decisions each tagger makes are based on selection criteria
developed using simulations and flavour-specific decays. While the tag decision
$d$ depends on a measured particle or vertex charge, the mistag estimate
$\mistagestimate$ is computed using \acp{ANN} trained on \sweighted $\BuToJpsiK$
data incorporating kinematic and geometric event properties.

In order to adjust for differences between the training on simulated data and
the application on data, the \ANN output is calibrated using flavour-specific
decays. For the \OS taggers and the \SSpi tagger, $\BuToJpsiK$ decays are used
to develop a calibration function $\omofeta$ (while the portion of the data
sample that was used to train the \ANN before is omitted). Decays of charged
\Bmesons are not subject to quark mixing, thus, the true mistag probability
$\omega$ can be determined directly by counting and comparing the final state
charges. In case of neutral $\B$ decays into flavour-specific final states a
full time-dependent mixing analysis is necessary.

In \cref{sec:flavour_tagging:calibration:method} general concepts and the choice
of the calibration function are presented, then the methodologies of a
calibration using decays of either charged or neutral \Bmesons is explained. Two
different approaches how to incorporate the per-event tagging information are
described and an outlook into the technique of combining different tagger's
output into a single tag decision is given. Then the calibration of the \OS and
the \SSpi taggers using $\BuToJpsiK$ and $\BdToJpsiKstarz$ decays is shown in
\cref{sec:flavour_tagging:calibration:os,sec:flavour_tagging:calibration:ss}.

% ------------------------------------------------------------------------------
\subsection{Methodology}
\label{sec:flavour_tagging:calibration:method}

The calibration of the flavour tagging output usually follows an iterative
procedure. All single taggers and the combination of all \OS tagging decisions
(see \cref{sec:flavour_tagging:combination}) is calibrated using the high yield
decay channel $\BuToJpsiK$. The correction determined in this step is then
applied to the tagging algorithms during the common data processing stage
(\emph{stripping}). Afterwards, the calibration is checked on several control
channels, \ie $\BuToJpsiK$, $\BuToDpi$, $\BdToJpsiKstarz$, $\BdToDpi$,
$\BdToDstarmunu$ and $\BsToDspi$ and possible small corrections can be applied
for individual analyses. Assuming no correlations amongst the single taggers, it
can be assumed, that the calibration is still valid for the combined tagging
estimates. This assumption is checked as well.

A linear calibration function $\omofeta$ is chosen with two parameters $\pzero$
and $\pone$,
%
\begin{equation}\label{eq:flavour_tagging:calibration:method:calibration}
  \omofeta = \pzero + \pone (\mistagestimate - \avgmistagestimate)\eqcm
\end{equation}
%
with $\avgmistagestimate$ being the average mistag estimate introduced to
decorrelate $\pzero$ and $\pone$. Hence, a perfectly calibrated tagging output
would result in $\pzero = \avgmistagestimate$ and $\pone = 1$. The performance
of the tagging algorithms may depend on the flavour of the initial \Bmeson.
Different interaction of particles, \eg kaons used by the \OSK tagger, induce
deviating reconstruction efficiencies, resulting in different tagging
efficiencies $\tageff$ and mistag probabilities $\mistag$ for initial $\Bd$ and
$\Bdbar$ states. To correct for asymmetries of the tagging calibration,
$\deltamistag = \mistag^{\Bd} - \mistag^{\Bdbar}$, two independent calibration
functions are defined
%
\begin{equation}\label{eq:flavour_tagging:calibration:method:asymmetric_calibration:omega}
  \begin{split}
    \omofetaBd    &= \pzeroBd    + \poneBd    (\mistagestimate - \avgmistagestimate)\eqcm \\
    \omofetaBdbar &= \pzeroBdbar + \poneBdbar (\mistagestimate - \avgmistagestimate)\eqcm
  \end{split}
\end{equation}
%
where the calibration parameters $\p{i}{}$ (with $i=0,1$) can be parametrised as
%
\begin{equation}\label{eq:flavour_tagging:calibration:method:asymmetric_calibration:ps}
  \p{i}{\Bd} = \p{i}{} + \frac{\deltap{i}{}}{2} \eqspace\text{and}\eqspace \p{i}{\Bdbar} = \p{i}{} - \frac{\deltap{i}{}}{2}\eqpd
\end{equation}

%...............................................................................
\subsubsection[Flavour tagging calibration using decays of charged \Bmesons]{Flavour tagging calibration using decays of charged \Bbfsf mesons}
\label{sec:flavour_tagging:calibration:method:charged}

Using decays of charged \Bmesons into flavour-specific final states like
$\BuToJpsiK$ and $\BuToDpi$ is the straightforward way to calibrate the flavour
tagging, as the decisions given by the tagging algorithms can be easily compared
to the final state charges. Using a fit to the $\Bu$ mass distribution, signal
weights (\sweights) can be computed using the \sPlot technique
\cite{Pivk:2004ty}. The mistag estimate distribution can then be split into $n$
bins to compare the average estimated mistag of each bin $\mistagestimate_i$ to
the counted mistag ratio $\mistag_i$.
\Cref{fig:flavour_tagging:calibration:method:charged:calibration_plot} shows an
exemplary calibration plot using a dataset of $\BuToJpsiK$ decays corresponding
to an integrated luminosity of $\SI{0.37}{\per\femtobarn}$ \cite{Aaij:2012mu}.
The data is categorised into $\num{21}$ bins of $\mistagestimate$ and a linear
function is fit to the data to estimate the calibration parameters $\pzero$ and
$\pone$.
%
\begin{figure}
\centering
\includegraphics[width=0.5\textwidth]{private/content/flavour-tagging/figs/lhcb_ft_calibration_bu2jpsik_037fb}
\caption{Measured mistag fraction $\mistag$ plotted against the mistag estimate
$\mistagestimate$ provided by the \OS tagging algorithms. Described in black are
data of $\BuToJpsiK$ decays corresponding to an integrated luminosity of
$\SI{0.37}{\per\femtobarn}$. The red line illustrates the linear calibration
function fit to the data. \cite{Aaij:2012mu}}
\label{fig:flavour_tagging:calibration:method:charged:calibration_plot}
\end{figure}

As an alternative, the calibration function parameters can be estimated through
an unbinned maximum likelihood fit to tagging decision, true state, and mistag
probability estimate, using a \PDF
$\Prob{}{\Sig}(\tagdecision,\tagdecision^\prime,\mistagestimate)$, where
$\tagdecision^\prime$ describes the true $\Bu$ tag derived from the final-state
particle charge,
%
\begin{equation}\label{eq:flavour_tagging:calibration:method:charged:tagdefinition}
  \Prob{}{\Sig}(\tagdecision,\tagdecision^\prime,\mistagestimate) = 
  \begin{cases}
        \tageff \bigl(1-\omofeta\bigr) \Prob{}{\Sig}(\mistagestimate)    & \text{if $\tagdecision =    \tagdecision^\prime$} \\
        \tageff              \omofeta  \Prob{}{\Sig}(\mistagestimate)    & \text{if $\tagdecision \neq \tagdecision^\prime$} \\
    1 - \tageff                                                          & \text{if $\tagdecision = 0$} \\
  \end{cases} \eqpd
\end{equation}
%

%...............................................................................
\subsubsection[Flavour tagging calibration using decays of neutral \Bmesons]{Flavour tagging calibration using decays of neutral \Bbfsf mesons}
\label{sec:flavour_tagging:calibration:method:neutral}

In the decay of neutral \Bmesons, the oscillation of the propagating $\B$ state
prevents the determination of the initial flavour just by measuring the final
state charge. Instead, a full decay time dependent mixing analysis has to be
performed. As described \eg in \cite{Aaij:2012nt}, the signal mixing asymmetry is
given by
%
\begin{equation}
  \MixingAsymmetry(\obsTime) = (1 - 2\mistag)\cos(\DMd \obsTime)\eqpd
\end{equation}
%
This can be either utilised in a simultaneous fit in different bins of
$\mistagestimate$ to extract per bin $\mistag_i$, then following the same
procedure as described for charged initial states or by implementing the
calibration function $\omofeta$ directly into the \PDF.

%...............................................................................
\subsubsection{Evaluation of systematic uncertainties}
\label{sec:flavour_tagging:calibration:method:systematics}

In the determination of systematic uncertainties two different causes of errors
are considered: \enquote{type I} uncertainties are related to systematic
uncertainties from the methodology of the calibration itself, while
\enquote{type II} uncertainties account for systematic uncertainties resulting
from using the calibration parameters determined in a control channel in the
signal channel. The standard type II uncertainties cover for differences between
all commonly used control channels, whereas the procedure followed in this
analysis differs, as described later on.

% ------------------------------------------------------------------------------
\subsection[
  head={\Acs*{OS} tagger calibration using \BuToJpsiK decays},
  tocentry={\Acl*{OS} tagger calibration using \BuToJpsiKHyperref decays}
]{\Acl*{OS} tagger calibration using \BuToJpsiKbfsf decays}
\label{sec:flavour_tagging:calibration:os}

Deviating from the standard procedure to calibrate the \OS taggers and their
combination using a set of physics channels, only $\BuToJpsiK$ is used. A
previous study \cite{Aaij:1497268} made apparent that the type II uncertainties
arising from the flavour tagging calibration provide the largest contribution to
the overall systematic uncertainties on the measured \CP parameters $\SJpsiKS$
and $\CJpsiKS$. Therefore, giving up on semileptonic and fully hadronic decay
modes with different kinematic properties and only using $\BuToJpsiK$ decays
with a nearly similar final state, can reduce these uncertainties. An additional
cross-check of the calibration is performed on $\BdToJpsiKstarz$ decays to
confirm the portability of the calibration on a $\Bu$ to a $\Bd$ initial state.

The calibration then follows the already outlined approach: extracting \sweights
by a fit to the \Bu mass distribution, then comparing the true fraction of
wrongly tagged candidates to the average mistag estimate in $\num{60}$ bins of
$\mistagestimate$. The mass model used in the fit is a sum of three Gaussian
\acp{PDF} with a shared mean to describe the signal shape and an exponential
\PDF for the background contributions. The fit to the
$(\mistagestimate_i,\mistag_i)$ pairs using
\cref{eq:flavour_tagging:calibration:method:calibration} yields
%
\begin{equation*}\label{eq:flavour_tagging:calibration:os:parameters}
  \begin{split}
    \p{0}{\text{\acs{OS}}} &= 0.3815 \pm 0.0011 \text{\,(\stat)} 
                               \pm 0.0015 \text{\,(\syst.\,type I)}  
                               \pm 0.0005 \text{\,(\syst.\,type II)} \eqcm\\ % total syst 0.0016 
    \p{1}{\text{\acs{OS}}} &= 0.978\phantom{0} \pm 0.012\phantom{0} \text{\,(\stat)} 
                                         \pm 0.006\phantom{0}  \text{\,(\syst.\,type I)} 
                                         \pm 0.007\phantom{0}  \text{\,(\syst.\,type II)} \eqcm\\ % total syst 0.009
    \langle\mistagestimate^{\text{\acs{OS}}}\rangle &= 0.3786 \eqcm \\
    \rho_{\pzero,\pone} &= 0.14 \eqcm
  \end{split}
\end{equation*}
%
where $\rho_{\pzero,\pone}$ is the statistical correlation between
$\p{0}{\text{\acs{OS}}}$ and $\p{1}{\text{\acs{OS}}}$ \cite{Aaij:2015vza}.
\Cref{fig:flavour_tagging:calibration:os:mass_and_calibration} shows the mass
distribution and the \PDF projection as well as the calibration plot.
%
\begin{figure}
  \centering
  \includegraphics[width=0.48\textwidth]{private/content/flavour-tagging/figs/lhcb_ft_calibration_bu2jpsik_mass}
  \includegraphics[width=0.48\textwidth]{private/content/flavour-tagging/figs/lhcb_ft_calibration_bu2jpsik_calibration}
  \caption{(Left) Distribution of $\BuToJpsiK$ mass and projection of the fit \PDF.
  The black solid line shows the total mass \PDF, the red solid line the signal
  and the blue dashed line the background component. (Right) Distribution of
  $(\mistagestimate_i,\mistag_i)$ pairs in the decay $\BuToJpsiK$ together with
  the linear calibration function in red. \cite{Aaij:2015vza}}
  \label{fig:flavour_tagging:calibration:os:mass_and_calibration}
\end{figure}
%
The asymmetries in the mistag probabilities are determined by repeating the
measurement when splitting the data sample into the initial $\Bd$ flavours and
comparing the values for $\pzero$ and $\pone$. The results \cite{Aaij:2015vza}
are
%
\begin{equation*}\label{eq:flavour_tagging:calibration:os:asymmetries}
  \begin{split}
    \deltap{0}{\text{\acs{OS}}} &= 0.0148 \pm 0.0016 \text{\,(\stat)} \pm  0.0007 \text{\,(\syst. type I)} \pm  0.0004 \text{\,(\syst. type II)} \eqcm \\ % total syst. 0.0008
    \deltap{1}{\text{\acs{OS}}} &= 0.070\phantom{0} \pm 0.018\phantom{0} \text{\,(\stat)} \pm 0.003\phantom{0} \text{\,(\syst. type I)} \pm 0.002\phantom{0} \text{\,(\syst. type II)} \eqcm \\ % total syst. 0.004
    \Delta \tageff^{\text{\acs{OS}}} &= \tageff^{\text{\acs{OS},\Bd}} - \tageff^{\text{\acs{OS},\Bdbar}} = (-0.09  \pm 0.09 \text{\,(\stat)} \pm   0.09 \text{\,(\syst)})\% \eqpd
  \end{split}
\end{equation*}
%
As the difference in the tagging efficiencies $\Delta \tageff^{\text{\acs{OS}}}$
is compatible with zero, it is neglected in the following.

The type I systematic uncertainties stem mainly from the method of background
subtraction. To study the size of the uncertainties, the model to determine the
\sweights is changed, the calibration procedure is repeated, and differences in
the results to the nominal determination are considered as a measure of the
systematic uncertainties.

To study the type II systematic uncertainties, possible differences between the
calibration channel $\BuToJpsiK$ and the signal channel $\BdToJpsiKS$ are
inspected. Distributions of parameters that are known to influence the tagging
algorithms as the \acf{pT}, the \pseudorapidity, the azimuthal angle ($\phi$) of
the \Bmeson candidate, and the number of tracks (nTracks) and \aclp{PV}
(n\acsp{PV}) in the event are compared using signal \sweights. The five
distributions in the calibration channel are reweighted to match the
distribution found in $\BdToJpsiKS$. For each correlated variable the deviation
is assigned and the sum in quadrature is taken as systematic uncertainty of type
II. The results are summarised in
\cref{tab:flavour_tagging:calibration:os:systematics}.
%
\begin{table}
  \centering
  \caption{Summary of type II systematic uncertainties for the \OS calibration
  parameters from reweighting the distributions of \ac{pT}, \pseudorapidity,
  $\phi$, and the number of tracks and \acp{PV} using signal \sweights.
  \cite{Aaij:2015vza}}
  \label{tab:flavour_tagging:calibration:os:systematics}
  \begin{tabular}{ccccc}
    \toprule
      & $\delta\pzero$ $(\num{e-3})$ & $\delta\pone$ $(\num{e-2})$ & $\delta\deltapzero$ $(\num{e-3})$ & $\delta\deltapone$ $(\num{e-2})$ \\
    \midrule
    $\text{\acs{pT}}$             & 0.2 & 0.2 & 0.2 & 0.2 \\
    $\text{\acs{pseudorapidity}}$ & 0.4 & 0.5 & 0.2 & 0.2 \\
    $\phi$                        & 0.0 & 0.1 & 0.3 & 0.1 \\
    nTracks                       & 0.3 & 0.4 & 0.1 & 0.1 \\
    nPVs                          & 0.1 & 0.2 & 0.2 & 0.1 \\
    \midrule
      Total                       & 0.5 & 0.7 & 0.4 & 0.2 \\
    \midrule
    Percentage of & 
    \multirow{2}[2]{*}{\SI{45.5}{\percent}} & 
    \multirow{2}[2]{*}{\SI{58.3}{\percent}} & 
    \multirow{2}[2]{*}{\SI{25.0}{\percent}} & 
    \multirow{2}[2]{*}{\SI{11.1}{\percent}} \\
    stat. uncert. \\
    \bottomrule
  \end{tabular}
\end{table}

% ------------------------------------------------------------------------------
\subsection[
  head={\Acs*{SS} tagger calibration using \BdToJpsiKstarz decays},
  tocentry={\Acl*{SS} tagger calibration using \BdToJpsiKstarzHyperref decays}
]{\Acl*{SS} tagger calibration using \BdToJpsiKstarzbfsf decays}
\label{sec:flavour_tagging:calibration:ss}

The \SSpi mistag estimate is pre-calibrated using the calibration parameters
%
\begin{equation}
\label{eq:tagging:sspion_precalibration}
    \p{0}{} = \num{0.425}\eqcm\eqspace
    \p{1}{} = \num{0.939}\eqcm\eqspace
    \langle\eta\rangle = \num{0.379}.
\end{equation}
%
Afterwards, the \SSpi tagger is calibrated using a dataset of $\BdToJpsiKstarz$
decays corresponding to an integrated luminosity of $\SI{3.0}{\per\fb}$
collected by the \LHCb experiment in \RunOne. As the \SSpi tagging algorithm
depends on the fragmentation process of the initial \Bmeson, a $\Bd$ decay mode
is preferred over a decay of $\Bu$ mesons. Only candidates carrying a tagging
response from the \SSpi tagger are considered in the following.

To determine the mixing asymmetry $\MixingAsymmetry(\obsTime)$, a fit to the
reconstructed decay time and mass distribution is performed. The signal mass
distribution is modelled utilising a \Ipatia \PDF \cite{Santos:2013gra},
composed of a generalised hyperbolic core and two-sided tails as introduced by
the Crystal ball \PDF \cite{set:crystalball}. A more detailed description of the
\Ipatia \PDF is provided in
\cref{sec:measurement_of_sin2beta:likelihood_fit:pdfs:ipatia}. In the present
case the tail parameters are fixed to values determined on simulations. The
description of the combinatorial background component consists of a sum of two
exponential \acp{PDF} with shared parameters in the mass dimension to reflect
two background components with different lifetimes seen in the decay time
dimension. Hence, the background decay time distribution is modelled by two
exponential functions as well. The decay time signal \PDF contains an acceptance
function $\eps(t)$ describing reconstruction and selection inefficiencies for
signal candidates with low decay times and a convolution of a $\B$ mixing \PDF
and a resolution model
%
\begin{equation}
  \Prob{\Sig}{} = \eps(t) \cdot \left[ \Prob{\text{Mix}}{}(t, \tagdecision, \tagdecision^\prime) \otimes \mathcal{R}(t - t_\text{true}) \right] \eqcm
\end{equation}
%
with
%
\begin{equation}
  \Prob{\text{Mix}}{}(t, \tagdecision, \tagdecision^\prime) 
  \propto 
  \exponential{-\sfrac{t}{\tau}} \left( 1 - \tagdecision \average{\deltamistag} + \tagdecision \tagdecision^\prime (1 - 2 \average{\mistag}) \cos(\DMd t) \right)\eqcm
\end{equation}
%
where the acceptance function is parametrised as
%
\begin{equation}
  \eps(t) = \arctan(t \exponential{\alpha t+ \beta})\eqpd
\end{equation}
%
The resolution model consists of a single Gaussian resolution function
$\mathcal{R}(t - t_\text{true})$ with the width fixed to
$\SI{50}{\femto\second}$ and no offset from zero. The parameters $\alpha$ and
$\beta$ of the acceptance function are fixed in the fit to values determined on
simulated candidates. The mixing \PDF depends on the decay time $t$, the tag
decision $\tagdecision$ of the \SSpi tagging algorithm, and the tag
$\tagdecision^\prime$ extracted from the final state flavour (\cf
\cref{eq:flavour_tagging:calibration:method:charged:tagdefinition}). This
parametrisation includes the average mistag difference $\average{\deltamistag}$
and the average of the mistag $\average{\mistag}$ for the given dataset and
therefore no per-event mistag information enter the fit model. To strike a
balance between the per-event technique outlined in
\cref{sec:flavour_tagging:calibration:method} and the complexity of the fit model,
an intermediate approach is chosen. The calibration is performed by a
simultaneous fit in $\num{5}$ equally filled bins of the mistag estimate
$\mistagestimate$. The per-bin average $\langle\mistagestimate\rangle_i$ of all
signal candidates is computed using the \splot method \cite{Pivk:2004ty}. Then,
the calibration can be parametrised inside the \PDF
%
\begin{equation}
  \begin{split}
    \average{\deltamistag}_i  &= \deltapzero + \deltapone ( \langle\mistagestimate\rangle_i - \langle\mistagestimate\rangle )\eqcm \\
    \average{\mistag}_i       &= \pzero + \pone ( \langle\mistagestimate\rangle_i - \langle\mistagestimate\rangle )\eqcm
  \end{split}
\end{equation}
%
with the calibration parameters $\pzero$, $\pone$, $\deltapzero$, and
$\deltapone$ are shared among all sub-samples of data where
$\langle\mistagestimate\rangle$ and $\langle\mistagestimate\rangle_i$ denote the
average mistag estimates for the total dataset and the correspondent
sub-samples. \Cref{fig:flavour_tagging:calibration:ss:fit} presents the mass and
decay time distributions of the $\BdToJpsiKstarz$ candidates, as well as
projections of the fitted \PDF components. The calibration parameters are
determined to be
%
\begingroup
  \thinmuskip=1mu
  \medmuskip=2mu plus 2mu minus 2mu
  \thickmuskip=3mu
\begin{equation*}\label{eq:flavour_tagging:calibration:ss:parameters}
  \begin{split}
    \p{0}{\text{\acs{SSpi}}}        &= \phantom{+}0.4232 \pm\, 
                                       0.0029 \text{\,(\stat)} \pm\, 
                                       0.0020 \text{\,(\syst.\,type I)} \pm\, 
                                       0.0019 \text{\,(\syst.\,type II)} \eqcm\\ % 0.0040 syst
    \p{1}{\text{\acs{SSpi}}}        &= \phantom{+}1.011\phantom{0} \pm\, 
                                       0.064\phantom{0} \text{\,(\stat)} \pm\, 
                                       0.009\phantom{0} \text{\,(\syst.\,type I)} \pm\, 
                                       0.030\phantom{0} \text{\,(\syst.\,type II)} \eqcm\\ % 0.034 syst
    \deltap{0}{\text{\acs{SSpi}}} &= -0.0026 \pm\, 
                                      0.0043 \text{\,(\stat)} \pm\, 
                                      0.0024 \text{\,(\syst.\,type I)} \pm\, 
                                      0.0013 \text{\,(\syst.\,type II)} \eqcm \\
    \deltap{1}{\text{\acs{SSpi}}} &= -0.171\phantom{0} \pm\, 
                                      0.096\phantom{0} \text{\,(\stat)} \pm\, 
                                      0.029\phantom{0} \text{\,(\syst.\,type I)} \pm\, 
                                      0.027\phantom{0} \text{\,(\syst.\,type II)} \eqcm \\
    \langle\mistagestimate^{\text{\acs{SSpi}}}\rangle &= \phantom{+}0.425 \eqcm
  \end{split}
\end{equation*}
\endgroup
%
and the parameter correlations are given in
\cref{tab:flavour_tagging:calibration:ss:correlations} \cite{Aaij:2015vza}. The
type I systematic uncertainties cover influences of the decay time acceptance
and resolution, the production and detection asymmetries, as well as possible
deviations of the results when performing a fit using a signal \sweighted
dataset. The study of the type II systematic uncertainties follows closely the
approach outlined in \cref{sec:flavour_tagging:os}. The results of the
reweighting procedure are presented in
\cref{tab:flavour_tagging:calibration:ss:systematics}.
%
\begin{table}
  \centering
  \caption{Statistical correlations between the \SSpi tagging calibration parameters. \cite{Aaij:2015vza}}
  \label{tab:flavour_tagging:calibration:ss:correlations}
  \begin{tabular}{ccccc}
    \toprule
    & $\p{0}{\text{\acs{SSpi}}}$ & $\p{1}{\text{\acs{SSpi}}}$ & $\deltap{0}{\text{\acs{SSpi}}}$ & $\deltap{1}{\text{\acs{SSpi}}}$ \\
    \midrule
    $\p{0}{\text{\acs{SSpi}}}$ & 1 & 0.04 & -0.007 & 0.0004 \\
    $\p{1}{\text{\acs{SSpi}}}$ & 0.04 & 1 & 0.0016 & -0.006 \\
    $\deltap{0}{\text{\acs{SSpi}}}$ & -0.007 & 0.0016 & 1 & 0.03 \\
    $\deltap{1}{\text{\acs{SSpi}}}$ & 0.0004 & -0.006 & 0.03 & 1 \\
    \bottomrule
  \end{tabular}
\end{table}
%
\begin{figure}[t]
  \includegraphics[width=0.48\textwidth]{private/content/flavour-tagging/figs/lhcb_ft_calibration_bd2jpsikstar_mass.pdf}
  \includegraphics[width=0.48\textwidth]{private/content/flavour-tagging/figs/lhcb_ft_calibration_bd2jpsikstar_time.pdf}
  \caption{(Left) Mass and (Right) decay time distributions of $\BdToJpsiKstarz$
  candidates. In blue the projections of the fitted signal components and in
  red/orange the projections of the combinatorial background components are
  shown. \cite{Aaij:2015vza}}
  \label{fig:flavour_tagging:calibration:ss:fit}
\end{figure}
%
\begin{table}
  \centering
  \caption{Summary of type II systematic uncertainties for the \SSpi calibration
  parameters from reweighting the distributions of \ac{pT}, \pseudorapidity,
  $\phi$, and the number of tracks and \acp{PV} using signal \sweights.
  \cite{Aaij:2015vza}}
  \label{tab:flavour_tagging:calibration:ss:systematics}
  \begin{tabular}{ccccc}
    \toprule
      & $\delta\pzero$ $(\num{e-3})$ & $\delta\pone$ $(\num{e-2})$ & $\delta\deltapzero$ $(\num{e-3})$ & $\delta\deltapone$ $(\num{e-2})$ \\
    \midrule
    $\text{\acs{pT}}$             & 1.3  & 2.1   & 0.6  & 2.7  \\
    $\text{\acs{pseudorapidity}}$ & 0.23 & 0.020 & 0.7  & 0.5  \\
    $\phi$                        & 0    & 0.6   & 0.14 & 0.27 \\
    nTracks                       & 1.5  & 2.0   & 1.1  & 0.08 \\
    nPVs                          & 0.6  & 0.5   & 0    & 0.05 \\
    \midrule
      Total                       & 1.9  & 3.0   & 1.3  & 2.7 \\
    \midrule
    Percentage of & 
    \multirow{2}[2]{*}{\SI{65.5}{\percent}} & 
    \multirow{2}[2]{*}{\SI{46.9}{\percent}} & 
    \multirow{2}[2]{*}{\SI{30.2}{\percent}} & 
    \multirow{2}[2]{*}{\SI{28.1}{\percent}} \\
    stat. uncert. \\
    \bottomrule
  \end{tabular}
\end{table}

% %%%%%%%%%%%%%%%%%%%%%%%%%%%%%%%%%%%%%%%%%%%%%%%%%%%%%%%%%%%%%%%%%%%%%%%%%%%%%%
\section{Combination of single tagger outputs}
\label{sec:flavour_tagging:combination}

In order to fully exploit the information of all tagging algorithms it is
desirable to combine their tag decisions $\tagdecision_i$ and mistag estimates
$\mistagestimate_i$ into one mutual decision. Given the probabilities
$p(\bquark)$ and $p(\bquarkbar)$ that a meson contains a $\bquark$ or a
$\bquarkbar$ quark
%
\begin{equation}\label{eq:flavour_tagging:combination:product}
  \begin{split}
    p(\bquark)    &= \prod_i\left( \frac{1 + \tagdecision_i}{2} - \tagdecision_i (1 - \mistagestimate_i) \right)\eqcm\\
    p(\bquarkbar) &= \prod_i\left( \frac{1 - \tagdecision_i}{2} + \tagdecision_i (1 - \mistagestimate_i) \right)\eqcm
  \end{split}
\end{equation}
%
the combined probabilities $P(\bquark)$ and $P(\bquarkbar)$ can be calculated
%
\begin{equation}\label{eq:flavour_tagging:combination:total}
  P(\bquark)    = \frac{p(\bquark)}{p(\bquark) + p(\bquarkbar)}\eqcm\eqspace
  P(\bquarkbar) = 1 - P(\bquark)\eqpd
\end{equation}
%
Then, the combined tag decision and mistag estimate are $\tagdecision = +1$ and
$\mistagestimate = 1 - P(\bquarkbar)$ if $P(\bquarkbar) > P(\bquark)$, otherwise
$\tagdecision = -1$ and $\mistagestimate = 1 - P(\bquark)$. As
\cref{eq:flavour_tagging:combination:product} ignores possible correlation among
the involved taggers, the joint mistag estimate might over- or underestimate the
true mistag. To correct for this the mistag estimates as well have to be
calibrated after the single tagger combination using a flavour-specific control
channel.

The analysis presented in this thesis utilises the combination of the calibrated
\OS tagging algorithms (see \cref{sec:flavour_tagging:calibration:os}) as well
as of the \SSpi tagger (see \cref{sec:flavour_tagging:calibration:ss}). As the
\OS and \SS taggers find their decisions using independent particle information
their output is assumed to be uncorrelated and no further calibration is
necessary after combining the \OS and \SSpi tagging decisions (\cf
\cref{sec:measurement_of_sin2beta:likelihood_fit:model:mistag}).

% %%%%%%%%%%%%%%%%%%%%%%%%%%%%%%%%%%%%%%%%%%%%%%%%%%%%%%%%%%%%%%%%%%%%%%%%%%%%%%
\section{Recent developments and \RunTwo}
\label{sec:flavour_tagging:developments}

The algorithms utilised in the flavour tagging are always subject of active
development with the aim to improve their performance and extend the tagging
strategy. Several taggers are being revisited using particle selections based on
\acp{ANN} or \acp{BDT}.

% SSpi BDT
A \BDT based \SSpi tagging algorithm (\acs{BDTSSpi}) is in the stage of early
development, intended to replace the existing cut-based \SSpi tagger in the
future.

% OS and SS K NNet
% https://twiki.cern.ch/twiki/pub/LHCbPhysics/NNetKaonTaggers/LHCb-ANA-2014-003-v1.0.pdf
For both the \OSK and the \SSK tagging algorithms \ANN based versions are
implemented (\acs{NNOSK} and \acs{NNSSK}) \cite{FT:KaonNNet}. 

The performance of the \NNSSK tagger shows a significant enhancement compared to
the cut-based \SSK implementation. In the $\BsToDspi$ channel an effective
tagging efficiency of $\efftageff = \SI{1.80 +- 0.23}{\percent}$ is found, while
in selected $\BsToJpsiphi$ candidates $\efftageff = \SI{1.26 +- 0.17}{\percent}$
is measured. This corresponds to an improvement of more than $\SI{40}{\percent}$
in both cases.

The \NNOSK tagger also shows an enhancement in the effective tagging efficiency
of around $\SI{10}{\percent}$ in decays of $\BdToDpi$ with respect to the
cut-based \OSK tagger. Still, more development effort is needed here, as the
overall efficiency gain is still small and the improvements can not be confirmed
yet in $\Bs$ decays.

% OS charm tagger
% https://twiki.cern.ch/twiki/pub/LHCbPhysics/OSCharmTagger/LHCb-PAPER-2015-027-v1r2.pdf
The study of opposite side decays of charm hadrons has lead to the development
of the \OSc tagger \cite{FT:OSCharm}. The \OSc algorithm exploits the charge of
charm hadrons or its daughter kaons produced in the decay of opposite side
\bhadrons. Candidates are identified using a set of multivariate selections of
fully or partially reconstructed Cabibbo-favoured decays, as \eg
$\Dz\to\Km\pip$, $\Dp\to\Km\pip\pip$, $\Dz/\Dp\to\Km\elp X$, or
$\Lambdac\to\proton\Km\pip$. 

The performance of the \OSc tagger is studied on data using selected
$\BuToJpsiK$ candidates where an effective tagging efficiency of $\efftageff =
(0.30 \pm 0.01\,\statp \pm 0.01\,\systp)\%$ is found. A cross-check on selected
$\BdToJpsiKstarz$ candidates provides compatible results with an effective
tagging efficiency of $\efftageff = (0.30 \pm 0.03\,\statp \pm 0.01\,\systp)\%$.
Fully-hadronic final states show a better performance: effective tagging
efficiencies of $\efftageff = (0.40 \pm 0.02\,\statp \pm 0.01\,\systp)\%$ and
$\efftageff = (0.39 \pm 0.03\,\statp \pm 0.01\,\systp)\%$ are found for
$\BdToDpi$ and $\BsToDspi$ candidates. As the \OSc decisions are highly
correlated with the \OSK output, only an additional $\SI{0.113}{\percent}$
absolute gain in effective tagging efficiency can be observed once the full \OS
combination is computed, still that compares to a month worth of \RunOne data
taking.

% SS proton
Similar to the \SSpi or \SSK algorithms a newly developed \SSp tagger infers
its tagging decision from protons produced in the fragmentation of the signal
\Bmeson. 

% Run II performance
Alongside new developments and optimisation of existing tagging algorithms the
performance of the flavour tagging will be affected by altered experimental
conditions in \RunTwo of the \LHC. The higher centre-of-mass energy of
$\sqrt{s}=\SI{13}{\TeV}$ leads to higher particle momenta, resulting in a better
selection of the particles involved in the tagging, thus leading to a higher
performance of the flavour tagging. Though, this will also result in higher
track multiplicities, impeding the selection, thus reducing the efficiency of
the involved algorithms. However, the higher peak luminosity due to the
increased centre-of-mass energy, allows for an extensive luminosity levelling
which itself will result in an effect the opposite of what was described before,
in a reduction of track multiplicities.

It can be summarised that the change in run conditions will force a
reoptimisation and recalibration of the flavour tagging algorithms. Studies on
\MC simulated data with \RunTwo conditions just started and data taken in the
first months of \RunTwo will be employed to study the performance and retune the
algorithms to be ready as soon as the \CP violation and \B oscillation
measurements are started. It is clear that with higher statistics available in
all physics and control channels, systematic effects will need to be even better
understood, thus making a universal calibration not feasible any more.



