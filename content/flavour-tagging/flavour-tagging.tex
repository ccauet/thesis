%!TEX root = ../../common/main.tex

\chapter{Flavour Tagging}
\label{ch:flavour_tagging}

The time-dependent measurement of the $\CP$ asymmetry $\CPAsymmetry$ in the
decay of $\Bd$ and $\Bdbar$ mesons into the $\CP$ eigenstate $\Jpsi\KS$ requires
the knowledge of the $\B$ meson flavour at production, \ie weather it contained
a $\bquark$ or a $\bquarkbar$. The method and algorithms used to infer this
information from all available event properties are named \enquote{flavour
tagging}.

Each tagging algorithm provides a tag $\tagdecision$ and a probability estimate
$\mistagestimate$ that the assigned tag is wrong, also called \enquote{mistag}
estimate. The tag is $\tagdecision=+1$ for an initial $\Bd$, $\tagdecision=-1$
for an initial $\Bdbar$, and $\tagdecision=0$ if the tagging algorithms were not
able to determine a decision. The mistag estimate interval is given by
$[0,0.5[$, where mistags $\mistagestimate^{\prime}>0.5$ are evaluated as
$\mistagestimate = 1 - \mistagestimate^{\prime}$ and the corresponding tag's
sign is reversed. If the algorithm is not able to determine a decision, the
mistag is set $\mistagestimate=0.5$.

The performance of the flavour tagging algorithms can be assigned using control
samples of $\B$ mesons whose final state determines the $\B$ flavour at decay
time, \eg $\BuToJpsiK$ where the charge of the kaon allows to infer the charge
of the $\B$.

Given the numbers of all right tagged $\B$ candidates $\NRtagged$, all wrong
tagged candidates $\NWtagged$, and all \enquote{untagged} candidates
$\NUtagged$, the \enquote{tagging efficiency} $\tageff$ can be assigned
%
\begin{equation}\label{eq:flavour_tagging:tageff}
  \tageff = \frac{\NRtagged + \NWtagged}{\NRtagged + \NWtagged + \NUtagged}\eqpd
\end{equation}
%
In addition to the mistag estimate, the true underlying fraction of wrong tagged
candidates $\mistag$ is given by
%
\begin{equation}\label{eq:flavour_tagging:mistag}
  \mistag = \frac{\NWtagged}{\NRtagged + \NWtagged}\eqpd
\end{equation}
%
The statistically relevant \acl{FoM} for time-dependent measurements of $\CP$
violation\addref{for $\efftageff$ for CPV measurements}, the \enquote{effective
tagging efficiency}
%
\begin{equation}\label{eq:flavour_tagging:efftageff}
  \efftageff = \tageff(1 - 2 \mistag)^2 = \tageff \tagdilution^2 \eqcm
\end{equation}
%
represents the fraction of candidates necessary to reach the same statistical
power if the tagging would be perfect, and hence should be maximized to achieve
the best performance. The quantity $\tagdilution$ is called \enquote{dilution},
taking a value of $1$ in case of perfect tagging, and $0$ in case of random
tagging. As described in \cref{missing} the dilution can be directly translated
to the amplitude of the measured $\CP$ asymmetry. \todo{Describe effect of dilution on CPV asymmetry}

In the following sections more details of the flavour tagging are provided. A
short introduction of the utilized algorithms and a comparison to the flavour
tagging used at lepton colliders is given in \cref{sec:flavour_tagging:lhcb}.
\Cref{sec:flavour_tagging:os,sec:flavour_tagging:ss} describe the two different
classes of flavour tagging algorithms used at \LHCb, while the calibration of
the tagging algorithms outputs is described in
\cref{sec:flavour_tagging:calibration}. The combination of different algorithm
responses into a single decision is outlined in
\cref{sec:flavour_tagging:combination}. Performance numbers are provided in
\cref{sec:flavour_tagging:performance}, recent developments are discussed in
\cref{sec:flavour_tagging:developments}, and finally the experimental impact of
the flavour tagging on the measurement of $\CP$ violation in $\BdToJpsiKS$
decays is provided in \cref{sec:flavour_tagging:sin2beta}.

More details on the flavour tagging used by \LHCb can be found in
\cite{Aaij:2012mu}.

\section{Tagging at LHCb}
\label{sec:flavour_tagging:lhcb}



Tagging at a hadron collider, comparison to e+e- machines, Belle II

\section{OS FT}
\label{sec:flavour_tagging:os}
OS algorithms: kaon, muon, electron, vertex charge, charm (+ MVA versions)

\section{SS FT}
\label{sec:flavour_tagging:ss}
SS algorithms: kaon/pion/proton

\section{Calibration}
\label{sec:flavour_tagging:calibration}
\begin{itemize}
  \item calibration method
  \item tagging correlations
  \item OS calibration using Bu2JpsiK: procedure, transferability/validity to Bd2JpsiKS
  \item SS calibration using Bd2JpsiKst: procedure, transferability/validity to Bd2JpsiKS
  \item determination of systematic uncertainties
\end{itemize}
\subsection{OS calibration using Bu2JpsiK}
\label{sec:flavour_tagging:calibration:os}
\subsection{SS calibration using Bd2JpsiKst}
\label{sec:flavour_tagging:calibration:ss}

\section{Combination}
\label{sec:flavour_tagging:combination}
Combination of OS and SS tagging, FT correlations

\section{Performance}
\label{sec:flavour_tagging:performance}
Performance of the FT algorithms, effect on dilution of asymmetry
influence on measuring sin2beta, Dilution

\section{Recent developments}
\label{sec:flavour_tagging:developments}

\section{Influence on the measurement of sin2beta}
\label{sec:flavour_tagging:sin2beta}
