%!TEX root = ../../common/main.tex

\chapter{Introduction}
\label{ch:introduction}

The visible universe shows a maximal asymmetry between matter and anti-matter
with complete absence of anti-matter (by definition) except in cosmic rays. With
the big bang theory being established a source of the asymmetry needs to be
determined. A possible explanation was provided by Sakharov
\cite{Sakharov:1967dj}. To achieve a universal baryon asymmetry three conditions
have to be fulfilled: 1) baryon number violating processes, 2) the violation of
\CSym and \CP in fundamental interactions, and 3) a deviation from the thermal
equilibrium during in the history of the universe.

The non-equilibrium state is guaranteed due to the expansion of the universe and
baryon number violation processes are known as well
\cite{tHooft:1976up,Rubakov:1996vz}, thus leaving us to find a source of \CP
violation large enough to explain the existing matter-antimatter asymmetry.

Discrete global symmetries play an outstanding role in the \SM. In particular,
the charge conjugation $\CSym$, the parity $\PSym$, and the time reversal
$\TSym$ transformations. The \CPT theorem \cite{set:cpt} states that the \CPT
symmetry holds for a quantum field theory, whereas the violation of each of
the discrete symmetries $\CSym$, $\PSym$, or $\TSym$ is still possible.

Lee and Yang where the first to question the general assumption of this time that all
physical interactions are invariant even under a single transformation by
stating a possible $\PSym$ violation in weak interactions \cite{Lee:1956qn}. This
statement was backed soon after by the measurement of $\PSym$ violation in the
$\beta^{-}$ decay of \cobaltsixty \cite{Wu:1957my}. The experiment
revealed that the weak interaction only couples to fermions with left-handed
chirality or anti-fermions with right-handed chirality, thus also violating
symmetry under the $\CSym$ transformation. Still, the interaction seems to behave
invariant under combined $\CP$ transformations. Not until 1964 the symmetry
violation under $\CP$ transformation was observed in the neutral kaon system
\cite{Christenson:1964fg} and consequently established in the \SM.

Writing the source of \CP violation into the \SM still was a challenge. The
triumphant idea came from Makoto Kobayashi and Toshihide Maskawa
\cite{Kobayashi:1973fv} proposing a third quark generation as a possibility to
explain \CP violation in the framework of the electroweak theory. Four years
later the discovery of the \bquark quark \cite{Herb:1977ek} provided enough
confidence in their theory to start planning the two \BFactories \Babar and
\Belle. At the two $\elel$ colliders PEP-II (Stanford, CA, US) and KEKB
(Tsukuba, JP) electrons and positrons were brought to collisions at asymmetric
energies. Both machines operated at a centre-of-mass energy of exactly the
\YFourS bottonium resonance at $\SI{10.58}{\GeV}$, producing $\B\Bbar$ meson
pairs at high rates of $\num{e6}$ a day, therefore their nomenclature
\cite{Bevan:2014iga}.

To definitely establishing \CP violation in the \SM it was necessary to extend
the measurements outside the kaon sector to the heavier \B and \D meson systems.
Predominantly built to study \CP violation in the \Bmeson system, \Babar and
\Belle were soon be successful and reported the observation of \CP violation in
decays of \Bd mesons to charmonium final states in 2001
\cite{Aubert:2001nu,Abe:2001xe}.



\begin{itemize}
  \item broken symmetries
  \item CPV in the SM, CKM mechanism
  \item CPV in lepton sector
  \item golden channel, sin2beta
  \item Measurement of sin2beta at B factories
  \item CKM unitarity
  \item ckm unitarity tests
  \item BSM physics
  \item indirect searches for BSM physics
  \item CPV at LHCb
  \item BdToJpsiKS (Vub tension)
  \item short summary of measurement
\end{itemize}

% References to people involved in the analysis
% \par\noindent\newline
\subsubsection*{Collaboration}
The work presented in this document was only possible to achieve in close
collaboration with colleagues from the \acs{LHCb} collaboration, most notably
Frank Meier and Julian Wishahi from the local working group, they are---together
with the author---the contact authors of this analysis published as:
%
\begin{quotation}
  \fullcite{Aaij:2015vza}.
\end{quotation}

Mirco Dorigo and Ulrich Eitschberger from the flavour tagging group provided the
flavour tagging calibration measurements (\cf
\cref{sec:flavour_tagging:calibration:os,sec:flavour_tagging:calibration:ss}).
The calculation of the kaon regeneration (\cf
\cref{sec:measurement_of_sin2beta:cpv_measurement:kaon_regeneration}) was
performed by Jeroen van Tilburg from the \B to charmonium working group.

% Structure of the document
% \par\noindent\newline
% The document is structured as follows:
\subsubsection*{Outline}

In \cref{ch:cpv_theory} the \SM is shortly recapped with an emphasis on flavour
physics and \CP violation in the quark sector. The nature of Yukawa
interactions and the appearance of the \CKM matrix through spontaneous symmetry
breaking are summarised. Then, the time evolution of \Bmesons including decay
and flavour oscillations is outlined. Finally, an overview of the measurement of
\CP violation in the decay of \BdToJpsiKS is given.

The \LHCb detector and its subsystems are briefly described in
\cref{ch:lhcb_experiment} with particular attention to the track reconstruction
and \acl{PID} techniques. The \LHCb trigger system is sketched and the software
stack is depicted. At this point, the data taking in the first successful years
of running is summarised also.

As it plays an important role in this analysis the flavour tagging algorithms
utilised at \LHCb are discussed in more detail in \cref{ch:flavour_tagging}.
After a general overview and a comparison to developments at the \BFactories the
flavour tagging algorithms are explained. Afterwards, the calibration of the
flavour tagging output is motivated and described. The performance of the
methods is then described and an outlook on recent developments is given.

The main part (\cref{ch:measurement_of_sin2beta}) contains the details of the
performed measurement. The data taking and preparation is explained as well as
studies of potential background contributions. The influence of decay time
acceptance and resolution effects is briefly summarised. Before the results are
presented the likelihood function is presented. Finally, all studies of
systematic effects are collected.

This thesis concludes with a summary and puts the results in context to the
global \CKM picture. Furthermore, \RunTwo and \LHCb upgrade prospects are
presented.
