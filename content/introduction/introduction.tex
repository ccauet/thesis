%!TEX root = ../../common/main.tex

\chapter{Introduction}
\label{ch:introduction}

With the Big Bang theory being the prevailing model to explain the earliest
known period of the universe, matter and anti-matter should have been produced
in equal quantities. To explain today's observation of a maximal asymmetry
between matter and anti-matter with complete absence of anti-matter in the
visible universe, the source of the asymmetry needs to be determined.

A possible solution was provided by Sakharov \cite{Sakharov:1967dj}. To achieve
a universal baryon asymmetry three conditions have to be fulfilled: 1) the
existence of baryon number violating processes, 2) the violation of \CSym and
\CP conservation in fundamental interactions, and 3) a deviation from the
thermal equilibrium during the history of the universe.

The non-equilibrium state is guaranteed due to the inflationary epoch of the
universe and baryon number violation processes are known as well
\cite{tHooft:1976up,Rubakov:1996vz}, thus leaving us to find a source of \CP
violation large enough to explain the existing matter-antimatter asymmetry.

Discrete global symmetries play an outstanding role in the \SM. Of particular
importance are the discrete transformations under charge conjugation $\CSym$,
parity $\PSym$, and time reversal $\TSym$. The \CPT theorem states that the \CPT
symmetry holds for any Lorentz invariant, local quantum field theory, whereas
the violation of each of the discrete symmetries $\CSym$, $\PSym$, or $\TSym$ is
still possible \cite{set:cpt}.

Lee and Yang where the first to question the general assumption that all
physical interactions are invariant even under a single transformation by
stating a possible $\PSym$ violation in weak interactions \cite{Lee:1956qn}.
Soon after, this statement was supported by the measurement of $\PSym$ violation
in the $\beta^{-}$ decay of \cobaltsixty \cite{Wu:1957my}. The experiment
revealed that the weak interaction only couples to fermions with left-handed
chirality or anti-fermions with right-handed chirality, which means that
symmetry under the $\CSym$ transformation is violated as well. Still, the
interaction seemed to behave invariant under combined $\CP$ transformations. But
already in 1964, the symmetry violation under $\CP$ transformation was observed
in the neutral kaon system and consequently established in the \SM
\cite{Christenson:1964fg}.

However, writing the source of \CP violation into the theoretical description of
the  \SM still was a challenge. The triumphant idea came from Makoto Kobayashi
and Toshihide Maskawa proposing a third quark generation as a possibility to
explain \CP violation in the framework of the electroweak theory
\cite{Kobayashi:1973fv}. The discovery of the \bquark quark
\cite{Herb:1977ek} provided enough confidence in their theory to start planning
the two \BFactories \Babar and \Belle. At the two $\elel$ colliders, PEP-II
(Stanford, CA, US) and KEKB (Tsukuba, JP), electrons and positrons were brought
to collisions at asymmetric beam energies. Both machines operated at
centre-of-mass energies of exactly the \YFourS bottonium resonance at
$\SI{10.58}{\GeV}$, producing $\B\Bbar$ meson pairs at high rates of $\num{e6}$
a day \cite{Bevan:2014iga}.

To establish \CP violation in the \SM it was necessary to extend
the measurements outside the kaon sector to the heavier \B and \D meson systems.
Predominantly built to study \CP violation in the \Bmeson system, \Babar and
\Belle had soon been successful and reported the observation of \CP violation in
decays of \Bd mesons to charmonium final states in 2001
\cite{Aubert:2001nu,Abe:2001xe}.

The \enquote{golden observable} \cite{Bevan:2014iga} of \CP violation in the \Bd
meson system is the decay-time dependent \CP asymmetry between the  \BdToJpsiKS
and \BdbarToJpsiKS decay rates
%
\begin{equation*}
    \CPAsymmetry(\obsTime) \equiv 
      \frac{\Gamma\bigl(\decay{\Bdbar(\obsTime)}{\Jpsi\KS}\bigr) - \Gamma\bigl(\decay{\Bd(\obsTime)}{\Jpsi\KS}\bigr)}
           {\Gamma\bigl(\decay{\Bdbar(\obsTime)}{\Jpsi\KS}\bigr) + \Gamma\bigl(\decay{\Bd(\obsTime)}{\Jpsi\KS}\bigr)} \eqcm
\end{equation*}
%
that can be related to the angle $\beta$ of the \CKM quark mixing matrix as the
source of \CP violation in the \SM \cite{Aaij:2015vza}. In recent years, the
value of \sintwobeta was measured to a precision of below $\SI{3}{\percent}$
with a world average of $\sintwobeta = \num{0.682 +- 0.019}$, dominated by the
measurements of \Babar and \Belle \cite{Amhis:2014hma}. Taking into account
measurements of other parameters that constrain the value of \sintwobeta to
$0.711\pm^{0.017}_{0.041}$
\cite{Charles:2015gya}, a small discrepancy to the direct measurement is
observed.

The presence of this statistically insignificant but visible tension illustrates
the principle idea of precision measurements of \CKM parameters. Although, the
conditions presented by Sakharov are met in the \SM, the magnitude of \CP
violation is not large enough to sufficiently explain the baryon asymmetry.
Consequently, the source of the \CP violation has to originate from physics
beyond the \SM. One way to cope with this, is the direct search for new
particles at higher mass scales. However, this has not been very successful
until now. Another promising way is to conduct indirect searches. Quantum loop
processes may contain contributions from heavy particles even if they take place
at far lower energies than necessary to produce those directly, thus allowing to
get a peek on physics scales far above the current experimental limits.
Therefore, together with precise theory predictions, possible deviations from
the \SM can appear in precision measurements of \SM parameters.

To meet the high demands defined by a precision measurement the experimental
setup has to be customised to this very purpose. In case of decay-time dependent
measurements, the spatial and time resolution has to be excellent to resolve
production and decay vertices, which allow for an efficient and precise
reconstruction of particle trajectories and decays. Additionally the \PID has to
be very efficient, with low mis-identification rates in order to correctly
identify final state particles.

The \LHCb detector located at the \LHC at the \CERN was designed with these
requirements in mind and operates very successfully since 2009. Following its
predecessors \Babar and \Belle, the \LHCb experiment aims to study heavy flavour
physics by exploiting direct and indirect experimental techniques to their full
extent. The physics program includes \CP violation measurements in $\B$ and $\D$
meson decays
\cite{Aaij:2015tza,Aaij:2015yda,Aaij:2014uva,Aaij:2014fba,Aaij:2014dka,Aaij:2014zsa,Aaij:2014kxa}, 
the search for (very) rare decays \cite{CMS:2014xfa}, tests of
lepton-universality \cite{Aaij:2014ora,Aaij:2015yra}, as well as heavy flavour
spectroscopy \cite{Aaij:2014yka,Aaij:2015tga}.

The analysis presented in this thesis describes the precise determination of the
\CKM parameter \sintwobeta in a measurement of the decay-time dependent \CP
asymmetry $\CPAsymmetry(\obsTime)$ in the decay of \Bd and \Bdbar mesons into
their common $\Jpsi\KS$ final state. Flavour tagging algorithms are used to
determine the state of the \B meson at production and are therefore of critical
importance to the measurement of the \CP asymmetry. The \LHCb experiment is the
only detector capable of sophisticated flavour tagging in a hadronic environment
and sets performance benchmarks for flavour tagged \CP analyses at hadron
colliders.

The measurement is conducted on a dataset collected by the \LHCb experiment during
the first \LHC run period in 2011 and 2012. Based on a dataset corresponding to
an integrated luminosity of $\SI{3.0}{\per\femto\barn}$ collected at a
centre-of-mass energy of $\num{7}$ and $\SI{8}{\TeV}$ the measurement supersedes
a previous measurement by \LHCb \cite{Aaij:1497268} on a subset of the data
corresponding to an integrated luminosity of $\SI{1.0}{\per\femto\barn}$.

The updated analysis makes use of the larger dataset, an optimized selection,
and adds an additional flavour tagging algorithm to identify the quark content
of the \Bmeson at production. Therewith, the statistical power of the
measurement can be increased by almost a factor of $\num{6}$.

\subsubsection*{Collaboration}
The work presented in this document was only possible to achieve in close
collaboration with colleagues from the \acs{LHCb} collaboration, most notably
Frank Meier and Julian Wishahi from the local working group, they are---together
with the author---the contact authors of this analysis published as:
%
\begin{quotation}
  \fullcite{Aaij:2015vza}.
\end{quotation}

Mirco Dorigo and Ulrich Eitschberger from the flavour tagging group provided the
flavour tagging calibration measurements (\cf
\cref{sec:flavour_tagging:calibration:os,sec:flavour_tagging:calibration:ss}).
The calculation of the kaon regeneration (\cf
\cref{sec:measurement_of_sin2beta:cpv_measurement:kaon_regeneration}) was
performed by Jeroen van Tilburg from the \B to charmonium working group.

\subsubsection*{Outline}

\Cref{ch:cpv_theory} shortly recapitulates the \SM with an emphasis on flavour
physics and \CP violation in the quark sector. The nature of Yukawa interactions
and the appearance of the \CKM matrix through spontaneous symmetry breaking are
summarised. Then, the time evolution of \Bmesons including decay and flavour
oscillations is outlined. Finally, an overview of the measurement of
\CP violation in the decay of \BdToJpsiKS is given.

The \LHCb detector and its subsystems are briefly described in
\cref{ch:lhcb_experiment} with particular attention to the track reconstruction
and \acl{PID} techniques. The \LHCb trigger system is sketched and the software
stack is depicted. Also, the data taking in the first successful years
of running is summarised.

Playing an important role in this analysis, the flavour tagging algorithms
utilised at \LHCb are discussed in more detail in \cref{ch:flavour_tagging}.
After a general overview and a comparison to developments at the \BFactories,
the flavour tagging algorithms are explained. Afterwards, the calibration of the
flavour tagging output is motivated and described. The chapter closes with an
outlook on recent developments.

The main part (\cref{ch:measurement_of_sin2beta}) contains the details of the
performed measurement. The data taking and preparation is explained as well as
studies of potential background contributions. The influence of decay time
acceptance and resolution effects is briefly summarised. Before the results are
presented, the likelihood function is described. Finally, all studies of
systematic effects are collected.

The results are summarised and put into context to the global \CKM picture and
prospects for \RunTwo and the \LHCb detector upgrade are presented.
