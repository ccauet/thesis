%!TEX root = ../../common/main.tex

\section{Flavour tagging}
\label{sec:measurement_of_sin2beta:flavour_tagging}

The flavour tagging employed is described in detail in
\cref{ch:flavour_tagging}. Two different tagging decisions are utilised: the
\acf{OS} tagger combination and the \acf{SSpi} tagger. The calibration and
subsequent combination of the two tagging decisions is performed within the fit
(\cf \cref{sec:measurement_of_sin2beta:likelihood_fit}). To assure correct
propagation of uncertainties and correlations of the calibration parameters to
the \CP parameters Gaussian constraints based on the calibration parameters'
central values and theirs statistical uncertainties are applied. The influence
of the systematic uncertainties on the calibration parameters is described in
\cref{sec:measurement_of_sin2beta:systematics:systematics}.

% %%%%%%%%%%%%%%%%%%%%%%%%%%%%%%%%%%%%%%%%%%%%%%%%%%%%%%%%%%%%%%%%%%%%%%%%%%%%%%
\subsection{Performance}
\label{sec:measurement_of_sin2beta:flavour_tagging:performance}

The effective tagging efficiency
(\cref{eq:flavour_tagging:efftageff})---inherent in the used data sample given
the choice of utilised taggers---is calculated as the sum of the effective
tagging efficiencies carried by each (including untagged) signal candidate
%
\begin{equation}
  \efftageff = \frac{\sum_{i=0}^{N} w_i \tagdilution_i^2\big(\tagOS, \tagSS, \mistagOS, \mistagSS \big)}{\sum_{i=0}^{N} w_i}\eqcm
\end{equation}
%
where $\mistagOS$ and $\mistagSS$ refer to the calibrated mistag estimates. The
signal \sWeights $w_i$ for each candidate are computed from an \sPlot fit to the
$\Bd$ mass distribution using the nominal mass \PDF described in
\cref{sec:measurement_of_sin2beta:likelihood_fit}. This adds up to an effective
tagging efficiency of $\efftageff = \SI{3.02 +- 0.05}{\percent}$, which splits
into a tagging efficiency of $\tageff = \SI{36.54
+- 0.14}{\percent}$ and an effective dilution of $\tagdilution = \SI{28.75 +-
0.24}{\percent}$ corresponding to an average mistag probability of $\mistag =
\SI{35.62 +- 0.12}{\percent}$.

\Cref{tab:flavour_tagging:performance:numbers} lists the tagging efficiency
separate for the \catOS, \catSS, and \catBS parts of the dataset, as well as for
the case where all \SSpi decisions are ignored and only the \OS tagging
information are exploited (\enquote{all \OS}) and vice versa (\enquote{all
\SSpi}). This allows the performance obtained on the given dataset to be
compared with the results of the previous \LHCb measurement \cite{Aaij:1497268},
where only the tagging information provided by the \OS algorithms were used.
Here, relative improvements of $\SI{10}{\percent}$ from $\SI{2.38}{\percent}$ to
$\SI{2.63}{\percent}$ can be observed. Finally, supplementing the measurement
with information from the \SSpi tagger allows for an overall improvement in
effective tagging efficiency of almost $\SI{27}{\percent}$.
%
\begin{table}
  \centering
  \caption{Effective tagging efficiency given for all three tagging categories.
  Additionally listed are the results just exploiting the information of the \OS
  or \SSpi tagging algorithms, as well as the total effective tagging efficiency
  of the dataset.}
  \label{tab:flavour_tagging:performance:numbers}
  \begin{tabular}{
      S[table-text-alignment = left]
      S
      [
        table-number-alignment = center,
        table-align-uncertainty = true,
        table-figures-decimal = 3,
        table-figures-uncertainty = 1,
      ]
    }
    \toprule
    {Tagger}        & {\efftageff [$\%$]}\\
    \midrule
    {\catOS}        & 2.259 +- 0.034 \\
    {\catSS}        & 0.262 +- 0.017 \\
    {\catBS}        & 0.503 +- 0.010 \\
    \midrule
    {all \OS}       & 2.63  +- 0.04  \\
    {all \SSpi}     & 0.376 +- 0.024 \\
    \midrule
    {Total}         & 3.02  +- 0.05  \\
    \bottomrule
  \end{tabular}
\end{table}
% 