%!TEX root = ../../common/main.tex

\section{Likelihood fit}
\label{sec:measurement_of_sin2beta:likelihood_fit}

This section presents the model developed to describe the data and estimate the
values and uncertainties of the physics observables using an \uEML fit. Starting
with a short review of the \uEML method, the \acp{PDF} in use to describe the
different dimensions and components is shown, and the fit model is outlined.

The \CP observables of interest \SJpsiKS and \CJpsiKS are estimated in a
multi-dimensional simultaneous \uEML fit. As summarised in
\cref{tab:measurement_of_sin2beta:data_preparation:observables} the seven
observables $\vec{x}$ are given by the reconstructed mass and decay time of the
\Bd candidate, its decay time estimate, and its \OS and \SSpi tag decision as
well as the corresponding mistag estimates
%
\begin{equation}\label{eq:measurement_of_sin2beta:likelihood_fit:observables}
  \vec{x} = (\obsMass, \obsTime, \obsTimeError, \obsTagOS, \obsTagSS, \obsEtaOS, \obsEtaSS)\eqpd  
\end{equation}
%
The extended likelihood for $N$ observed events is then defined as
%
\begin{equation}\label{sec:measurement_of_sin2beta:likelihood_fit:uEML}
  \Likelihood{}{}\left(\vec{\theta},\vec{n};\vec{x}\right) = \frac{\exponential{-N} N^n}{n!} \prod_j \prod_{i=1}^{n_j} \Prob{j}{}\left(\vec{x}_i;\vec{\theta}\right)\eqcm
\end{equation}
%
where $\vec{n}$ is the numbers of events, and $\vec{\theta}$ the parameters with
unknown values to be estimated by the \uEML fit.

\newpage
true parameter values == point estimation / best guess
+ estimation of uncertainties (usually expressed as confidence intervals)

dataset of N observations with observables $\vec{x} = {x_1, x_2, x_3}$

(unknown) PDF f(x, vec(theta)) with parameters vec(theta)




% ------------------------------------------------------------------------------
\subsection{Fitter validation}
\label{sec:measurement_of_sin2beta:likelihood_fit:validation}
