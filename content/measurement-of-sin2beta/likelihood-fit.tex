%!TEX root = ../../common/main.tex

\section{Likelihood fit}
\label{sec:measurement_of_sin2beta:likelihood_fit}

This section presents the model developed to describe the data and estimate the
values and uncertainties of the physics observables using an \uEML fit. Starting
with a short review of the \uEML method, the \acp{PDF} in use to describe the
different dimensions and components is shown, and the fit model is outlined.

The \CP observables of interest \SJpsiKS and \CJpsiKS are estimated in a
multi-dimensional simultaneous \uEML fit. As summarised in
\cref{tab:measurement_of_sin2beta:data_preparation:observables} the seven
observables $\vec{x}$ are given by the reconstructed mass and decay time of the
\Bd candidate, its decay time estimate, and its \OS and \SSpi tag decision as
well as the corresponding mistag estimates
%
\begin{equation}\label{eq:measurement_of_sin2beta:likelihood_fit:observables}
  \vec{x} = (\obsMass, \obsTime, \obsTimeError, \obsTagOS, \obsTagSS, \obsEtaOS, \obsEtaSS)\eqpd  
\end{equation}
%
The extended likelihood for a total of $N$ observed events in $j$ subsamples of
data, where $n=\sum_j n_j$, is then defined as
%
\begin{equation}\label{eq:measurement_of_sin2beta:likelihood_fit:uEML}
  \Likelihood{}{}\left(\vec{\theta},\vec{n};\vec{x}\right) = \frac{\exponential{-N} N^n}{n!} \prod_j \prod_{i=1}^{n_j} n_j \Prob{j}{}\left(\vec{x}_i;\vec{\theta}\right)\eqcm
\end{equation}
%
where $\vec{n}$ is the numbers of events, and $\vec{\theta}$ the parameters with
unknown values to be estimated by the \uEML fit. \important{Check definition of uEML}

\subsection{List of used \aclp{PDF}}
\label{sec:measurement_of_sin2beta:likelihood_fit:pdfs}

The following \acp{PDF} are employed in the fit to model the data.

\subsubsection{Exponential}
\label{sec:measurement_of_sin2beta:likelihood_fit:pdfs:exponential}

The exponential function with the parameter $\alpha$ to describe \eg the
distribution of combinatorial background candidates with masses $m$.
%
\begin{equation}\label{eq:measurement_of_sin2beta:likelihood_fit:pdfs:exponential}
  \Prob{\text{Exponential}}{}(m, \alpha) = \exponential{\alpha m}
\end{equation}

\subsubsection{Decay}
\label{sec:measurement_of_sin2beta:likelihood_fit:pdfs:decay}

The decay function is derived from the exponential function with $\alpha =
-\sfrac{1}{\tau}$, where $\tau$ describes the lifetime of candidates with decay
times $t$.
%
\begin{equation}\label{eq:measurement_of_sin2beta:likelihood_fit:pdfs:decay}
  \Prob{\text{Decay}}{}(t, \tau) = \exponential{- \frac{t}{\tau}}
\end{equation}

\subsubsection{Gaussian}
\label{sec:measurement_of_sin2beta:likelihood_fit:pdfs:gaussian}

A simple Gaussian function described by the central value or mean $\mu$ and the
width $\sigma$ of the distribution.
%
\begin{equation}\label{eq:measurement_of_sin2beta:likelihood_fit:pdfs:gaussian}
  \Prob{\text{Gaussian}}{}(x, \mu, \sigma) = \frac{1}{\sigma\sqrt{2\pi}}\exponential{-\frac{1}{2} \left(\frac{x-\mu}{\sigma}\right)^2}
\end{equation}

\subsubsection{Lognormal}
\label{sec:measurement_of_sin2beta:likelihood_fit:pdfs:lognormal}

The lognormal function is \eg used to describe the distribution of the decay
time error estimate $\obsTimeError$ and is parametrised by its median $\mu$ and
the parameter $k = \exponential{\sigma}$, where $\sigma$ is named the shape
parameter.
%
\begin{equation}\label{eq:measurement_of_sin2beta:likelihood_fit:pdfs:lognormal}
  \Prob{\text{Lognormal}}{}(\obsTimeError, \mu, k) = \frac{1}{\obsTimeError \sqrt{2\pi} \log (k)} \exponential{-\frac{\log^2 \left(\sfrac{\obsTimeError}{\mu}\right)}{2 \log^2(k)}}
\end{equation}

\subsubsection{BDecay}
\label{sec:measurement_of_sin2beta:likelihood_fit:pdfs:bdecay}

Generalised exponential function to describe the time evolution of \Bmeson
states with decay times $t$. The coefficients $A$, $B$, $C$, and $D$ can be
adapted to describe \Bmeson mixing, \CP violation, and different asymmetries \eg
in the production of the \Bmesons. The \PDF is further parametrised by the
lifetime parameter $\tau$, the decay width difference $\DG$ and the mass
difference $\dm$ of the \Bmeson mass eigenstates.
%
\begin{multline}\label{eq:measurement_of_sin2beta:likelihood_fit:pdfs:bdecay}
  \Prob{\text{BDecay}}{}(t;...) = \\ \exponential{- \frac{t}{\tau}}\left( A \cosh (\DG t) + B \sinh (\DG t) + C \cos (\dm t) + D \sin (\dm t) \right)
\end{multline}

\subsubsection{Ipatia}
\label{sec:measurement_of_sin2beta:likelihood_fit:pdfs:ipatia}

The Ipatia \PDF \cite{Santos:2013gra} is a generalisation of the Crystal ball
\PDF \cite{Oreglia:1980cs,Gaiser:1982yw,Skwarnicki:1986xj}, marginalised over
the a priori unknown per-event mass resolution and is parametrised as
%
\begin{multline}\label{eq:measurement_of_sin2beta:likelihood_fit:pdfs:ipatia}
  \Prob{\text{Ipatia}}{}(x, \mu, \sigma, \lambda, \zeta, \beta, a_1, a_2, n_1, n_2) = \\
    \begin{cases}
      G(x, \mu, \sigma, \lambda, \zeta, \beta)    & \text{if $-a_1 < \frac{x-\mu}{\sigma} < a_2$} \\
      \frac{G(\mu - a_1 \sigma, \mu, \sigma, \lambda, \zeta, \beta)}{
        \left( 1 - \sfrac{x}{\left( n_1 \frac{G(\mu - a_1 \sigma, \mu, \sigma, \lambda, \zeta, \beta)}{G^\prime(\mu - a_1 \sigma, \mu, \sigma, \lambda, \zeta, \beta)} - a_1 \sigma \right)} \right)^{n_1}
      }     & \text{if $-a_1 > \frac{x-\mu}{\sigma}$} \\
      \frac{G(\mu - a_2 \sigma, \mu, \sigma, \lambda, \zeta, \beta)}{
        \left( 1 - \sfrac{x}{\left( n_2 \frac{G(\mu - a_2 \sigma, \mu, \sigma, \lambda, \zeta, \beta)}{G^\prime(\mu - a_2 \sigma, \mu, \sigma, \lambda, \zeta, \beta)} - a_2 \sigma \right)} \right)^{n_2}
      }     & \text{if $\phantom{-}a_2 < \frac{x-\mu}{\sigma}$} \\
  \end{cases}
\end{multline}
%
where $G(x, \mu, \sigma, \lambda, \zeta, \beta)$ defines the generalised hyperbolic function
\begin{multline}\label{eq:measurement_of_sin2beta:likelihood_fit:pdfs:generalised_hyperbolic}
  G(x, \mu, \sigma, \lambda, \zeta, \beta) = \\
  \left(\left(x - \mu\right)^2 + A_\lambda^2(\zeta) \sigma^2 \right)^{\frac{1}{2} \lambda - \frac{1}{4}}
  \exponential{\beta (x - \mu)} K_{\lambda-\frac{1}{2}}
  \left(\zeta \sqrt{1 + \left(\sfrac{x - \mu}{A_\lambda(\zeta) \sigma}\right)^2} \right)\eqcm
\end{multline}
%
with the cylindrical harmonics $K_\lambda$ and
%
\begin{equation}
  A_\lambda^2(\zeta) = \frac{\zeta K_\lambda(\zeta)}{K_{\lambda+1}(\zeta)}\eqpd
\end{equation}


\newpage




% ------------------------------------------------------------------------------
\subsection{Fitter validation}
\label{sec:measurement_of_sin2beta:likelihood_fit:validation}
