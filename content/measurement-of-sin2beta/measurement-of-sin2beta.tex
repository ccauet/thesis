%!TEX root = ../../common/main.tex

\chapter{The measurement of sin2beta}
\label{ch:measurement_of_sin2beta}

\section{Overview of the measurement ingredients}
\label{sec:measurement_of_sin2beta:overview}
\missing{General idea of the measurement and the key concepts}

\section{Data preparation}
\label{sec:measurement_of_sin2beta:data_preparation}

The presented analysis is performed on a dataset collected by the \LHCb
experiment during two run periods in 2011 and 2012, dubbed
\enquote{\RunOne}\footnote{In contrast to the second \enquote{\RunTwo} period
scheduled to start in 2015 and last until mid 2018.}.

The 2011 subsample has an integrated luminosity of
$\SI[separate-uncertainty=true]{1.00\pm0.04}{\per\femto\barn}$ at a
centre-of-mass energy of $\sqrt{s}=\SI{7}{\TeV}$, while the 2012 subsample has
an integrated luminosity of
$\SI[separate-uncertainty=true]{1.990\pm0.024}{\per\femto\barn}$ at
$\sqrt{s}=\SI{8}{\TeV}$.

The $\BdToJpsiKS$ decay candidates are reconstructed in the $\JpsiToMuMu$ and
$\KSToPiPi$ final state. In order to enhance the purity of the signal
candidates, several selection steps are performed. At first a two-layer
\emph{trigger} system selects events from the \protonproton collisions (\cf
\cref{sec:measurement_of_sin2beta:data_preparation:trigger}). Then a loose and very general
selection, the so called \emph{stripping} is applied to the triggered data
written to tape (\cf
\cref{sec:measurement_of_sin2beta:data_preparation:stripping}). The final step
is an offline selection of the remaining candidates that is further adjusted to
the specific decay and the reconstructed final state (\cf
\cref{sec:measurement_of_sin2beta:data_preparation:offline_selection}).

Besides reducing the background level and enhancing the fraction of signal in
the data set, the selection tries to minimise a possible bias of the measured
\Bd lifetime caused by selection steps that preferably remove candidates with
small decay times. If it is inevitable, then at least the biasing effects should
be understood and accounted for (\cf
\cref{sec:measurement_of_sin2beta:resolution_and_acceptance:acceptance}).

\subsubsection*{Global decay chain fit}
In order to correctly comprise correlations and uncertainties on vertex
positions, particle momenta, flight distances, decay times, and invariant
masses, the \DTF \cite{Hulsbergen:2005pu} was used in the tuple processing.

Entities decorated with the expression \dtfpv stem from a \DTF fit where the
knowledge about the primary vertex has been used to constrain the production
vertex of the \Bd meson, while the expression \dtf means that additionally to
the PV constraint the \jpsi and \KS invariant masses are constrained to their
masses listed by the \PDG ($m_{\KS}^{\text{\acs{PDG}}}=\SI{497.614}{\MeVcc}$,
$m_{\jpsi}^{\text{\acs{PDG}}} = \SI[per-mode=symbol]{3096.916}{\MeVcc}$, see
\cite{Agashe:2014kda}) inside the \DTF fit. In the Stripping and if not stated
otherwise the \verb=OfflineVertexFitter= \addref{OfflineVertexFitter} was used.

\subsection{Trigger}
\label{sec:measurement_of_sin2beta:data_preparation:trigger}
L0, HLT1, HLT2 requirements, TCKs?, efficiencies

\subsection{Stripping}
\label{sec:measurement_of_sin2beta:data_preparation:stripping}
Stripping details, cuts, ...

\subsection{Offline selection}
\label{sec:measurement_of_sin2beta:data_preparation:offline_selection}
Selection cuts, FoM, efficiencies

\subsection{Multiple candidates}
Why do we see multiple candidates, difference between multiple PVs and multiple B candidates, how to remove them

\subsection{Simulated data}
Data sets used, software stack and versions, simulation conditions, DEC file, generation parameters

\section{Flavour tagging}
\label{sec:measurement_of_sin2beta:flavour_tagging}
\section{Backgrounds}
\label{sec:measurement_of_sin2beta:physic_backgrounds}
\section{Decay time resolution and acceptance}
\label{sec:measurement_of_sin2beta:resolution_and_acceptance}
\subsection{Resolution}
\label{sec:measurement_of_sin2beta:resolution_and_acceptance:resolution}
\subsection{Acceptance}
\label{sec:measurement_of_sin2beta:resolution_and_acceptance:acceptance}
\section{Likelihood fit}
\subsection{Fitter validation}
\section{CPV measurement}
\subsection{Fit results}
\subsection{Kaon regeneration}
\section{Study of systematic effects}
\subsection{Cross-checks}
\subsection{Systematics}
\subsection{Summary of systematic effects}
