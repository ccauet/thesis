%!TEX root = ../../common/main.tex

\chapter{The measurement of sin2beta}
\label{ch:measurement_of_sin2beta}

\section{Overview of the measurement ingredients}
\label{sec:measurement_of_sin2beta:overview}
\missing{General idea of the measurement and the key concepts}

\section{Data preparation}
\label{sec:measurement_of_sin2beta:data_preparation}

The presented analysis is performed on a dataset collected by the \LHCb
experiment during two run periods in 2011 and 2012, dubbed
\enquote{\RunOne}\footnote{In contrast to the second \enquote{\RunTwo} period
scheduled to start in 2015 and last until mid 2018.}.

The 2011 subsample has an integrated luminosity of
$\SI[separate-uncertainty=true]{1.00\pm0.04}{\per\femto\barn}$ at a
centre-of-mass energy of $\sqrt{s}=\SI{7}{\TeV}$, while the 2012 subsample has
an integrated luminosity of
$\SI[separate-uncertainty=true]{1.990\pm0.024}{\per\femto\barn}$ at
$\sqrt{s}=\SI{8}{\TeV}$.

The $\BdToJpsiKS$ decay candidates are reconstructed in the $\JpsiToMuMu$ and
$\KSToPiPi$ final state. In order to enhance the purity of the signal
candidates, several selection steps are performed. At first a two-layer
\emph{trigger} system selects events from the \protonproton collisions (\cf
\cref{sec:measurement_of_sin2beta:data_preparation:trigger}). Then a loose and very general
selection, the so called \emph{stripping} is applied to the triggered data
written to tape (\cf
\cref{sec:measurement_of_sin2beta:data_preparation:stripping}). The final step
is an offline selection of the remaining candidates that is further adjusted to
the specific decay and the reconstructed final state (\cf
\cref{sec:measurement_of_sin2beta:data_preparation:offline_selection}).

Besides reducing the background level and enhancing the fraction of signal in
the data set, the selection tries to minimise a possible bias of the measured
\Bd lifetime caused by selection steps that preferably remove candidates with
small decay times. If it is inevitable, then at least the biasing effects should
be understood and accounted for (\cf
\cref{sec:measurement_of_sin2beta:resolution_and_acceptance:acceptance}).

\subsubsection*{Global decay chain fit}
In order to correctly comprise correlations and uncertainties on vertex
positions, particle momenta, flight distances, decay times, and invariant
masses, a global decay chain fit (\acs{DTF}) \cite{Hulsbergen:2005pu} is
performed. Instead of fitting \enquote{leaf-by-leaf} starting from the final
state particles to determine the parameters of their composite mother particle,
then repeating this step until the initial \bhadron is reached, the \DTF fit
extracts all parameters in the decay chain simultaneously.

Entities decorated with the expression \dtfpv stem from a \DTF fit where the
knowledge about the \PV has been used to constrain the production
vertex of the \Bd meson, while the expression \dtf means that additionally to
the \PV constraint the \jpsi and \KS invariant masses are constrained to their
masses listed by the \PDG ($m_{\KS}^{\text{\acs{PDG}}}=\SI{497.614}{\MeVcc}$,
$m_{\jpsi}^{\text{\acs{PDG}}} = \SI[per-mode=symbol]{3096.916}{\MeVcc}$, see
\cite{Agashe:2014kda}) inside the \DTF fit. In the stripping and if not stated
otherwise the much simpler leaf-by-leaf fitting was used.

\subsubsection*{Observables}

Throughout the analysis, several observables are used: the \Bd invariant mass
$\obsMass$, the decay time of the \Bd in its rest frame $\obsTime$, the \Bd
decay time error estimate $\obsTimeError$, and the tag decisions
\obsTagOSSS and mistag estimates \obsEtaOSSS of the opposite site and same side
tagging algorithms (\cf \cref{ch:flavour_tagging}).
\Cref{tab:measurement_of_sin2beta:data_preparation:observables} summarises all
used observables, provides the considered observable ranges, and defines the
used fit constraints.

\begin{table}[!htb]
\centering
\caption{Used observables, observable ranges and \DTF fit properties.\info{checkmark does not work with font}}
\label{tab:measurement_of_sin2beta:data_preparation:observables}
\begin{tabular}{llccc}
\toprule
Name              & Range           & \multicolumn{3}{c}{Fit constraints} \\ 
                  &                 & $m_{\KS}^{\text{PDG}}$ & $m_{\jpsi}^{\text{PDG}}$ & \acs{PV} position \\
\midrule    
$\obsMass$        & $\SIrange[range-phrase = -, range-units = single]{5230}{5330}{\MeVcc}$ & \checkmark & \checkmark & \checkmark \\
$\obsTime$        & $\SIrange[range-phrase = -, range-units = single]{0.3}{18.3}{\pico\second}$ & - & - & \checkmark \\
$\obsTimeError$   & $\SIrange[range-phrase = -, range-units = single]{0.0}{0.2}{\pico\second}$ & - & - & \checkmark \\
$\obsTagOS$       & \num[retain-explicit-plus]{+1}, \num{-1} & - & - & - \\
$\obsEtaOS$       & $\SIrange[range-phrase = -]{0.0}{0.5}{}$ & - & - & - \\
$\obsTagSS$       & \num[retain-explicit-plus]{+1}, \num{-1} & - & - & - \\
$\obsEtaSS$       & $\SIrange[range-phrase = -]{0.0}{0.5}{}$ & - & - & - \\
\bottomrule
\end{tabular}
\end{table}

\subsubsection*{Data set subsamples}

\begin{description}
  \item[Year] The data consists of subsamples from data-taking in both the 2011
and 2012 run periods. To differentiate between these samples, they are indicated
by the terms \textbf{\catOO} and \textbf{\catOT}.

  \item[Track type] Due to the long lifetime of the \KS, its daughter pions may
or may not leave hits in the \VELO (\cf
\cref{sec:lhcb_experiment:tracking:techniques_and_performance}). \KS candidates
(and accordingly the associated \Bmeson candidate) of which both reconstructed
pions have hits in the \VELO are classified as \emph{long track}
(\textbf{\catLL}) candidates. If both pions do not leave hits in the \VELO, the
candidate is called \emph{downstream track} (\textbf{\catDD}) candidate.
Candidates where only one of the reconstructed pions has a long track are not
considered in this analysis.
  
  \item[Tagger] Depending on the tag decisions, events are categorised: as
  exclusively \emph{opposite side tagged} (\textbf{\catOS}) if \textbf{only} the
opposite side tagger combination returns a tag decision, as exclusively
\emph{same side pion tagged} (\textbf{\catSS}) if \textbf{only} the same side
pion tagger returns a decision, as tagged by both tagging algorithms (or
\emph{tagged by both sides} (\textbf{\catBS})), and as \emph{untagged}
(\textbf{\catUT}) if none of the used taggers returns a tag decision. See
\cref{ch:flavour_tagging} for an elaborated description.
  
  \item[Trigger] As described in \cref{sec:data_preparation:trigger:acceptance}, the
different trigger lines induce decay time acceptance effects. To describe these
effects independently the data is split into a sample where almost no decay time
acceptance effects are visible, called \emph{almost unbiased} subsample
(\textbf{\catAU}), and a sample that includes the major fraction of events
affected by a decay time acceptance, called \emph{exclusively biased} subsample
(\textbf{\catEB}).
\end{description}

Unless explicitly stated otherwise, no untagged (\textbf{\catUT}) events are
used. Therefore, splitting the data set in all possible categories results in a
total of $\num{24}$ disjoint subsamples.

\clearpage
\subsection{Trigger}
\label{sec:measurement_of_sin2beta:data_preparation:trigger}
L0, HLT1, HLT2 requirements, TCKs?, efficiencies

\subsection{Stripping}
\label{sec:measurement_of_sin2beta:data_preparation:stripping}
Stripping details, cuts, ...

\subsection{Offline selection}
\label{sec:measurement_of_sin2beta:data_preparation:offline_selection}
Selection cuts, FoM, efficiencies

\subsection{Multiple candidates}
Why do we see multiple candidates, difference between multiple PVs and multiple B candidates, how to remove them

\subsection{Simulated data}
Data sets used, software stack and versions, simulation conditions, DEC file, generation parameters

\section{Flavour tagging}
\label{sec:measurement_of_sin2beta:flavour_tagging}
\section{Backgrounds}
\label{sec:measurement_of_sin2beta:physic_backgrounds}
\section{Decay time resolution and acceptance}
\label{sec:measurement_of_sin2beta:resolution_and_acceptance}
\subsection{Resolution}
\label{sec:measurement_of_sin2beta:resolution_and_acceptance:resolution}
\subsection{Acceptance}
\label{sec:measurement_of_sin2beta:resolution_and_acceptance:acceptance}
\section{Likelihood fit}
\subsection{Fitter validation}
\section{CPV measurement}
\subsection{Fit results}
\subsection{Kaon regeneration}
\section{Study of systematic effects}
\subsection{Cross-checks}
\subsection{Systematics}
\subsection{Summary of systematic effects}
