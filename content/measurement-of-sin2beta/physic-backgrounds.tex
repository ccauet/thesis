%!TEX root = ../../common/main.tex

\section{Backgrounds}
\label{sec:measurement_of_sin2beta:physic_backgrounds}

This section describes the evaluation of background contributions in the data.
Besides combinatorial background from particles combined by chance with
properties that allow to pass all selection steps two other sources are
investigated: physics backgrounds from decays with mis-identified final state
particles are studied in
\cref{sec:measurement_of_sin2beta:physic_backgrounds:physic_backgrounds} and the
influence of non-uniform tagging responses for the background candidates is
examined in
\cref{sec:measurement_of_sin2beta:physic_backgrounds:tagging_asymmetries}.

% ------------------------------------------------------------------------------
\subsection{Physics backgrounds}
\label{sec:measurement_of_sin2beta:physic_backgrounds:physic_backgrounds}

To ensure that the background consists of combinatorial backgrounds, and that
exclusive backgrounds are negligible, possible physics backgrounds are studied
with simulated data (see \cref{sec:measurement_of_sin2beta:data_preparation:datasamples:mc}). 

Using the true particle identities from \MC reveals the existence of remnants
from physic backgrounds in the $\inclJPsi$ and the $\BdToJpsiX$ \MC samples. In
the $\BdToJpsiX$, around $\SI{0.04}{\percent}$ $\BdToJpsiKstar$ candidates
survive the applied selection criteria compared to $\SI{0.5}{\percent}$ signal
events. In the $\inclJPsi$ sample $\SI{0.0005}{\percent}$ $\BdToJpsiKstar$,
$\SI{0.0002}{\percent}$ $\BsToJpsiKS$, and $\SI{0.0002}{\percent}$
$\LbToJpsiLambda$ background events contribute, compared to
$\SI{0.01}{\percent}$ $\BdToJpsiKS$ candidates.

Since the production of $\Bs$ is a factor $\num{100}$ less frequent than the
$\Bd$ and reconstructed $\BsToJpsiKS$ candidates lie outside the nominal mass
range, we neglect possible contributions. While $\BdToJpsiKstar$ background
results from kaon-pion mis-identification, $\LbToJpsiLambda$ candidates are
found in the sample due to a proton-pion mis-identification. Their expected
contribution in the data sample is further studied by using signal \MC samples
of $\BdToJpsiKstar$ and $\LbToJpsiLambda$.

\subsubsection{Physics backgrounds from $\mathbfsfit{\kaon}$-$\mathbfsfit{\pion}$ mis-ID}
\label{sec:measurement_of_sin2beta:physic_backgrounds:physic_backgrounds:kstar}

In the $\BdToJpsiKstar$ signal \MC $\SI{0.3}{\percent}$ of $\num{4e6}$ generated
background candidates get reconstructed and survive the applied stripping cuts.
This number already reduces to $\SI{0.004}{\percent}$ after applying cuts on the
nominal mass and time ranges. After full offline selection, considering a
tagging efficiency of $\SI{38}{\percent}$, and scaling the numbers to match a
data sample corresponding to
$\SI[separate-uncertainty=true]{3}{\per\femto\barn}$ integrated luminosity,
$\num{20}$ $\BdToJpsiKstar$ candidates remain. This corresponds to a fraction of
roughly $\SI{0.04}{\percent}$ of the background in the nominal data sample. The
source of the remaining contributions are \eg decays like
$\decay{\Bdstar}{\Bd(\to\jpsi\Kstarz)\,\Xparticle}$.

\subsubsection{Physics backgrounds from $\mathbfsfit{\proton}$-$\mathbfsfit{\pion}$ mis-ID}
\label{sec:measurement_of_sin2beta:physic_backgrounds:physic_backgrounds:lambda}

Out of $\num{3e6}$ generated $\LbToJpsiLambda$ signal candidates only
$\SI{2}{\percent}$ get reconstructed and pass the stripping. The cuts on the
nominal mass and time ranges reduce this number down to $\SI{0.32}{\percent}$.
The veto cut applied in the offline selection (see
\cref{sec:measurement_of_sin2beta:data_preparation:offline_selection:daughters})
decreases the number of candidates inside this mass and decay time window by
more than $\SI{50}{\percent}$ to $\SI{0.15}{\percent}$. After applying the full
offline selection, including the tagging efficiency, and scaling the sample to
correspond to the collected integrated luminosity in data, $\num{120}$
$\LbToJpsiLambda$ candidates remain. Therefore, we expect a $\SI{0.2}{\percent}$
contribution of $\LbToJpsiLambda$ background candidates in our nominal data
sample.

% ------------------------------------------------------------------------------
\subsection{Background tagging asymmetries}
\label{sec:measurement_of_sin2beta:physic_backgrounds:tagging_asymmetries}

As shown in the previous section any physics induced background in the data
sample can be neglected and the remaining background candidates are assumed to
be purely of combinatorial origin. Following this, one can conclude that all
flavour tagging decisions provided for background candidates are randomly
distributed. This assumption can be tested in two ways: counting the
time-integrated numbers of \Bd and \Bdbar tagged background candidates or
checking for a non-vanishing time-dependent tag asymmetry in the background. 

The tests are performed independently for the \catDD and \catLL sub sample as
well as for both employed tagging algorithms (\OS and \SSpi). In order to study
the background sample an unbinned maximum likelihood fit is performed on the \Bd
mass distribution and background \sWeights are computed. The signal component is
modelled using an Ipatia \PDF while the background component is described by an
exponential \PDF (\cf \cref{sec:measurement_of_sin2beta:likelihood_fit:model:mass}).

% ..............................................................................
\subsubsection{Time integrated background asymmetry}
\label{sec:measurement_of_sin2beta:physic_backgrounds:tagging_asymmetries:time_integrated}

The time-integrated asymmetry
%
\begin{equation}\label{eq:measurement_of_sin2beta:physic_backgrounds:tagging_asymmetries:time_integrated}
  \Asym{\Bkg}{\text{int}} = \frac{N_{\Bkg}^{\Bdbar} - N_{\Bkg}^{\Bd}}{N_{\Bkg}^{\Bdbar} + N_{\Bkg}^{\Bd}}
\end{equation}
%
is computed for all four sub samples using the \sweighted dataset and the
results are collected in 
\cref{tab:measurement_of_sin2beta:physic_backgrounds:tagging_asymmetries:time_integrated}. 
It can be concluded that except for the \catDD \OS subsample no significant
asymmetry is present, while the result for this single category might be
explained by a statistical fluctuation. A vanishing time-integrated asymmetry
does not eliminate the possibility of a time-dependent asymmetry, thus further
studies are necessary.
%
\begin{table}[h]
  \centering
  \caption{Time-integrated asymmetry of \sweighted background distributions for
  \catDD and \catLL \OS and \SSpi tagged events.}
  \label{tab:measurement_of_sin2beta:physic_backgrounds:tagging_asymmetries:time_integrated}
  \begin{tabular}{llr@{$\,\pm\,$}l}
    \toprule
    \multicolumn{2}{c}{category} & \multicolumn{2}{c}{$\Asym{\Bkg}{\text{int}}$}\\
    \midrule
    \multirow{2}{*}{\catDD} & \acs*{OS}    & $0.017$     & $0.005$ \\
                            & \acs*{SSpi}  & $-0.016$    & $0.011$ \\
    \multirow{2}{*}{\catLL} & \acs*{OS}    & $-0.005$    & $0.012$ \\
                            & \acs*{SSpi}  & $0.044$     & $0.034$ \\
    \bottomrule
  \end{tabular}
\end{table}

% ..............................................................................
\subsubsection{Decay time dependent background asymmetry}
\label{sec:measurement_of_sin2beta:physic_backgrounds:tagging_asymmetries:time_dependent}

The decay time dependent background tagging asymmetry is defined similar to 
\cref{eq:measurement_of_sin2beta:physic_backgrounds:tagging_asymmetries:time_integrated} as
%
\begin{equation}\label{eq:measurement_of_sin2beta:physic_backgrounds:tagging_asymmetries:time_dependent}
  \Asym{\Bkg}{}(\obsTime) = \frac{N_{\Bkg}^{\Bdbar}(\obsTime) - N_{\Bkg}^{\Bd}(\obsTime)}{N_{\Bkg}^{\Bdbar}(\obsTime) + N_{\Bkg}^{\Bd}(\obsTime)}\eqpd
\end{equation}
%
At first histograms are consulted to check for the null-hypothesis using a
$\chisq$-test. The test is performed on the background \sweighted dataset as
well as on a background \sweighted cocktail \MC sample built from signal \MC and
background candidates from a \ToyMC sample generated with randomly distributed
tag decisions. Finally, a likelihood fit to the weighted background sample is
performed using a \ac{PDF} $\ProbArg{\text{TagAsym}}{}{\obsTime, \tagdecision}$
modelling a potential tag asymmetry.
%
\begin{equation}\label{eq:measurement_of_sin2beta:physic_backgrounds:tagging_asymmetries:time_dependent:pdf}
  \ProbArg{\text{TagAsym}}{}{\obsTime, \tagdecision} \propto \exponential{- \frac{t}{\tau}} \left( 1 + \tagdecision \Asym{}{} \right)
\end{equation}

\paragraph{Histograms of the background candidates} binned in the reconstructed
decay time are created for \catDD/\catLL candidates with \OS and \SSpi tag
decisions. This is done for the \sweighted data sample and for a cocktail \MC
sample, where the \sWeights are extracted using the same model as used before.
\Cref{fig:measurement_of_sin2beta:physic_backgrounds:tagging_asymmetries:data,fig:measurement_of_sin2beta:physic_backgrounds:tagging_asymmetries:toymc} 
show the histograms. The corresponding $p$-values from a $\chisq$-test are
summarised in \cref{tab:measurement_of_sin2beta:physic_backgrounds:tagging_asymmetries:time_dependent:chisq}.
%
\begin{table}[h]
  \centering
  \caption{Resulting $p$-values from a $\chisq$-test for the time-integrated
  asymmetry. The values were computed on \sweighted background distributions for
  \catDD and \catLL \OS and \SSpi tagged events.}
  \label{tab:measurement_of_sin2beta:physic_backgrounds:tagging_asymmetries:time_dependent:chisq}
  \begin{tabular}{llSS}
    \toprule
                                 &        & \multicolumn{2}{c}{$p$-value} \\
    \multicolumn{2}{c}{category} & {data} & {cocktail \acs*{MC}} \\
    \midrule
    \multirow{2}[2]{*}{\catDD} & \acs*{OS}    &  0.100  &  0.525 \\
                               & \acs*{SSpi}  &  0.437  &  0.386 \\
    \multirow{2}[2]{*}{\catLL} & \acs*{OS}    &  0.617  &  0.110 \\
                               & \acs*{SSpi}  &  0.969  &  0.989 \\
    \bottomrule
  \end{tabular}
\end{table}
%
All $p$-values do not contradict the tested null-hypothesis. Increased
$p$-values for \catLL \SSpi are found both on data and cocktail \MC and might be
explained by the low statistics in this particular subsample.
%
\begin{figure}[h]
\includegraphics[width=0.49\textwidth]{private/content/measurement-of-sin2beta/figs/tagged_bkg_data_dd_os.pdf}
\includegraphics[width=0.49\textwidth]{private/content/measurement-of-sin2beta/figs/tagged_bkg_data_dd_ss.pdf}
\includegraphics[width=0.49\textwidth]{private/content/measurement-of-sin2beta/figs/tagged_bkg_data_ll_os.pdf}
\includegraphics[width=0.49\textwidth]{private/content/measurement-of-sin2beta/figs/tagged_bkg_data_ll_ss.pdf}
\caption{Tagging asymmetry $\Asym{}{}(\obsTime)$ in background \sweighted data.
The binning of the $x$-axis is chosen logarithmic. The top (bottom) plots show
\catDD (\catLL) candidates, while \OS (\SSpi) tagged candidates are depicted in
the left (right) side.}
\label{fig:measurement_of_sin2beta:physic_backgrounds:tagging_asymmetries:data}
\end{figure}
%
\begin{figure}[h]
\includegraphics[width=0.49\textwidth]{private/content/measurement-of-sin2beta/figs/tagged_bkg_toymc_dd_os.pdf}
\includegraphics[width=0.49\textwidth]{private/content/measurement-of-sin2beta/figs/tagged_bkg_toymc_dd_ss.pdf}
\includegraphics[width=0.49\textwidth]{private/content/measurement-of-sin2beta/figs/tagged_bkg_toymc_ll_os.pdf}
\includegraphics[width=0.49\textwidth]{private/content/measurement-of-sin2beta/figs/tagged_bkg_toymc_ll_ss.pdf}
\caption{Tagging asymmetry $\Asym{}{}(\obsTime)$ in cocktail \MC with a signal
candidates taken from \BdToJpsiKS signal \MC and background candidates from a
\ToyMC sample. The binning of the $x$-axis is chosen logarithmic. The top
(bottom) plots show \catDD (\catLL) candidates, while \OS (\SSpi) tagged
candidates are depicted in the left (right) side.}
\label{fig:measurement_of_sin2beta:physic_backgrounds:tagging_asymmetries:toymc}
\end{figure}

\paragraph{A likelihood fit} to the \sweighted background candidates is
conducted in order to exploit the per-event tag information. To do so the
distributions are fitted using a \ac{PDF} as presented in
\cref{eq:measurement_of_sin2beta:physic_backgrounds:tagging_asymmetries:time_dependent:pdf} 
where a potential asymmetry in data has been modelled. The parametrisation uses
the sum of three (two in case of the \catDD \SSpi subsample) exponential
\acp{PDF} with independent pseudo-lifetimes $\tau_i$ and individual asymmetry
parameters $\Asym{i}{}$.
%
\begin{table}[h]
\centering
\caption{Fit results of the asymmetries $\Asym{i}{}$ of a fit to the \sweighted
background distributions of \catDD and \catLL \OS and \SSpi tagged events.}
\label{sec:measurement_of_sin2beta:physic_backgrounds:tagging_asymmetries:time_dependent:likelihood:results}
\begin{tabular}{llr@{$\,\pm\,$}lr@{$\,\pm\,$}lr@{$\,\pm\,$}l}
\toprule
\multicolumn{2}{c}{category}  &   \multicolumn{2}{c}{$\Asym{1}{}$}  & \multicolumn{2}{c}{$\Asym{2}{}$}  & \multicolumn{2}{c}{$\Asym{3}{}$} \\
\midrule
\catDD & \acs*{OS}    &   $0.001$     &   $0.028$                 &   $-0.039$    &   $0.030$                 &   $-0.01$    &   $0.07$  \\
\catDD & \acs*{SSpi}  &   $-0.01$     &   $0.04$                  &   $0.14$      &   $0.14$                  &   \multicolumn{2}{c}{-}       \\
\catLL & \acs*{OS}    &   $0.05$      &   $0.05$                  &   $-0.03$     &   $0.13$                  &   $-0.11$     &   $0.12$  \\
\catLL & \acs*{SSpi}  &   $-0.01$     &   $0.20$                  &   $-0.18$     &   $0.18$                  &   $0.4$       &   $0.4$    \\
\bottomrule
\end{tabular}
\end{table}

\paragraph{A final conclusion} on how to proceed is difficult as the results are
inconclusive due to a lack of statistical power. Although all results are
compatible with the hypothesis of a vanishing background tagging asymmetry,
there is evidence of non-vanishing asymmetries supported by fluctuations found
in the results.

Thus, the nominal fit model does not incorporates a description of a potential
tagging asymmetry in the background component. The impact of this decision is
further investigated in a \ToyMC study presented in
\cref{sec:measurement_of_sin2beta:systematics:systematics:fit_model}
