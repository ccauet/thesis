%!TEX root = ../../common/main.tex

\section{Backgrounds}
\label{sec:measurement_of_sin2beta:physic_backgrounds}

This section describes the evaluation of background contributions in the data.
Besides combinatorial background from particles combined by chance with
properties that allow to pass all selection steps two other sources are
investigated: physics backgrounds from decays with mis-identified final state
particles are studied in
\cref{sec:measurement_of_sin2beta:physic_backgrounds:physic_backgrounds} and the
influence of non-uniform tagging responses for the background candidates is
examined in
\cref{sec:measurement_of_sin2beta:physic_backgrounds:tagging_asymmetries}.

% ------------------------------------------------------------------------------
\subsection{Physics backgrounds}
\label{sec:measurement_of_sin2beta:physic_backgrounds:physic_backgrounds}

To ensure that the background consists of combinatorial backgrounds, and that
exclusive backgrounds are negligible, possible physics backgrounds are studied
with simulated data (see \cref{sec:measurement_of_sin2beta:data_preparation:datasamples:mc}). 

Using the true particle identities from \MC reveals the existence of remnants
from physic backgrounds in the $\inclJPsi$ and the $\BdToJpsiX$ \MC samples. In
the $\BdToJpsiX$, around $\SI{0.04}{\percent}$ $\BdToJpsiKstar$ candidates
survive the applied selection criteria compared to $\SI{0.5}{\percent}$ signal
events. In the $\inclJPsi$ sample $\SI{0.0005}{\percent}$ $\BdToJpsiKstar$,
$\SI{0.0002}{\percent}$ $\BsToJpsiKS$, and $\SI{0.0002}{\percent}$
$\LbToJpsiLambda$ background events contribute, compared to
$\SI{0.01}{\percent}$ $\BdToJpsiKS$ candidates.

Since the production of $\Bs$ is a factor $\num{100}$ less frequent than the
$\Bd$ and reconstructed $\BsToJpsiKS$ candidates lie outside the nominal mass
range, we neglect possible contributions. While $\BdToJpsiKstar$ background
results from kaon-pion mis-identification, $\LbToJpsiLambda$ candidates are
found in the sample due to a proton-pion mis-identification. Their expected
contribution in the data sample is further studied by using signal \MC samples
of $\BdToJpsiKstar$ and $\LbToJpsiLambda$.

\subsubsection{Physics backgrounds from $\kaon$-$\pion$ mis-ID}
\label{sec:measurement_of_sin2beta:physic_backgrounds:physic_backgrounds:kstar}

In the $\BdToJpsiKstar$ signal \MC $\SI{0.3}{\percent}$ of $\num{4e6}$ generated
background candidates get reconstructed and survive the applied stripping cuts.
This number already reduces to $\SI{0.004}{\percent}$ after applying cuts on the
nominal mass and time ranges. After the full offline selection, considering a
$\SI{38}{\percent}$ tagging efficiency, and scaling the numbers to match a data
sample corresponding to $\SI[separate-uncertainty=true]{3}{\per\femto\barn}$
integrated luminosity, $\num{20}$ $\BdToJpsiKstar$ candidates with a positive
\MC information match remain. This corresponds to a fraction of roughly
$\SI{0.04}{\percent}$ of the background in the nominal data sample. The source
of the remaining contributions are \eg decays like
$\decay{\Bdstar}{\Bd(\to\jpsi\Kstarz)\,\Xparticle}$.

\subsubsection{Physics backgrounds from $\proton$-$\pion$ mis-ID}
\label{sec:measurement_of_sin2beta:physic_backgrounds:physic_backgrounds:lambda}

Out of $\num{3e6}$ generated $\LbToJpsiLambda$ signal candidates only
$\SI{2.0}{\percent}$ get reconstructed and pass the stripping. The cuts on the
nominal mass and time ranges reduce this number down to $\SI{0.32}{\percent}$.
The veto cut applied in the offline selection (see
\cref{sec:measurement_of_sin2beta:data_preparation:offline_selection:daughters})
decreases the number of candidates inside this mass and decay time window by
more than $\SI{50}{\percent}$ to $\SI{0.15}{\percent}$. After applying the full
offline selection, including the tagging efficiency, and scaling the sample to
correspond to the collected integrated luminosity in data, $\num{120}$ MC
truth-matched $\LbToJpsiLambda$ candidates remain. Therefore, we expect a
$\SI{0.2}{\percent}$ contribution of $\LbToJpsiLambda$ background candidates in
our nominal data sample.

% ------------------------------------------------------------------------------
\subsection{Background tagging asymmetries}
\label{sec:measurement_of_sin2beta:physic_backgrounds:tagging_asymmetries}
\missing{Background tagging asymmetries}
