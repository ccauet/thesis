%!TEX root = ../../common/main.tex

\section{Studies of systematic effects}
\label{sec:measurement_of_sin2beta:systematics}

The determination of potential systematic uncertainties on the \CP parameters
complete the measurement.

The following section will describe the evaluation of systematic effects of the
fit model and influences of the flavour tagging, the decay time resolution, the
decay time acceptance, and several other inputs to the measurement of \CP
violation.

In \cref{sec:measurement_of_sin2beta:systematics:cross_checks} cross-checks are
outlined to test the reproducibility and stability of the fit model. Studies to
estimate the effect of various fit model properties are listed in
\cref{sec:measurement_of_sin2beta:systematics:systematics} and
\cref{sec:measurement_of_sin2beta:systematics:summary} gives a summary of all
found systematic effects.

% ------------------------------------------------------------------------------
\subsection{Cross-checks}
\label{sec:measurement_of_sin2beta:systematics:cross_checks}

When performing cross-checks, either results obtained on different subsamples or
results obtained from different methods on the same sample are applied. In both
cases, the deviation of the difference in fit results in terms of its
uncertainty is used as a measure of agreement.

When comparing results on different, distinct samples, the uncertainty on the
difference is estimated by summing the uncertainties of the single results in
quadrature. In contrast, when applying different methods to the same sample, the
compatibility of the results is compared by taking into account a full
correlation of the data sets. Following Ref.~\cite{Barlow:2002yb}, the
uncertainty on the difference of the two results $A$ and $B$ on the same data
set is defined as
%
\begin{equation}
  \sigma^2_\Delta = \left\vert\sigma^2_{\text{A}} - \sigma^2_{\text{B}}\right\vert .
\end{equation}
%
Again, both measurements can be interpreted as compatible if the uncertainty
lies in the same order of magnitude as the difference of the central values.

% ------------------------------------------------------------------------------
\subsubsection{Second fitter implementation}
\label{sec:measurement_of_sin2beta:systematics:cross_checks:second_fitter}
%
To provide a cross-check of the fitter implementation, two different,
independent fitter implementations were developed by the author and another
member of the working group. Being built up on the \RooFit library they do not
share any common code base. For simplification they will be denoted as
\emph{Fitter A} and \emph{Fitter B}. Both fitters are tested against each other
in the nominal fitter setup (\cref{sec:measurement_of_sin2beta:likelihood_fit}).
Applying the measure described before, the results from both fitters are in
excellent agreement, as is shown in
\Cref{fig:measurement_of_sin2beta:systematics:cross_checks:second_fitter}.
%
\begin{figure}
\centering
%!TEX root = ../main.tex

\definecolor{fcdGrayA}{HTML}{111111}
\definecolor{fcdGrayB}{HTML}{222222}
\definecolor{fcdGrayC}{HTML}{333333}
\definecolor{fcdGrayD}{HTML}{444444}
\definecolor{fcdGrayE}{HTML}{555555}
\definecolor{fcdGrayF}{HTML}{666666}
\definecolor{fcdGrayG}{HTML}{777777}
\definecolor{fcdGrayH}{HTML}{888888}
\definecolor{fcdGrayI}{HTML}{999999}
\definecolor{fcdGrayJ}{HTML}{AAAAAA}
\definecolor{fcdGrayK}{HTML}{BBBBBB}
\definecolor{fcdGrayL}{HTML}{CCCCCC}
\definecolor{fcdGrayM}{HTML}{DDDDDD}
\definecolor{fcdGrayN}{HTML}{EEEEEE}

\colorlet{ClrFitterA}{blue}
\colorlet{ClrFitterB}{red}

\begin{tikzpicture}[
  exp_label/.style={
    anchor=west,
    %minimum width=10em,
    align=left,
    font=\small\sffamily,
    inner sep=0.5em,
    outer sep=0,
  },
  exp_result/.style={
    anchor=east,
    align=right,
    font=\footnotesize\sffamily,
    inner sep=0.5em,
    outer sep=0,
    yshift=0.22em
  }
]
\begin{axis}[
  width=\textwidth,
  height=22ex,
  font=\small,
  xmin=0.62,xmax=0.80,ymin=0.3,ymax=2.7,
  xlabel={${\SJpsiKS}$},
  xlabel style={
        at={(ticklabel cs:1)},
        anchor=north east,
    },%
  xtick={0.67, 0.69, 0.71, 0.73, 0.75, 0.77},
  xticklabel style={%
    major tick length=3pt,
    /pgf/number format/fixed
  },
  hide y axis
]

\begin{pgfonlayer}{background}
  \draw[ultra thin,color=fcdGrayM,dashed]({rel axis cs:0,0}-|{axis cs:0.67,0}) -- ({rel axis cs:0.67,1}-|{axis cs:0.67,0});
  \draw[ultra thin,color=fcdGrayM,dashed]({rel axis cs:0,0}-|{axis cs:0.68,0}) -- ({rel axis cs:0.68,1}-|{axis cs:0.68,0});
  \draw[ultra thin,color=fcdGrayM,dashed]({rel axis cs:0,0}-|{axis cs:0.69,0}) -- ({rel axis cs:0.69,1}-|{axis cs:0.69,0});
  \draw[ultra thin,color=fcdGrayM,dashed]({rel axis cs:0,0}-|{axis cs:0.70,0}) -- ({rel axis cs:0.70,1}-|{axis cs:0.70,0});
  \draw[ultra thin,color=fcdGrayM,dashed]({rel axis cs:0,0}-|{axis cs:0.71,0}) -- ({rel axis cs:0.71,1}-|{axis cs:0.71,0});
  \draw[ultra thin,color=fcdGrayM,dashed]({rel axis cs:0,0}-|{axis cs:0.72,0}) -- ({rel axis cs:0.72,1}-|{axis cs:0.72,0});
  \draw[ultra thin,color=fcdGrayM,dashed]({rel axis cs:0,0}-|{axis cs:0.73,0}) -- ({rel axis cs:0.73,1}-|{axis cs:0.73,0});
  \draw[ultra thin,color=fcdGrayM,dashed]({rel axis cs:0,0}-|{axis cs:0.74,0}) -- ({rel axis cs:0.74,1}-|{axis cs:0.74,0});
  \draw[ultra thin,color=fcdGrayM,dashed]({rel axis cs:0,0}-|{axis cs:0.75,0}) -- ({rel axis cs:0.75,1}-|{axis cs:0.75,0});
  \draw[ultra thin,color=fcdGrayM,dashed]({rel axis cs:0,0}-|{axis cs:0.76,0}) -- ({rel axis cs:0.76,1}-|{axis cs:0.76,0});
  \draw[ultra thin,color=fcdGrayM,dashed]({rel axis cs:0,0}-|{axis cs:0.77,0}) -- ({rel axis cs:0.77,1}-|{axis cs:0.77,0});

  % per-event reso
  \fill[color=fcdGrayM] ({rel axis cs:0,0}-|{axis cs:0.737125640334,0}) rectangle ({rel axis cs:0.718208386766,1}-|{axis cs:0.718208386766,0});
  \fill[color=fcdGrayK] ({rel axis cs:0,0}-|{axis cs:0.732396326942,0}) rectangle ({rel axis cs:0.722937700158,1}-|{axis cs:0.722937700158,0});
  \draw[ultra thin,color=fcdGrayI]({rel axis cs:0,0}-|{axis cs:0.72766701355,0}) -- ({rel axis cs:0.72766701355,1}-|{axis cs:0.72766701355,0});
\end{pgfonlayer}

%-------------------------------------------------------------------------------
% PER_EVENT RESOLUTION
\node[exp_label, color=ClrFitterA] (FitterA_lbl) at (axis cs: \pgfkeysvalueof{/pgfplots/xmin},2)  
{Fitter A};
\node[exp_result,color=ClrFitterA] (FitterA_rsl) at (axis cs: \pgfkeysvalueof{/pgfplots/xmax},2) 
{${0.727 \pm\,^{0.034}_{0.035}}$};

\addplot+[only marks,
    thin,
    solid,
    color = ClrFitterA,
    mark=none,
    mark options={%
      scale=0.7,
      draw=ClrFitterA
    },
    error bars/.cd,
    x dir=both, x explicit,
    y dir=both, y explicit,
    error mark options={%
      rotate=90,
      mark size=5pt,
      color=ClrFitterA
    }
]
table[
        x error plus=ex+,
        x error minus=ex-,
]{
  x        y  ex+        ex-  
  0.72683  2  0.034456   0.034641
};

\node[exp_label, color=ClrFitterB] (FitterB_lbl) at (axis cs: \pgfkeysvalueof{/pgfplots/xmin},1)  
{Fitter B};
\node[exp_result,color=ClrFitterB] (FitterB_rsl) at (axis cs: \pgfkeysvalueof{/pgfplots/xmax},1) 
{${0.729 \pm 0.035}$};

\addplot+[only marks,
    thin,
    solid,
    color = ClrFitterB,
    mark=none,
    mark options={%
      scale=0.7,
      draw=ClrFitterB
    },
    error bars/.cd,
    x dir=both, x explicit,
    y dir=both, y explicit,
    error mark options={%
      rotate=90,
      mark size=5pt,
      color=ClrFitterB
    }
]
table[
        x error plus=ex+,
        x error minus=ex-,
]{
  x              y  ex+             ex-  
  0.7285040271   1  0.03461272157   0.03477905032
};

\end{axis}
\end{tikzpicture}

%!TEX root = ../main.tex

\definecolor{fcdGrayA}{HTML}{111111}
\definecolor{fcdGrayB}{HTML}{222222}
\definecolor{fcdGrayC}{HTML}{333333}
\definecolor{fcdGrayD}{HTML}{444444}
\definecolor{fcdGrayE}{HTML}{555555}
\definecolor{fcdGrayF}{HTML}{666666}
\definecolor{fcdGrayG}{HTML}{777777}
\definecolor{fcdGrayH}{HTML}{888888}
\definecolor{fcdGrayI}{HTML}{999999}
\definecolor{fcdGrayJ}{HTML}{AAAAAA}
\definecolor{fcdGrayK}{HTML}{BBBBBB}
\definecolor{fcdGrayL}{HTML}{CCCCCC}
\definecolor{fcdGrayM}{HTML}{DDDDDD}
\definecolor{fcdGrayN}{HTML}{EEEEEE}

\colorlet{ClrFitterA}{blue}
\colorlet{ClrFitterB}{red}

\begin{tikzpicture}[
  exp_label/.style={
    anchor=west,
    %minimum width=10em,
    align=left,
    font=\small\sffamily,
    inner sep=0.5em,
    outer sep=0,
  },
  exp_result/.style={
    anchor=east,
    align=right,
    font=\footnotesize\sffamily,
    inner sep=0.5em,
    outer sep=0,
    yshift=0.22em
  }
]
\begin{axis}[
  width=\textwidth,
  height=22ex,
  font=\small,
  xmin=-0.137,xmax=0.043,ymin=0.3,ymax=2.7,
  xlabel={${\CJpsiKS}$},
  xlabel style={
        at={(ticklabel cs:1)},
        anchor=north east,
    },%
  xtick={-0.09, -0.07, -0.05, -0.03, -0.01, 0.01},
  xticklabel style={%
    major tick length=3pt,
    /pgf/number format/fixed
  },
  hide y axis
]

\begin{pgfonlayer}{background}
  \draw[ultra thin,color=fcdGrayM,dashed]({rel axis cs:0,0}-|{axis cs:-0.09,0}) -- ({rel axis cs:-0.09,1}-|{axis cs:-0.09,0});
  \draw[ultra thin,color=fcdGrayM,dashed]({rel axis cs:0,0}-|{axis cs:-0.08,0}) -- ({rel axis cs:-0.08,1}-|{axis cs:-0.08,0});
  \draw[ultra thin,color=fcdGrayM,dashed]({rel axis cs:0,0}-|{axis cs:-0.07,0}) -- ({rel axis cs:-0.07,1}-|{axis cs:-0.07,0});
  \draw[ultra thin,color=fcdGrayM,dashed]({rel axis cs:0,0}-|{axis cs:-0.06,0}) -- ({rel axis cs:-0.06,1}-|{axis cs:-0.06,0});
  \draw[ultra thin,color=fcdGrayM,dashed]({rel axis cs:0,0}-|{axis cs:-0.05,0}) -- ({rel axis cs:-0.05,1}-|{axis cs:-0.05,0});
  \draw[ultra thin,color=fcdGrayM,dashed]({rel axis cs:0,0}-|{axis cs:-0.04,0}) -- ({rel axis cs:-0.04,1}-|{axis cs:-0.04,0});
  \draw[ultra thin,color=fcdGrayM,dashed]({rel axis cs:0,0}-|{axis cs:-0.03,0}) -- ({rel axis cs:-0.03,1}-|{axis cs:-0.03,0});
  \draw[ultra thin,color=fcdGrayM,dashed]({rel axis cs:0,0}-|{axis cs:-0.02,0}) -- ({rel axis cs:-0.02,1}-|{axis cs:-0.02,0});
  \draw[ultra thin,color=fcdGrayM,dashed]({rel axis cs:0,0}-|{axis cs:-0.01,0}) -- ({rel axis cs:-0.01,1}-|{axis cs:-0.01,0});
  \draw[ultra thin,color=fcdGrayM,dashed]({rel axis cs:0,0}-|{axis cs:-0.00,0}) -- ({rel axis cs:-0.00,1}-|{axis cs:-0.00,0});
  \draw[ultra thin,color=fcdGrayM,dashed]({rel axis cs:0,0}-|{axis cs:0.01,0}) -- ({rel axis cs:0.01,1}-|{axis cs:0.01,0});
  
  % per-event reso
  \fill[color=fcdGrayM] ({rel axis cs:0,0}-|{axis cs:-0.0255884399411,0}) rectangle ({rel axis cs:-0.0396240151589,1}-|{axis cs:-0.0396240151589,0});
  \fill[color=fcdGrayK] ({rel axis cs:0,0}-|{axis cs:-0.0290973337456,0}) rectangle ({rel axis cs:-0.0361151213544,1}-|{axis cs:-0.0361151213544,0});
  \draw[ultra thin,color=fcdGrayI]({rel axis cs:0,0}-|{axis cs:-0.03260622755,0}) -- ({rel axis cs:-0.03260622755,1}-|{axis cs:-0.03260622755,0});
\end{pgfonlayer}


%-------------------------------------------------------------------------------
% PER_EVENT RESOLUTION
\node[exp_label, color=ClrFitterA] (FitterA_lbl) at (axis cs: \pgfkeysvalueof{/pgfplots/xmin},2)  
{Fitter A};
\node[exp_result,color=ClrFitterA] (FitterA_rsl) at (axis cs: \pgfkeysvalueof{/pgfplots/xmax},2) 
{${-0.032 \pm 0.032}$};

\addplot+[only marks,
    thin,
    solid,
    color = ClrFitterA,
    mark=none,
    mark options={%
      scale=0.7,
      draw=ClrFitterA
    },
    error bars/.cd,
    x dir=both, x explicit,
    y dir=both, y explicit,
    error mark options={%
      rotate=90,
      mark size=5pt,
      color=ClrFitterA
    }
]
table[
        x error plus=ex+,
        x error minus=ex-,
]{
  x          y  ex+         ex-  
  -0.032078  2  0.032122   0.032312
};

\node[exp_label, color=ClrFitterB] (FitterB_lbl) at (axis cs: \pgfkeysvalueof{/pgfplots/xmin},1)  
{Fitter B};
\node[exp_result,color=ClrFitterB] (FitterB_rsl) at (axis cs: \pgfkeysvalueof{/pgfplots/xmax},1) 
{${-0.033 \pm 0.032}$};

\addplot+[only marks,
    thin,
    solid,
    color = ClrFitterB,
    mark=none,
    mark options={%
      scale=0.7,
      draw=ClrFitterB
    },
    error bars/.cd,
    x dir=both, x explicit,
    y dir=both, y explicit,
    error mark options={%
      rotate=90,
      mark size=5pt,
      color=ClrFitterB
    }
]
table[
        x error plus=ex+,
        x error minus=ex-,
]{
  x             y  ex+             ex-  
  -0.0331344551 1  0.03212091232   0.03213103531
};

\end{axis}
\end{tikzpicture}

\caption{
Fit results for \SJpsiKS and \CJpsiKS for both fitter implementations.
In blue (red) the result and the statistical error for Fitter A (B). The
grey solid line shows the average of both fit results, while the filled grey
area describes the uncertainty on the difference $\sigma_\Delta$ of the two
results (darker grey $1\sigma_\Delta$, light grey $2\sigma_\Delta$).
\textit{Please note: The shown numbers are rounded following the PDG rules. The
plot itself is produced using the full precision values.} }
\label{fig:measurement_of_sin2beta:systematics:cross_checks:second_fitter}
\end{figure}

% ------------------------------------------------------------------------------
\subsubsection{\sPlot fit}
\label{sec:measurement_of_sin2beta:systematics:cross_checks:splot_fit}
%
As the background decay time model might influence the measurement, an \sPlot fit is
performed using the nominal signal mass \PDF. As the standard \Minuit/\Hesse error
estimations in \sPlot fits can lead to uncertainties with an incorrect coverage,
an additional bootstrapping (\cf \eg \cite{Behnke:2013pga}) study is performed
to arrive at more correct estimates of uncertainties. The procedure followed is:
%
\begin{enumerate}
  \item take the \sweighted data sample consisting of $N$ events,
  \item sample $N$ randomly selected events with replacement into a new data sample,
  \item perform a fit to the resulting sample to measure \SJpsiKS and \CJpsiKS,
  \item repeat the steps to obtain a distribution of the measured \CP parameters.
\end{enumerate}
%
\Cref{fig:measurement_of_sin2beta:systematics:cross_checks:splot_fit:s_and_c}
shows the fit result of the \sPlot fit with the corresponding \Hesse uncertainty
estimates, as well as the nominal fit result. Additionally, the uncertainty
estimates from the bootstrapping study are included. For the latter, the
medians of the measured \SJpsiKS and \CJpsiKS distributions are chosen as
central values, while the quoted uncertainties are the quantile-based estimates
for two-sided $\SI{68.27}{\percent}$ quantiles of the same distribution. The
resulting distributions based on $\num{1000}$ bootstrapping iterations are
presented in
\cref{fig:measurement_of_sin2beta:systematics:cross_checks:splot_fit:bootstrapping}.

The central values resulting from the bootstrapped \sPlot fits are very well
compatible with the naive \sPlot fit. Compared to the nominal fit this holds
true for \CJpsiKS while \SJpsiKS is slightly lower. The uncertainty estimates of
the bootstrapping study are slightly larger than the corresponding uncertainties
from the nominal fit, but still in the same order of magnitude. In contrast, the
uncertainty estimate of the naive \sPlot fit for \SJpsiKS is too small. Overall,
the results of the \sPlot fits and the nominal fit are well compatible.
%
\begin{figure}
\centering
%!TEX root = ../main.tex

\definecolor{fcdGrayA}{HTML}{111111}
\definecolor{fcdGrayB}{HTML}{222222}
\definecolor{fcdGrayC}{HTML}{333333}
\definecolor{fcdGrayD}{HTML}{444444}
\definecolor{fcdGrayE}{HTML}{555555}
\definecolor{fcdGrayF}{HTML}{666666}
\definecolor{fcdGrayG}{HTML}{777777}
\definecolor{fcdGrayH}{HTML}{888888}
\definecolor{fcdGrayI}{HTML}{999999}
\definecolor{fcdGrayJ}{HTML}{AAAAAA}
\definecolor{fcdGrayK}{HTML}{BBBBBB}
\definecolor{fcdGrayL}{HTML}{CCCCCC}
\definecolor{fcdGrayM}{HTML}{DDDDDD}
\definecolor{fcdGrayN}{HTML}{EEEEEE}

\colorlet{ClrSFit}{blue}
\colorlet{ClrSFitBT}{blue}
\colorlet{ClrNominalFit}{red}

\begin{tikzpicture}[
  exp_label/.style={
    anchor=west,
    %minimum width=10em,
    align=left,
    font=\small\sffamily,
    inner sep=0.5em,
    outer sep=0,
  },
  exp_result/.style={
    anchor=east,
    align=right,
    font=\footnotesize\sffamily,
    inner sep=0.5em,
    outer sep=0,
    yshift=0.22em
  }
]
\begin{axis}[
  width=\textwidth,
  height=30ex,
  font=\small,
  xmin=0.62,xmax=0.80,ymin=0.3,ymax=3.7,
  xlabel={${\SJpsiKS}$},
  xlabel style={
        at={(ticklabel cs:1)},
        anchor=north east,
    },%
  xtick={0.67, 0.69, 0.71, 0.73, 0.75, 0.77},
  xticklabel style={%
    major tick length=3pt,
    /pgf/number format/fixed
  },
  hide y axis
]

\begin{pgfonlayer}{background}
  \draw[ultra thin,color=fcdGrayM,dashed]({rel axis cs:0,0}-|{axis cs:0.67,0}) -- ({rel axis cs:0.67,1}-|{axis cs:0.67,0});
  \draw[ultra thin,color=fcdGrayM,dashed]({rel axis cs:0,0}-|{axis cs:0.68,0}) -- ({rel axis cs:0.68,1}-|{axis cs:0.68,0});
  \draw[ultra thin,color=fcdGrayM,dashed]({rel axis cs:0,0}-|{axis cs:0.69,0}) -- ({rel axis cs:0.69,1}-|{axis cs:0.69,0});
  \draw[ultra thin,color=fcdGrayM,dashed]({rel axis cs:0,0}-|{axis cs:0.70,0}) -- ({rel axis cs:0.70,1}-|{axis cs:0.70,0});
  \draw[ultra thin,color=fcdGrayM,dashed]({rel axis cs:0,0}-|{axis cs:0.71,0}) -- ({rel axis cs:0.71,1}-|{axis cs:0.71,0});
  \draw[ultra thin,color=fcdGrayM,dashed]({rel axis cs:0,0}-|{axis cs:0.72,0}) -- ({rel axis cs:0.72,1}-|{axis cs:0.72,0});
  \draw[ultra thin,color=fcdGrayM,dashed]({rel axis cs:0,0}-|{axis cs:0.73,0}) -- ({rel axis cs:0.73,1}-|{axis cs:0.73,0});
  \draw[ultra thin,color=fcdGrayM,dashed]({rel axis cs:0,0}-|{axis cs:0.74,0}) -- ({rel axis cs:0.74,1}-|{axis cs:0.74,0});
  \draw[ultra thin,color=fcdGrayM,dashed]({rel axis cs:0,0}-|{axis cs:0.75,0}) -- ({rel axis cs:0.75,1}-|{axis cs:0.75,0});
  \draw[ultra thin,color=fcdGrayM,dashed]({rel axis cs:0,0}-|{axis cs:0.76,0}) -- ({rel axis cs:0.76,1}-|{axis cs:0.76,0});
  \draw[ultra thin,color=fcdGrayM,dashed]({rel axis cs:0,0}-|{axis cs:0.77,0}) -- ({rel axis cs:0.77,1}-|{axis cs:0.77,0});

  % 2 sigma error band
  \fill[color=fcdGrayM] ({rel axis cs:0,0}-|{axis cs:0.72510503111,0}) rectangle ({rel axis cs:0.71465031339,1}-|{axis cs:0.71465031339,0});
  % 1 sigma error band
  \fill[color=fcdGrayK] ({rel axis cs:0,0}-|{axis cs:0.72249135168,0}) rectangle ({rel axis cs:0.71726399282,1}-|{axis cs:0.71726399282,0});
  % average
  \draw[ultra thin,color=fcdGrayI]({rel axis cs:0,0}-|{axis cs:0.71987767225,0}) -- ({rel axis cs:0.71987767225,1}-|{axis cs:0.71987767225,0});
\end{pgfonlayer}

\node[exp_label, color=ClrSFit] (SFit_lbl) at (axis cs: \pgfkeysvalueof{/pgfplots/xmin},3)  
{\sPlot fit\\[-0.3ex]\footnotesize \Hesse};
\node[exp_result,color=ClrSFit] (SFit_rsl) at (axis cs: \pgfkeysvalueof{/pgfplots/xmax},3) 
{${0.710 \pm 0.027}$};

\addplot+[only marks,
    thin,
    solid,
    color = ClrSFit,
    mark=none,
    mark options={%
      scale=0.7,
      draw=ClrSFit
    },
    error bars/.cd,
    x dir=both, x explicit,
    y dir=both, y explicit,
    error mark options={%
      rotate=90,
      mark size=5pt,
      color=ClrSFit
    }
]
table[
        x error plus=ex+,
        x error minus=ex-,
]{
  x            y  ex+            ex-  
  0.7103944525 3  0.02727126003  0.02727126003
};

\node[exp_label, color=ClrSFitBT] (SFit_lbl) at (axis cs: \pgfkeysvalueof{/pgfplots/xmin},2)  
{\sPlot\\[-0.3ex]\footnotesize bootstrapped fits};
\node[exp_result,color=ClrSFitBT] (SFit_rsl) at (axis cs: \pgfkeysvalueof{/pgfplots/xmax},2) 
{${0.711 \pm\,^{0.035}_{0.036}}$};

\addplot+[only marks,
    thin,
    solid,
    color = ClrSFitBT,
    mark=none,
    mark options={%
      scale=0.7,
      draw=ClrSFitBT
    },
    error bars/.cd,
    x dir=both, x explicit,
    y dir=both, y explicit,
    error mark options={%
      rotate=90,
      mark size=5pt,
      color=ClrSFitBT
    }
]
table[
        x error plus=ex+,
        x error minus=ex-,
]{
  x            y  ex+           ex-  
  0.7112513174 2  0.03487712232 0.03601769196
};

\node[exp_label, color=ClrNominalFit] (NominalFit_lbl) at (axis cs: \pgfkeysvalueof{/pgfplots/xmin},1)  
{Nominal fit};
\node[exp_result,color=ClrNominalFit] (NominalFit_rsl) at (axis cs: \pgfkeysvalueof{/pgfplots/xmax},1) 
{${0.729 \pm 0.035}$};

\addplot+[only marks,
    thin,
    solid,
    color = ClrNominalFit,
    mark=none,
    mark options={%
      scale=0.7,
      draw=ClrNominalFit
    },
    error bars/.cd,
    x dir=both, x explicit,
    y dir=both, y explicit,
    error mark options={%
      rotate=90,
      mark size=5pt,
      color=ClrNominalFit
    }
]
table[
        x error plus=ex+,
        x error minus=ex-,
]{
  x            y  ex+             ex-  
  0.7285040271 1 0.03461272157    0.03477905032
};

\end{axis}
\end{tikzpicture}

%!TEX root = ../main.tex

\definecolor{fcdGrayA}{HTML}{111111}
\definecolor{fcdGrayB}{HTML}{222222}
\definecolor{fcdGrayC}{HTML}{333333}
\definecolor{fcdGrayD}{HTML}{444444}
\definecolor{fcdGrayE}{HTML}{555555}
\definecolor{fcdGrayF}{HTML}{666666}
\definecolor{fcdGrayG}{HTML}{777777}
\definecolor{fcdGrayH}{HTML}{888888}
\definecolor{fcdGrayI}{HTML}{999999}
\definecolor{fcdGrayJ}{HTML}{AAAAAA}
\definecolor{fcdGrayK}{HTML}{BBBBBB}
\definecolor{fcdGrayL}{HTML}{CCCCCC}
\definecolor{fcdGrayM}{HTML}{DDDDDD}
\definecolor{fcdGrayN}{HTML}{EEEEEE}

\colorlet{ClrSFit}{blue}
\colorlet{ClrSFitBT}{blue}
\colorlet{ClrNominalFit}{red}

\begin{tikzpicture}[
  exp_label/.style={
    anchor=west,
    %minimum width=10em,
    align=left,
    font=\small\sffamily,
    inner sep=0.5em,
    outer sep=0,
  },
  exp_result/.style={
    anchor=east,
    align=right,
    font=\footnotesize\sffamily,
    inner sep=0.5em,
    outer sep=0,
    yshift=0.22em
  }
]
\begin{axis}[
  width=\textwidth,
  height=30ex,
  font=\small,
  xmin=-0.117,xmax=0.063,ymin=0.3,ymax=3.7,
  xlabel={${\CJpsiKS}$},
  xlabel style={
        at={(ticklabel cs:1)},
        anchor=north east,
    },%
  xtick={-0.07, -0.05, -0.03, -0.01, 0.01, 0.03},
  xticklabel style={%
    major tick length=3pt,
    /pgf/number format/fixed
  },
  hide y axis
]

\begin{pgfonlayer}{background}
  \draw[ultra thin,color=fcdGrayM,dashed]({rel axis cs:0,0}-|{axis cs:-0.07,0}) -- ({rel axis cs:-0.07,1}-|{axis cs:-0.07,0});
  \draw[ultra thin,color=fcdGrayM,dashed]({rel axis cs:0,0}-|{axis cs:-0.06,0}) -- ({rel axis cs:-0.06,1}-|{axis cs:-0.06,0});
  \draw[ultra thin,color=fcdGrayM,dashed]({rel axis cs:0,0}-|{axis cs:-0.05,0}) -- ({rel axis cs:-0.05,1}-|{axis cs:-0.05,0});
  \draw[ultra thin,color=fcdGrayM,dashed]({rel axis cs:0,0}-|{axis cs:-0.04,0}) -- ({rel axis cs:-0.04,1}-|{axis cs:-0.04,0});
  \draw[ultra thin,color=fcdGrayM,dashed]({rel axis cs:0,0}-|{axis cs:-0.03,0}) -- ({rel axis cs:-0.03,1}-|{axis cs:-0.03,0});
  \draw[ultra thin,color=fcdGrayM,dashed]({rel axis cs:0,0}-|{axis cs:-0.02,0}) -- ({rel axis cs:-0.02,1}-|{axis cs:-0.02,0});
  \draw[ultra thin,color=fcdGrayM,dashed]({rel axis cs:0,0}-|{axis cs:-0.01,0}) -- ({rel axis cs:-0.01,1}-|{axis cs:-0.01,0});
  \draw[ultra thin,color=fcdGrayM,dashed]({rel axis cs:0,0}-|{axis cs:-0.00,0}) -- ({rel axis cs:-0.00,1}-|{axis cs:-0.00,0});
  \draw[ultra thin,color=fcdGrayM,dashed]({rel axis cs:0,0}-|{axis cs: 0.01,0}) -- ({rel axis cs: 0.01,1}-|{axis cs: 0.01,0});
  \draw[ultra thin,color=fcdGrayM,dashed]({rel axis cs:0,0}-|{axis cs: 0.02,0}) -- ({rel axis cs: 0.02,1}-|{axis cs: 0.02,0});
  \draw[ultra thin,color=fcdGrayM,dashed]({rel axis cs:0,0}-|{axis cs: 0.03,0}) -- ({rel axis cs: 0.03,1}-|{axis cs: 0.03,0});

  % 2 sigma error band
  \fill[color=fcdGrayM] ({rel axis cs:0,0}-|{axis cs:-0.00103402373304,0}) rectangle ({rel axis cs:-0.042529957587,1}-|{axis cs:-0.042529957587,0});
  % 1 sigma error band
  \fill[color=fcdGrayK] ({rel axis cs:0,0}-|{axis cs:-0.0114080071965,0}) rectangle ({rel axis cs:-0.0321559741235,1}-|{axis cs:-0.0321559741235,0});
  % average
  \draw[ultra thin,color=fcdGrayI]({rel axis cs:0,0}-|{axis cs:-0.02178199066,0}) -- ({rel axis cs:-0.02178199066,1}-|{axis cs:-0.02178199066,0});
\end{pgfonlayer}

\node[exp_label, color=ClrSFit] (SFit_lbl) at (axis cs: \pgfkeysvalueof{/pgfplots/xmin},3)  
{\sPlot fit\\[-0.3ex]\footnotesize \Hesse};
\node[exp_result,color=ClrSFit] (SFit_rsl) at (axis cs: \pgfkeysvalueof{/pgfplots/xmax},3) 
{${-0.011 \pm 0.037}$};

\addplot+[only marks,
    thin,
    solid,
    color = ClrSFit,
    mark=none,
    mark options={%
      scale=0.7,
      draw=ClrSFit
    },
    error bars/.cd,
    x dir=both, x explicit,
    y dir=both, y explicit,
    error mark options={%
      rotate=90,
      mark size=5pt,
      color=ClrSFit
    }
]
table[
        x error plus=ex+,
        x error minus=ex-,
]{
  x              y  ex+             ex-  
  -0.01110790192 3  0.03708065836   0.03708065836
};

\node[exp_label, color=ClrSFitBT] (SFit_lbl) at (axis cs: \pgfkeysvalueof{/pgfplots/xmin},2)  
{\sPlot\\[-0.3ex]\footnotesize bootstrapped fits};
\node[exp_result,color=ClrSFitBT] (SFit_rsl) at (axis cs: \pgfkeysvalueof{/pgfplots/xmax},2) 
{${-0.010 \pm\,^{0.038}_{0.034}}$};

\addplot+[only marks,
    thin,
    solid,
    color = ClrSFitBT,
    mark=none,
    mark options={%
      scale=0.7,
      draw=ClrSFitBT
    },
    error bars/.cd,
    x dir=both, x explicit,
    y dir=both, y explicit,
    error mark options={%
      rotate=90,
      mark size=5pt,
      color=ClrSFitBT
    }
]
table[
        x error plus=ex+,
        x error minus=ex-,
]{
  x              y  ex+             ex-  
  -0.01042952622 2  0.03750748602   0.03375459289
};

\node[exp_label, color=ClrNominalFit] (NominalFit_lbl) at (axis cs: \pgfkeysvalueof{/pgfplots/xmin},1)  
{Nominal fit};
\node[exp_result,color=ClrNominalFit] (NominalFit_rsl) at (axis cs: \pgfkeysvalueof{/pgfplots/xmax},1) 
{${-0.033 \pm 0.032}$};

\addplot+[only marks,
    thin,
    solid,
    color = ClrNominalFit,
    mark=none,
    mark options={%
      scale=0.7,
      draw=ClrNominalFit
    },
    error bars/.cd,
    x dir=both, x explicit,
    y dir=both, y explicit,
    error mark options={%
      rotate=90,
      mark size=5pt,
      color=ClrNominalFit
    }
]
table[
        x error plus=ex+,
        x error minus=ex-,
]{
  x             y  ex+             ex-  
  -0.0331344551 1  0.03212091232   0.03213103531
};

\end{axis}
\end{tikzpicture}

\caption{
Fit results for \SJpsiKS and \CJpsiKS from the \sPlot fit and the nominal fit as
a reference. In blue (red) the result and the statistical error for the \sPlot
fit (nominal fit) using a per-event decay time resolution. The grey solid line
shows the average of the two below fit results, while the filled grey area
describes the uncertainty on the difference $\sigma_\Delta$ of the two lower
results (darker grey $1\sigma_\Delta$, light grey $2\sigma_\Delta$).
\textit{Please note: The shown numbers are rounded following the PDG rules. The
plot itself is produced using the full precision values.}}
\label{fig:measurement_of_sin2beta:systematics:cross_checks:splot_fit:s_and_c}
\end{figure}
%
\begin{figure}
\centering
\includegraphics[width=0.48\textwidth]{private/content/measurement-of-sin2beta/figs/bootstrapping_parSigTimeSin2b_blind.pdf}
\hfill
\includegraphics[width=0.48\textwidth]{private/content/measurement-of-sin2beta/figs/bootstrapping_parSigTimeCjpsiKS_blind.pdf}
\caption{
Parameter distributions for \SJpsiKS and \CJpsiKS from \sPlot fits to
$\num{1000}$ bootstrapped data samples.}
\label{fig:measurement_of_sin2beta:systematics:cross_checks:splot_fit:bootstrapping}
\end{figure}

% ------------------------------------------------------------------------------
\subsubsection{Subsamples}
\label{sec:measurement_of_sin2beta:systematics:cross_checks:subsamples}
%
To check for possible systematic effects, fits in different subsamples of the
nominal data set are conducted. The cross-checks are performed in categories of
track type (\catDD \vs \catLL), trigger requirement (\catAU \vs \catEB), tagging
algorithm (\catOS \vs \catSS \vs \catBS), magnet polarity (Up \vs Down), and
admixtures of those together with the year of data-taking (\catOO \vs \catOT).
As a control sample the full data set is also split randomly into five disjoint
subsets (Rndm1-5). \Cref{fig:measurement_of_sin2beta:systematics:cross_checks:subsamples:s_and_c}
illustrates the outcome. No significant deviation is present.
%
\begin{figure}
\centering
%!TEX root = ../main.tex

\definecolor{Black}{HTML}{000000}
\definecolor{Red}{HTML}{FC5716}

\colorlet{ClrRndm}{Black}
\colorlet{ClrTrigger}{Black}
\colorlet{ClrTagger}{Black}
\colorlet{ClrYear}{Black}
\colorlet{ClrYearTagger}{Black}
\colorlet{ClrTrack}{Black}
\colorlet{ClrFullFit}{Red}
\colorlet{ClrUncert}{Red!40}

\begin{tikzpicture}[
  exp_label/.style={
    anchor=west,
    %minimum width=10em,
    align=left,
    font=\small\sffamily,
    inner sep=0.5em,
    outer sep=0,
  },
  exp_result/.style={
    anchor=east,
    align=right,
    font=\footnotesize\sffamily,
    inner sep=0.5em,
    outer sep=0,
    yshift=0.22em
  }
]
\begin{axis}[
  width=\textwidth,
  height=65ex,
  font=\small,
  xmin=0.1,xmax=1.1,ymin=0.3,ymax=23.7,
  xlabel={$\SJpsiKS$},
  xlabel style={
        at={(ticklabel cs:1)},
        anchor=north east,
    },%
  xtick={0.3, 0.4, 0.5, 0.6, 0.7, 0.8, 0.9, 1.0},
  % xticklabels={,,},
  xticklabel style={%
    major tick length=3pt
  },
  hide y axis
]

\begin{pgfonlayer}{background}
% line at 0
% \draw[ultra thin,color=Black!80]({rel axis cs:0,0}-|{axis cs:0,0}) -- ({rel axis cs:0,1}-|{axis cs:0,0});
% line at -1
% \draw[ultra thin,color=Black!80,dashed]({rel axis cs:0,0}-|{axis cs:-1,0}) -- ({rel axis cs:-1,1}-|{axis cs:-1,0});
% line at +1
% \draw[ultra thin,color=Black!80,dashed]({rel axis cs:0,0}-|{axis cs:+1,0}) -- ({rel axis cs:+1,1}-|{axis cs:+1,0});
% filled area at average value
\fill[color=ClrUncert] ({rel axis cs:0,0}-|{axis cs:0.693724977,0}) rectangle ({rel axis cs:0.763116749,1}-|{axis cs:0.763116749,0});
% line at average value
\draw[ultra thin,color=ClrFullFit]({rel axis cs:0,0}-|{axis cs:0.7285040271,0}) -- ({rel axis cs:0.7285040271,1}-|{axis cs:0.7285040271,0});

% lines to separate the different subsamples categories
\draw[ultra thin, dashed, color=Black!80] (axis cs: \pgfkeysvalueof{/pgfplots/xmin},18.5) -- (axis cs: \pgfkeysvalueof{/pgfplots/xmax},18.5);
\draw[ultra thin, dashed, color=Black!80] (axis cs: \pgfkeysvalueof{/pgfplots/xmin},16.5) -- (axis cs: \pgfkeysvalueof{/pgfplots/xmax},16.5);
\draw[ultra thin, dashed, color=Black!80] (axis cs: \pgfkeysvalueof{/pgfplots/xmin},13.5) -- (axis cs: \pgfkeysvalueof{/pgfplots/xmax},13.5);
\draw[ultra thin, dashed, color=Black!80] (axis cs: \pgfkeysvalueof{/pgfplots/xmin},11.5) -- (axis cs: \pgfkeysvalueof{/pgfplots/xmax},11.5);
\draw[ultra thin, dashed, color=Black!80] (axis cs: \pgfkeysvalueof{/pgfplots/xmin}, 9.5) -- (axis cs: \pgfkeysvalueof{/pgfplots/xmax}, 9.5);
\draw[ultra thin, dashed, color=Black!80] (axis cs: \pgfkeysvalueof{/pgfplots/xmin}, 7.5) -- (axis cs: \pgfkeysvalueof{/pgfplots/xmax}, 7.5);
\draw[ultra thin, dashed, color=Black!80] (axis cs: \pgfkeysvalueof{/pgfplots/xmin}, 1.5) -- (axis cs: \pgfkeysvalueof{/pgfplots/xmax}, 1.5);
\end{pgfonlayer}

%-------------------------------------------------------------------------------
% Random
\node[exp_label, color=ClrRndm] (Rndm1_lbl) at (axis cs: \pgfkeysvalueof{/pgfplots/xmin},23){Rndm1};
\node[exp_label, color=ClrRndm] (Rndm2_lbl) at (axis cs: \pgfkeysvalueof{/pgfplots/xmin},22){Rndm2};
\node[exp_label, color=ClrRndm] (Rndm3_lbl) at (axis cs: \pgfkeysvalueof{/pgfplots/xmin},21){Rndm3};
\node[exp_label, color=ClrRndm] (Rndm4_lbl) at (axis cs: \pgfkeysvalueof{/pgfplots/xmin},20){Rndm4};
\node[exp_label, color=ClrRndm] (Rndm5_lbl) at (axis cs: \pgfkeysvalueof{/pgfplots/xmin},19){Rndm5};


% Full uncertainty
\addplot+[only marks,
    thin,
    solid,
    color = ClrRndm,
    mark=*,
    mark options={%
      scale=0.7,
      draw=ClrRndm
    },
    error bars/.cd,
    x dir=both, x explicit,
    y dir=both, y explicit,
    error mark options={%
      rotate=90,
      mark size=3pt,
      color=ClrRndm
    }
]
table[
        x error plus=ex+,
        x error minus=ex-,
]{
  x             y  ex+            ex-  
  0.7712377343  23 0.073430739    0.074585294
  0.7008958291  22 0.07418394334  0.07429740627
  0.6470130176  21 0.0762116547   0.07704696965
  0.8668497014  20 0.078519274    0.07990252538
  0.6722234942  19 0.07662743282  0.0767069747
};

%-------------------------------------------------------------------------------
% Trigger
\node[exp_label, color=ClrTrigger] (AU_lbl) at (axis cs: \pgfkeysvalueof{/pgfplots/xmin},18){Almost unbiased};
\node[exp_label, color=ClrTrigger] (EB_lbl) at (axis cs: \pgfkeysvalueof{/pgfplots/xmin},17){Exclusively biased};


% Full uncertainty
\addplot+[only marks,
    thin,
    solid,
    color = ClrTrigger,
    mark=*,
    mark options={%
      scale=0.7,
      draw=ClrTrigger
    },
    error bars/.cd,
    x dir=both, x explicit,
    y dir=both, y explicit,
    error mark options={%
      rotate=90,
      mark size=3pt,
      color=ClrTrigger
    }
]
table[
        x error plus=ex+,
        x error minus=ex-,
]{
  x            y   ex+            ex-  
  0.710745651  18  0.03795927938  0.03833662133
  0.8250789821 17  0.08032890424  0.08135593581
};

%-------------------------------------------------------------------------------
% Tagger
\node[exp_label, color=ClrTagger] (OS_lbl) at (axis cs: \pgfkeysvalueof{/pgfplots/xmin},16){Excl. OS};
\node[exp_label, color=ClrTagger] (SS_lbl) at (axis cs: \pgfkeysvalueof{/pgfplots/xmin},15){Excl. SS};
\node[exp_label, color=ClrTagger] (OL_lbl) at (axis cs: \pgfkeysvalueof{/pgfplots/xmin},14){Excl. BS};


% Full uncertainty
\addplot+[only marks,
    thin,
    solid,
    color = ClrTagger,
    mark=*,
    mark options={%
      scale=0.7,
      draw=ClrTagger
    },
    error bars/.cd,
    x dir=both, x explicit,
    y dir=both, y explicit,
    error mark options={%
      rotate=90,
      mark size=3pt,
      color=ClrTagger
    }
]
table[
        x error plus=ex+,
        x error minus=ex-,
]{
  x             y   ex+            ex-  
  0.7288118238  16  0.04027423794  0.04031385447
  0.6422124342  15  0.1142810315   0.1140158165
  0.7777873414  14  0.08196002749  0.0831860115
};

%-------------------------------------------------------------------------------
% Year
\node[exp_label, color=ClrYear] (11OS_lbl) at (axis cs: \pgfkeysvalueof{/pgfplots/xmin},13){2011};
\node[exp_label, color=ClrYear] (11SS_lbl) at (axis cs: \pgfkeysvalueof{/pgfplots/xmin},12){2012};

% Full uncertainty
\addplot+[only marks,
    thin,
    solid,
    color = ClrYear,
    mark=*,
    mark options={%
      scale=0.7,
      draw=ClrYear
    },
    error bars/.cd,
    x dir=both, x explicit,
    y dir=both, y explicit,
    error mark options={%
      rotate=90,
      mark size=3pt,
      color=ClrYear
    }
]
table[
        x error plus=ex+,
        x error minus=ex-,
]{
  x            y  ex+            ex-  
  0.6268677992 13 0.06008359319  0.06057614937
  0.7765020167 12 0.04163076047  0.0417892058
};

%-------------------------------------------------------------------------------
% Track type
\node[exp_label, color=ClrTrack] (11DD_lbl) at (axis cs: \pgfkeysvalueof{/pgfplots/xmin},11){downstream};
\node[exp_label, color=ClrTrack] (11LL_lbl) at (axis cs: \pgfkeysvalueof{/pgfplots/xmin},10){long};


% Full uncertainty
\addplot+[only marks,
    thin,
    solid,
    color = ClrTrack,
    mark=*,
    mark options={%
      scale=0.7,
      draw=ClrTrack
    },
    error bars/.cd,
    x dir=both, x explicit,
    y dir=both, y explicit,
    error mark options={%
      rotate=90,
      mark size=3pt,
      color=ClrTrack
    }
]
table[
        x error plus=ex+,
        x error minus=ex-,
]{
  x            y  ex+            ex-  
  0.7509159107 11 0.04218332052  0.0424135402
  0.6846890918 10 0.05867667758  0.05950751898
};

%-------------------------------------------------------------------------------
% Magnet polarity and Year
\node[exp_label, color=ClrYearTagger] (11UP_lbl) at (axis cs: \pgfkeysvalueof{/pgfplots/xmin},9){Up};
\node[exp_label, color=ClrYearTagger] (11DW_lbl) at (axis cs: \pgfkeysvalueof{/pgfplots/xmin},8){Down};


% Full uncertainty
\addplot+[only marks,
    thin,
    solid,
    color = ClrYearTagger,
    mark=*,
    mark options={%
      scale=0.7,
      draw=ClrYearTagger
    },
    error bars/.cd,
    x dir=both, x explicit,
    y dir=both, y explicit,
    error mark options={%
      rotate=90,
      mark size=3pt,
      color=ClrYearTagger
    }
]
table[
        x error plus=ex+,
        x error minus=ex-,
]{
  x            y  ex+            ex-  
  0.763932558  9  0.05054925972  0.05095339912
  0.6973324655 8  0.0463213394   0.04667624179
};

%-------------------------------------------------------------------------------
% Year and Tagger
\node[exp_label, color=ClrYearTagger] (11OS_lbl) at (axis cs: \pgfkeysvalueof{/pgfplots/xmin},7){11 Excl. OS};
\node[exp_label, color=ClrYearTagger] (11SS_lbl) at (axis cs: \pgfkeysvalueof{/pgfplots/xmin},6){11 Excl. SS};
\node[exp_label, color=ClrYearTagger] (11BS_lbl) at (axis cs: \pgfkeysvalueof{/pgfplots/xmin},5){11 Excl. BS};
\node[exp_label, color=ClrYearTagger] (12OS_lbl) at (axis cs: \pgfkeysvalueof{/pgfplots/xmin},4){12 Excl. OS};
\node[exp_label, color=ClrYearTagger] (12SS_lbl) at (axis cs: \pgfkeysvalueof{/pgfplots/xmin},3){12 Excl. SS};
\node[exp_label, color=ClrYearTagger] (12BS_lbl) at (axis cs: \pgfkeysvalueof{/pgfplots/xmin},2){12 Excl. BS};

% Full uncertainty
\addplot+[only marks,
    thin,
    solid,
    color = ClrYearTagger,
    mark=*,
    mark options={%
      scale=0.7,
      draw=ClrYearTagger
    },
    error bars/.cd,
    x dir=both, x explicit,
    y dir=both, y explicit,
    error mark options={%
      rotate=90,
      mark size=3pt,
      color=ClrYearTagger
    }
]
table[
        x error plus=ex+,
        x error minus=ex-,
]{
  x            y  ex+            ex-  
  0.6065272082 7  0.06982395205  0.07056593733
  0.5470900868 6  0.1983877985   0.1995795379
  0.766496126  5  0.1417018624   0.1464811229
  0.7878244056 4  0.04842544413  0.04880814202
  0.686046884  3  0.1358504128   0.1356937662
  0.7839887379 2  0.09580978913  0.09899488064
};

%-------------------------------------------------------------------------------
% Full Fit
\node[exp_label, color=ClrFullFit] (Average_lbl) at (axis cs: \pgfkeysvalueof{/pgfplots/xmin},1)  
{Nominal fit};

% Full uncertainty
\addplot+[only marks,
    thin,
    solid, 
    color = ClrFullFit,
    mark=none,
    mark options={%
      scale=0.7,
      draw=ClrFullFit
    },
    error bars/.cd,
    x dir=both, x explicit,
    y dir=both, y explicit,
    error mark options={%
      rotate=90,
      mark size=3pt,
      color=ClrFullFit
    }
]
table[
        x error plus=ex+,
        x error minus=ex-,
]{
  x            y  ex+            ex-
  0.7285040271 1  0.03461272157  0.03477905032
};

\end{axis}
\end{tikzpicture}

%!TEX root = ../main.tex

\definecolor{Black}{HTML}{000000}
\definecolor{Red}{HTML}{FC5716}

\colorlet{ClrRndm}{Black}
\colorlet{ClrTrigger}{Black}
\colorlet{ClrTagger}{Black}
\colorlet{ClrTaggerYear}{Black}
\colorlet{ClrMagnetYear}{Black}
\colorlet{ClrTrackYear}{Black}
\colorlet{ClrFullFit}{Red}
\colorlet{ClrUncert}{Red!40}

\begin{tikzpicture}[
  exp_label/.style={
    anchor=west,
    %minimum width=10em,
    align=left,
    font=\small\sffamily,
    inner sep=0.5em,
    outer sep=0,
  },
  exp_result/.style={
    anchor=east,
    align=right,
    font=\footnotesize\sffamily,
    inner sep=0.5em,
    outer sep=0,
    yshift=0.22em
  }
]
\begin{axis}[
  width=\textwidth,
  height=65ex,
  font=\small,
  xmin=-0.661638482,xmax=0.338361518,ymin=0.3,ymax=23.7,
  xlabel={$\CJpsiKS$},
  xlabel style={
        at={(ticklabel cs:1)},
        anchor=north east,
    },%
  xtick={-0.4, -0.3, -0.2, -0.1, 0, 0.1, 0.2, 0.3},
  % xticklabels={,,},
  xticklabel style={%
    major tick length=3pt
  },
  hide y axis
]

\begin{pgfonlayer}{background}
% line at 0
% \draw[ultra thin,color=Black!80]({rel axis cs:0,0}-|{axis cs:0,0}) -- ({rel axis cs:0,1}-|{axis cs:0,0});
% line at -1
% \draw[ultra thin,color=Black!80,dashed]({rel axis cs:0,0}-|{axis cs:-1,0}) -- ({rel axis cs:-1,1}-|{axis cs:-1,0});
% line at +1
% \draw[ultra thin,color=Black!80,dashed]({rel axis cs:0,0}-|{axis cs:+1,0}) -- ({rel axis cs:+1,1}-|{axis cs:+1,0});
% filled area at average value
\fill[color=ClrUncert] ({rel axis cs:0,0}-|{axis cs:-0.06526549,0}) rectangle ({rel axis cs:-0.001013543,1}-|{axis cs:-0.001013543,0});
% line at average value
\draw[ultra thin,color=ClrFullFit]({rel axis cs:0,0}-|{axis cs:-0.0331344551,0}) -- ({rel axis cs:-0.0331344551,1}-|{axis cs:-0.0331344551,0});

% lines to separate the different subsamples categories
\draw[ultra thin, dashed, color=Black!80] (axis cs: \pgfkeysvalueof{/pgfplots/xmin},18.5) -- (axis cs: \pgfkeysvalueof{/pgfplots/xmax},18.5);
\draw[ultra thin, dashed, color=Black!80] (axis cs: \pgfkeysvalueof{/pgfplots/xmin},16.5) -- (axis cs: \pgfkeysvalueof{/pgfplots/xmax},16.5);
\draw[ultra thin, dashed, color=Black!80] (axis cs: \pgfkeysvalueof{/pgfplots/xmin},13.5) -- (axis cs: \pgfkeysvalueof{/pgfplots/xmax},13.5);
\draw[ultra thin, dashed, color=Black!80] (axis cs: \pgfkeysvalueof{/pgfplots/xmin},11.5) -- (axis cs: \pgfkeysvalueof{/pgfplots/xmax},11.5);
\draw[ultra thin, dashed, color=Black!80] (axis cs: \pgfkeysvalueof{/pgfplots/xmin}, 9.5) -- (axis cs: \pgfkeysvalueof{/pgfplots/xmax}, 9.5);
\draw[ultra thin, dashed, color=Black!80] (axis cs: \pgfkeysvalueof{/pgfplots/xmin}, 7.5) -- (axis cs: \pgfkeysvalueof{/pgfplots/xmax}, 7.5);
\draw[ultra thin, dashed, color=Black!80] (axis cs: \pgfkeysvalueof{/pgfplots/xmin}, 1.5) -- (axis cs: \pgfkeysvalueof{/pgfplots/xmax}, 1.5);
\end{pgfonlayer}

%-------------------------------------------------------------------------------
% Random
\node[exp_label, color=ClrRndm] (Rndm1_lbl) at (axis cs: \pgfkeysvalueof{/pgfplots/xmin},23){Rndm1};
\node[exp_label, color=ClrRndm] (Rndm2_lbl) at (axis cs: \pgfkeysvalueof{/pgfplots/xmin},22){Rndm2};
\node[exp_label, color=ClrRndm] (Rndm3_lbl) at (axis cs: \pgfkeysvalueof{/pgfplots/xmin},21){Rndm3};
\node[exp_label, color=ClrRndm] (Rndm4_lbl) at (axis cs: \pgfkeysvalueof{/pgfplots/xmin},20){Rndm4};
\node[exp_label, color=ClrRndm] (Rndm5_lbl) at (axis cs: \pgfkeysvalueof{/pgfplots/xmin},19){Rndm5};


% Full uncertainty
\addplot+[only marks,
    thin,
    solid,
    color = ClrRndm,
    mark=*,
    mark options={%
      scale=0.7,
      draw=ClrRndm
    },
    error bars/.cd,
    x dir=both, x explicit,
    y dir=both, y explicit,
    error mark options={%
      rotate=90,
      mark size=3pt,
      color=ClrRndm
    }
]
table[
        x error plus=ex+,
        x error minus=ex-,
]{
  x             y  ex+            ex-  
  0.02558441126 23 0.07021323593  0.07052451479
 -0.02565350348 22 0.07032629746  0.07023613489
 -0.06004908338 21 0.07253280748  0.07258763226
  0.04845072951 20 0.07425395458  0.07438156378
 -0.1428659543  19 0.07206119305  0.07164963211
};

%-------------------------------------------------------------------------------
% Trigger
\node[exp_label, color=ClrTrigger] (AU_lbl) at (axis cs: \pgfkeysvalueof{/pgfplots/xmin},18){Almost unbiased};
\node[exp_label, color=ClrTrigger] (EB_lbl) at (axis cs: \pgfkeysvalueof{/pgfplots/xmin},17){Exclusively biased};


% Full uncertainty
\addplot+[only marks,
    thin,
    solid,
    color = ClrTrigger,
    mark=*,
    mark options={%
      scale=0.7,
      draw=ClrTrigger
    },
    error bars/.cd,
    x dir=both, x explicit,
    y dir=both, y explicit,
    error mark options={%
      rotate=90,
      mark size=3pt,
      color=ClrTrigger
    }
]
table[
        x error plus=ex+,
        x error minus=ex-,
]{
  x             y  ex+            ex-  
 -0.05724869114 18 0.0348958308   0.03505910715
  0.1135758275  17 0.08424427739  0.08455088326
};

%-------------------------------------------------------------------------------
% Tagger
\node[exp_label, color=ClrTagger] (OS_lbl) at (axis cs: \pgfkeysvalueof{/pgfplots/xmin},16){Excl. OS};
\node[exp_label, color=ClrTagger] (SS_lbl) at (axis cs: \pgfkeysvalueof{/pgfplots/xmin},15){Excl. SS};
\node[exp_label, color=ClrTagger] (OL_lbl) at (axis cs: \pgfkeysvalueof{/pgfplots/xmin},14){Excl. BS};


% Full uncertainty
\addplot+[only marks,
    thin,
    solid,
    color = ClrTagger,
    mark=*,
    mark options={%
      scale=0.7,
      draw=ClrTagger
    },
    error bars/.cd,
    x dir=both, x explicit,
    y dir=both, y explicit,
    error mark options={%
      rotate=90,
      mark size=3pt,
      color=ClrTagger
    }
]
table[
        x error plus=ex+,
        x error minus=ex-,
]{
  x              y  ex+            ex-  
 -0.02182150859  16 0.03814187442  0.03797590581
 -0.1537951611   15 0.09971756852  0.100219264
 -0.003393348382 14 0.07641748829  0.07583334821
};

%-------------------------------------------------------------------------------
% Year
\node[exp_label, color=ClrTaggerYear] (11OS_lbl) at (axis cs: \pgfkeysvalueof{/pgfplots/xmin},13){2011};
\node[exp_label, color=ClrTaggerYear] (11SS_lbl) at (axis cs: \pgfkeysvalueof{/pgfplots/xmin},12){2012};

% Full uncertainty
\addplot+[only marks,
    thin,
    solid,
    color = ClrTaggerYear,
    mark=*,
    mark options={%
      scale=0.7,
      draw=ClrTaggerYear
    },
    error bars/.cd,
    x dir=both, x explicit,
    y dir=both, y explicit,
    error mark options={%
      rotate=90,
      mark size=3pt,
      color=ClrTaggerYear
    }
]
table[
        x error plus=ex+,
        x error minus=ex-,
]{
  x             y  ex+            ex-  
 -0.1377370909  13 0.0566527629   0.05670454656
  0.01779585578 12 0.03918441137  0.03919889162
};

%-------------------------------------------------------------------------------
% Track type
\node[exp_label, color=ClrTrackYear] (11DD_lbl) at (axis cs: \pgfkeysvalueof{/pgfplots/xmin},11){downstream};
\node[exp_label, color=ClrTrackYear] (11LL_lbl) at (axis cs: \pgfkeysvalueof{/pgfplots/xmin},10){long};


% Full uncertainty
\addplot+[only marks,
    thin,
    solid,
    color = ClrTrackYear,
    mark=*,
    mark options={%
      scale=0.7,
      draw=ClrTrackYear
    },
    error bars/.cd,
    x dir=both, x explicit,
    y dir=both, y explicit,
    error mark options={%
      rotate=90,
      mark size=3pt,
      color=ClrTrackYear
    }
]
table[
        x error plus=ex+,
        x error minus=ex-,
]{
  x              y ex+            ex-  
  0.02305527233 11 0.03990761127  0.03995817734
 -0.1348991135  10 0.05465942884  0.05464262893
};

%-------------------------------------------------------------------------------
% Magnet polarity and Year
\node[exp_label, color=ClrMagnetYear] (11UP_lbl) at (axis cs: \pgfkeysvalueof{/pgfplots/xmin},9){Up};
\node[exp_label, color=ClrMagnetYear] (11DW_lbl) at (axis cs: \pgfkeysvalueof{/pgfplots/xmin},8){Down};


% Full uncertainty
\addplot+[only marks,
    thin,
    solid,
    color = ClrMagnetYear,
    mark=*,
    mark options={%
      scale=0.7,
      draw=ClrMagnetYear
    },
    error bars/.cd,
    x dir=both, x explicit,
    y dir=both, y explicit,
    error mark options={%
      rotate=90,
      mark size=3pt,
      color=ClrMagnetYear
    }
]
table[
        x error plus=ex+,
        x error minus=ex-,
]{
  x             y  ex+            ex-  
 -0.02376582561 9  0.04758795926  0.04766667594
 -0.03958096368 8  0.04345994293  0.04349885924
};

%-------------------------------------------------------------------------------
% Year and Tagger
\node[exp_label, color=ClrYearTagger] (11OS_lbl) at (axis cs: \pgfkeysvalueof{/pgfplots/xmin},7){11 Excl. OS};
\node[exp_label, color=ClrYearTagger] (11SS_lbl) at (axis cs: \pgfkeysvalueof{/pgfplots/xmin},6){11 Excl. SS};
\node[exp_label, color=ClrYearTagger] (11BS_lbl) at (axis cs: \pgfkeysvalueof{/pgfplots/xmin},5){11 Excl. BS};
\node[exp_label, color=ClrYearTagger] (12OS_lbl) at (axis cs: \pgfkeysvalueof{/pgfplots/xmin},4){12 Excl. OS};
\node[exp_label, color=ClrYearTagger] (12SS_lbl) at (axis cs: \pgfkeysvalueof{/pgfplots/xmin},3){12 Excl. SS};
\node[exp_label, color=ClrYearTagger] (12BS_lbl) at (axis cs: \pgfkeysvalueof{/pgfplots/xmin},2){12 Excl. BS};

% Full uncertainty
\addplot+[only marks,
    thin,
    solid,
    color = ClrMagnetYear,
    mark=*,
    mark options={%
      scale=0.7,
      draw=ClrMagnetYear
    },
    error bars/.cd,
    x dir=both, x explicit,
    y dir=both, y explicit,
    error mark options={%
      rotate=90,
      mark size=3pt,
      color=ClrMagnetYear
    }
]
table[
        x error plus=ex+,
        x error minus=ex-,
]{
  x             y  ex+            ex-  
 -0.1416388853  7  0.06636570598  0.06650951901
 -0.2190836868  6  0.1798749598   0.1802323117
 -0.06927973414 5  0.1361005602   0.1364507049
  0.03692319471 4  0.04628104909  0.04632236703
 -0.1231213771  3  0.1193367706   0.1196054976
  0.02946883833 2 0.09129763057   0.09168226591
};

%-------------------------------------------------------------------------------
% Full Fit
\node[exp_label, color=ClrFullFit] (Average_lbl) at (axis cs: \pgfkeysvalueof{/pgfplots/xmin},1)  
{Nominal fit};

% Full uncertainty
\addplot+[only marks,
    thin,
    solid, 
    color = ClrFullFit,
    mark=none,
    mark options={%
      scale=0.7,
      draw=ClrFullFit
    },
    error bars/.cd,
    x dir=both, x explicit,
    y dir=both, y explicit,
    error mark options={%
      rotate=90,
      mark size=3pt,
      color=ClrFullFit
    }
]
table[
        x error plus=ex+,
        x error minus=ex-,
]{
  x            y  ex+            ex-
 -0.0331344551 1  0.03212091232  0.03213103531
};
\end{axis}
\end{tikzpicture}

\caption{
Comparison of fit results of \SJpsiKS and \CJpsiKS for fits on various
subsamples.}
\label{fig:measurement_of_sin2beta:systematics:cross_checks:subsamples:s_and_c}
\end{figure}

% ------------------------------------------------------------------------------
\subsubsection{Pure time-dependent and time-integrated fit}
\label{sec:measurement_of_sin2beta:systematics:cross_checks:time_integrated}

The fit is tested furthermore by disentangling the time-integrated and the
time-dependent parts of it. To ensure a purely time-dependent extraction of the
\CP violating parameters a binned $\chisq$-fit to the signal \CP asymmetry (\cf
\cref{fig:measurement_of_sin2beta:cpv_measurement:results:plots:asymmetry}) is
performed. The theoretical distribution $\CPAsymmetry$ given in
\cref{eq:cpv_theory:bd2jpsiks:cp_asymmetry}, modified to cover for tagging
and production asymmetries results in
%
\begin{multline}
  \Asym{\CP}{\text{exp}} = \\
    \frac{\mistag^{\Bd} - \mistag^{\Bdbar} + A_{\text{P}}^{\catOO} \Bigl(1 - \mistag^{\Bd} - \mistag^{\Bdbar}\Bigr) + \CPAsymmetry \Bigl(1 - \mistag^{\Bd} - \mistag^{\Bdbar} + A_{\text{P}}^{\catOO} \bigl(\mistag^{\Bd} - \mistag^{\Bdbar}\bigr) \Bigr)}
    {1 + A_{\text{P}}^{\catOO} \Bigl(\SJpsiKS \sin (\DMd t) - \CJpsiKS \cos (\DMd t)\Bigr) } \eqpd
\end{multline}
%
At this point the \OS and \SSpi mistag probabilities are combined on a per-event
basis assuming true \Bd and \Bdbar mesons. Using signal weights extracted by an
\sPlot fit on the mass distribution, the weighted averages of $\mistag^{\Bd} =
\num{0.3869}$ and $\mistag^{\Bdbar} = \num{0.3777}$ are incorporated while the
values for $A_{\text{P}}^{\catOO}$ and \DMd are taken as found in the nominal
fit. This results in central values fully compatible to the ones extracted from
the nominal fit
%
\begin{equation*}
  \begin{split}
    \SJpsiKS &= \num{0.713 +- 0.053} \eqcm \\
    \CJpsiKS &= \num{0.003 +- 0.055} \eqpd
  \end{split}
\end{equation*}
%
Leaving the mass difference \DMd floating in the fit, the fit yields
%
\begin{equation*}
  \begin{split}
    \SJpsiKS &= \num{0.718 +- 0.053} \eqcm \\
    \CJpsiKS &= \num{0.040 +- 0.077} \eqcm \\
    \DMd     &= \SI{0.540 +- 0.043}{\planckbar\per\pico\second} \eqpd
  \end{split}
\end{equation*}
%
In the purely time-integrated approach the number of tagged \Bd and \Bdbar
signal candidates is counted on an \sweighted sample to compute the
time-integrated \CP asymmetry
%
\begin{equation}
  \begin{split}
    \Asym{\CP}{\text{int}} &= \frac{N_{\Bdbar} - N_{\Bd}}{N_{\Bdbar} + N_{\Bd}} = \num{0.103 +- 0.00} \eqpd
  \end{split}
\end{equation}
%
Under the \SM assumption of $\CJpsiKS = \num{0}$ the relation
%
\begin{multline}\label{eq:measurement_of_sin2beta:systematics:cross_checks:time_integrated}
  \SJpsiKS^{\text{int}} = 
  \frac{ \Asym{\CP}{\text{int}} - \bigl( 1 - \mistag^{\Bd} - \mistag^{\Bdbar}\bigr) A_{\text{P}} \Asym{\CP}{\text{int}} \bigl( \mistag^{\Bd} - \mistag^{\Bdbar} \bigr)}
    {1 - \mistag^{\Bd} - \mistag^{\Bdbar} - A_{\text{P}} \Asym{\CP}{\text{int}} - \Asym{\CP}{\text{int}} \bigl( \mistag^{\Bd} - \mistag^{\Bdbar} \bigr) } \\
  \cdot\frac{1 - (\DMd \tau)^2}{ \sin (\DMd t_{\text{min}}) + \DMd \tau \cos (\DMd t_{\text{min}})} \eqcm
\end{multline}
%
leads to 
%
\begin{equation*}
  \SJpsiKS^{\text{int}} = \num{0.787 +- 0.009} \eqcm
\end{equation*}
%
where the lower integration limit $t_{\text{min}}$ is employed and the values
for the other parameters are handled as in the time-dependent case described
before. Taking into account the correlation between the \CP parameters \SJpsiKS
and \CJpsiKS and using the value for \CJpsiKS found in the nominal fit, the
previous equation extends by another summand,
%
\begin{equation}
  \SJpsiKS^{\text{int}} = \text{\cref{eq:measurement_of_sin2beta:systematics:cross_checks:time_integrated}} 
  + \CJpsiKS \frac{\DMd \tau \sin (\DMd t_{\text{min}}) + \cos (\DMd t_{\text{min}})}{\sin (\DMd t_{\text{min}}) + \DMd \tau \cos (\DMd t_{\text{min}})} \eqcm
\end{equation}
%
resulting in
%
\begin{equation*}
  \SJpsiKS^{\text{int}} = \num{0.746 +- 0.040} \eqpd
\end{equation*}
%

% ------------------------------------------------------------------------------
\subsection{Systematics}
\label{sec:measurement_of_sin2beta:systematics:systematics}

Systematic uncertainties resulting from the choice or the handling of the fit
model, the flavour tagging calibration, decay time resolution and acceptance,
and other ingredients to the measurement are summarised in the following
section. All pull and residual distributions originating from the studies are
collected in \cref{sec:app:measurement_of_sin2beta:systematics}. If not stated
otherwise the \CP parameter values are set to $\SJpsiKS = \num{0.7}$ and
$\CJpsiKS = \num{0.03}$ in the \acf{ToyMC} sample generation.

%...............................................................................
\subsubsection{Fit model}
\label{sec:measurement_of_sin2beta:systematics:systematics:fit_model}

\paragraph{A background tagging asymmetry} \ie a non-vanishing tagging
asymmetry in the background candidates sample is found and described in
\cref{sec:measurement_of_sin2beta:physic_backgrounds:tagging_asymmetries}. A
\ToyMC study with $\num{1000}$ iterations is performed to estimate the potential
influence on the measurement of $\SJpsiKS$ and $\CJpsiKS$. 

For each repetition a dataset is generated using the time-dependent asymmetries
provided by the histograms shown in
\cref{fig:measurement_of_sin2beta:physic_backgrounds:tagging_asymmetries:data}.
The asymmetry is sampled by incorporating the bin contents and errors. As the
decay time resolution model is assumed to not affect the outcome, an average
decay time resolution model is incorporated. A significant bias is observed in
both \CP parameters (see 
\cref{fig:app:measurement_of_sin2beta:systematics:systematics:fit_model:bkg_tagging_asymmetry}), 
thus the central value of the residual distributions is used as an estimate of
the systematic uncertainties
%
\begin{equation}
  \delta_{\SJpsiKS} = \num{0.0179}\eqthinspace\text{and}\eqthinspace\delta_{\CJpsiKS} = \num{0.0015}\eqpd
\end{equation}

\paragraph{The influence of a correlation between the mass and the decay time}
is investigated using a \ToyMC study. The nominal \PDF model assumes no
correlation, therefore the mass and decay time dimensions are directly
multiplied. Using the signal \MC sample the correlation between the mass and the
decay time resolutions, defined as $m-m_\text{true}$ and
$\obsTime-\obsTime_\text{true}$, is studied and depicted in
\cref{fit:measurement_of_sin2beta:systematics:systematics:fit_model:mass_time_correlations} 
as binned 2-dimensional histograms as well as profile plots. While the
correlation is insignificant for \catDD candidates, a small positive correlation
is found for \catLL candidates.
%
\begin{figure}[h]
  \includegraphics[width=0.49\textwidth]{private/content/measurement-of-sin2beta/figs/systematics_fitmodel_mtcorr_dd.pdf}\hfill
  \includegraphics[width=0.49\textwidth]{private/content/measurement-of-sin2beta/figs/systematics_fitmodel_mtcorr_ll.pdf}
  \includegraphics[width=0.49\textwidth]{private/content/measurement-of-sin2beta/figs/systematics_fitmodel_mtcorr_profile_dd.pdf}\hfill
  \includegraphics[width=0.49\textwidth]{private/content/measurement-of-sin2beta/figs/systematics_fitmodel_mtcorr_profile_ll.pdf}
\caption{Visualisation of the correlation between the mass and the decay time
resolution defined $m-m_\text{true}$ and $\obsTime-\obsTime_\text{true}$. On top
the binned 2-dimensional distribution is shown, while on the bottom profile
plots show the average of the decay time resolution as a function of the mass
resolution. Shown are the (left) \catDD and the (right) \catLL subsamples.}
\label{fit:measurement_of_sin2beta:systematics:systematics:fit_model:mass_time_correlations}
\end{figure}
%
In a study with $\num{250}$ iterations data are generated assuming the given
correlation for \catLL candidates while fitted with the nominal model. The pull
and residual plots shown in 
\cref{fig:app:measurement_of_sin2beta:systematics:systematics:fit_model:mass_time_correlations} 
show no significant bias on the \CP parameters.

%...............................................................................
\subsubsection{Flavour Tagging}
\label{sec:measurement_of_sin2beta:systematics:systematics:tagging}

The statistical uncertainties on the flavour tagging calibration parameters
($\p{0,1}{\text{\acs*{OS},\acs*{SSpi}}}$ and
$\deltap{0,1}{\text{\acs*{OS},\acs*{SSpi}}}$) are taken into account using
Gaussian constraints (\cf
\cref{sec:measurement_of_sin2beta:cpv_measurement:constrained_parameters}). The
influence of the systematic uncertainties are investigated by a \ToyMC study.

First the eight parameters are varied up and down by one systematic uncertainty
and the effect on the effective dilution is studied. This reduces to $16$
combinations if only varying $\p{i}{j}$ and $\deltap{i}{j}$ in the same direction.
For each combination $\num{750}$ generation/fit iterations are performed. The
largest deviation is found when all $\p{0}{j}$ parameters are varied up and all
$\p{1}{j}$ parameters are varied down. The smallest deviations are observed if
the parameters are moved vice versa. As the dilution and the \CP parameters are
part of a product in the \PDF the relative change in dilution of
$\SI{1.5}{\percent}$ is already an estimate of the systematic uncertainty.

A \ToyMC study with $\num{1000}$ iterations is performed using values for the
\CP parameters drawn from a Gaussian distribution where the central value is
shifted according to the result of the preparatory study and the statistical
uncertainty is used as width while including all correlations of the calibration
parameters 
(\cf \cref{eq:flavour_tagging:calibration:os:parameters,tab:flavour_tagging:calibration:ss:correlations}).
The nominal model is used in the subsequent fit. As depicted in 
\cref{fig:app:measurement_of_sin2beta:systematics:systematics:tagging} a
significant bias is observed. The offsets of the residual distributions
%
\begin{equation}
  \delta_{\SJpsiKS} = \num{0.0062}\eqthinspace\text{and}\eqthinspace\delta_{\CJpsiKS} = \num{0.0024}
\end{equation}
%
are used as an estimate of the systematic uncertainties.

%...............................................................................
\subsubsection{Decay time resolution}
\label{sec:measurement_of_sin2beta:systematics:systematics:resolution}

\paragraph{In the calibration of the decay time estimate} a linear function is
used. In a \ToyMC study the influence of a different parametrisation of the
calibration model is investigated. Based on the results given in
\cref{sec:measurement_of_sin2beta:resolution_and_acceptance:resolution} a
parabolic function with non-zero (zero) offset is used in the \catDD
(\catLL) sub sample to generate data, that is then fitted using the nominal
calibration model. The parameter values used in the generation are given in
\cref{tab:measurement_of_sin2beta:systematics:systematics:resolution:parabolic_parameters}.
In a study with $\num{250}$ iterations no significant bias is found as can been
seen in the pull and residual distributions provided in
\cref{fig:app:measurement_of_sin2beta:systematics:systematics:resolution:calibration}.
%
\begin{table}[h]
\centering
\caption{Fit parameters of the parabolic decay time resolution calibration
function for \catDD and \catLL candidates. No offset parameter is used in case
of the \catLL model, thus the correspondent entries are marked with a dash.}
\label{tab:measurement_of_sin2beta:systematics:systematics:resolution:parabolic_parameters}
  \begin{tabular}{llr@{$\,\pm\,$}lr@{$\,\pm\,$}l}
    \toprule
    \multicolumn{2}{c}{Parameter}               &   \multicolumn{2}{c}{\catDD}      &   \multicolumn{2}{c}{\catLL}\\
    \midrule
    $a_{1}$         &   (\si{\per\pico\second}) &   $-1.7$        &   $1.3$         &   $-4.3$        &   $0.7$         \\
    $b_{1}$         &                           &   $1.04$        &   $0.014$       &   $1.33$        &   $0.07$        \\
    $c_{1}$         &   (\si{\pico\second})     &   $0.0044$      &   $0.0027$      &   \multicolumn{2}{c}{---} \\
    $a_{2}$         &   (\si{\per\pico\second}) &   $-1.5$        &   $3.4$         &   $-2.5$        &   $1.0$         \\
    $b_{2}$         &                           &   $1.5$         &   $0.4$         &   $2.15$        &   $0.27$        \\
    $c_{2}$         &   (\si{\pico\second})     &   $0.016$       &   $0.008$       &   \multicolumn{2}{c}{---} \\
    \bottomrule
  \end{tabular}
\end{table}

\paragraph{An offset to the central values} of the three Gaussian \acp{PDF} is
used in the parametrisation of the decay time resolution model (\cf
\cref{eq:measurement_of_sin2beta:resolution_and_acceptance:resolution}). Due to
technical limitations in the implementation of the cubic spline \acp{PDF} the
value of this offset $\mu_{\obsTime}$ has to be set to zero in the nominal fit
although non-zero values are found in the determination of the resolution model.
The influence of ignoring the non-zero values is tested in a \ToyMC study with
$\num{250}$ iterations, where the data is generated using the offset's value
originally determined, while fitting with the nominal fit model. No significant
bias is observed. Plots of the pull and residual distributions are shown in
\cref{fig:app:measurement_of_sin2beta:systematics:systematics:resolution:offset}.

\paragraph{The fraction of wrongly associated \acp{PV}} is assumed to be
independent of the decay time resolution estimate $\obsTimeError$. This ignores
an increase in the fraction $f_\text{\acs*{PV}}$ as a function of the decay time
error. The influence of this simplification is tested in a \ToyMC study with
$\num{1000}$ iterations, where the fraction varies following a parabolic
function with offset fixed to zero in the generation, while $f_\text{\acs*{PV}}$
is set as in the nominal model in the subsequent fit. As depicted in
\cref{fig:app:measurement_of_sin2beta:systematics:systematics:resolution:wrong_pv} 
a significant bias is present for the parameter $\SJpsiKS$ as well as for
$\CJpsiKS$. Thus, systematic uncertainties of
%
\begin{equation}
  \delta_{\SJpsiKS} = \num{0.0021}\eqthinspace\text{and}\eqthinspace\delta_{\CJpsiKS} = \num{0.0011}
\end{equation}
%
are assigned.

%...............................................................................
\subsubsection{Decay time acceptance}
\label{sec:measurement_of_sin2beta:systematics:systematics:acceptance}

\paragraph{The low decay time acceptance} influence on the \CP parameters is
studied using a modified acceptance shape. In the generation histograms of the
time-dependent ratios $\varepsilon_{\catAU}$ and $\varepsilon_{\catEB}$ (\cf
\cref{sec:measurement_of_sin2beta:resolution_and_acceptance:acceptance:lower})
are used that are determined on the signal \MC sample using the same methodology
as used for the nominal model. In $\num{250}$ iterations the generated data are
fitted using the nominal (\catOT and \catOS) fit model, as no influence from the
tagging decision and the year of data taking is expected. The \ToyMC study shows
no significant bias. Pull and residual distributions can be found
in
\cref{fig:app:measurement_of_sin2beta:systematics:systematics:acceptance:lower}.

\paragraph{The upper decay time acceptance} is parametrised in the nominal fit
using a linear scale model. As outlined in
\cref{sec:measurement_of_sin2beta:resolution_and_acceptance:acceptance:upper}
higher order effects are studied using a quadratic correction function but are
not considered in the fit later on. To estimate the impact of a deviating
correction function on the \CP parameters a \ToyMC study with $\num{1000}$
iterations is performed. As the upper decay time acceptance is not expected to
dependent on the trigger category nor the tagging category, data are only
generated using the (\catAU and \catOS) fit model. The generation is performed
using the parametrisation given in \cref{eq:measurement_of_sin2beta:resolution_and_acceptance:acceptance:upper:quadratic} 
with parameters fixed to the values shown in
\cref{tab:measurement_of_sin2beta:resolution_and_acceptance:acceptance:upper:quadratic}. 
In the fit the nominal model with a linear correction function is used. No
deviation of a standard normal distribution is found for the \SJpsiKS pull
distribution. A small bias on \CJpsiKS is visible in the pull distribution,
therefore a systematic uncertainty of
%
\begin{equation}
  \delta_{\CJpsiKS} = \num{0.0012}
\end{equation}
%
is assigned. Pull and residual distributions can be found in
\cref{fig:app:measurement_of_sin2beta:systematics:systematics:acceptance:upper}.

%...............................................................................
\subsubsection[Production asymmetry, $z$-scale, \DMd, and \DGd]{Production asymmetry, $\mathbfsfit{z}$-scale, $\mathbfsfit{\DMd}$, and $\mathbfsfit{\DGd}$}
\label{sec:measurement_of_sin2beta:systematics:systematics:further_studies}

\paragraph{The $\mathbfsfit{z}$-scale alignment} of the \LHCb detector is known
up to a relative uncertainty of $\sigma_\text{$z$-scale} =
\SI{0.022}{\percent}$. To study how this affects the measurement of the \CP
parameters, data are generated using an offset of $\mu_{t} =
\sigma_\text{$z$-scale} t$ in the decay time resolution model. In the fit the
offset is set to zero as in the nominal fit model. In a \ToyMC study with
$\num{1000}$ iterations a small bias is observed for \SJpsiKS as well as for
\CJpsiKS and systematic uncertainties of
%
\begin{equation}
  \delta_{\SJpsiKS} = \num{0.0012}\eqthinspace\text{and}\eqthinspace\delta_{\CJpsiKS} = \num{0.0023}
\end{equation}
%
are assigned. Pull and residual distributions can be found in
\cref{fig:app:measurement_of_sin2beta:systematics:systematics:further_studies:zscale}.

\paragraph{The influence of the production asymmetry} on the measurement of
\SJpsiKS and \CJpsiKS is investigated in a \ToyMC study. In $\num{1000}$
iterations, samples are produced with an enlarged production asymmetry and
subsequently fitted with the nominal fit model. The production asymmetry is
varied by one statistical uncertainty: $A_P^{\catOO} = \num{-0.0122}$ while
$\Delta A_P$ is not shifted, as any systematic effects are already covered. No
systematic shift is observed. The pull and residual distributions are depicted
in \cref{fig:app:measurement_of_sin2beta:systematics:systematics:further_studies:production_asymmetry}.

\paragraph{The \Bdbfsf decay width difference's} impact on the \CP measurement
is examined in a \ToyMC study with $\num{1000}$ iterations. The data is
generated using $\DGd = \SI{0.007}{\per\pico\second}$ as an upper approximation
based on the current world average of the ratio between the decay width
difference and the absolute decay width $\DGd/\Gd = \num{0.001 +- 0.010}$
\cite{Amhis:2014hma}. The fit then neglects the non-zero $\DGd$ value as in the
nominal model. A significant bias is observed in the pull distribution of
$\SJpsiKS$, thus a systematic uncertainty of
%
\begin{equation}
  \delta_{\SJpsiKS} = \num{0.0047}
\end{equation}
%
is assigned as an estimate to cover for this effect. The pull and residual
distributions for $\SJpsiKS$ and $\CJpsiKS$ are depicted in
\cref{fig:app:measurement_of_sin2beta:systematics:systematics:further_studies:decay_width_difference}.

\paragraph{The \Bdbfsf mass difference $\mathbfsfit{\DMd}$} is taken as a
constrained external input to the fit model. The influence of the systematic
uncertainty on $\DMd$ is treated in a \ToyMC study with $\num{1000}$ iterations
of generation and fit. In the generation the value of the mass difference is
enlarged by the systematic uncertainty, \ie $\DMd =
\SI{0.512}{\planckbar\per\pico\second}$, the nominal model is used in the fit.
The pull distribution of $\CJpsiKS$ shows a significant deviation, therefore the
offset of the residual distribution
%
\begin{equation}
  \delta_{\CJpsiKS} = \num{0.0034}
\end{equation}
%
is taken as an estimate of the systematic uncertainty.
\Cref{fig:app:measurement_of_sin2beta:systematics:systematics:further_studies:mass_difference} 
shows the pull and residual distributions for $\SJpsiKS$ and $\CJpsiKS$.

% ------------------------------------------------------------------------------
\subsection{Summary of systematic effects}
\label{sec:measurement_of_sin2beta:systematics:summary}

The systematic uncertainties are summarised in
\cref{tab:measurement_of_sin2beta:systematics:summary}. The overall systematic
uncertainty is calculated by summing the single uncertainties in quadrature. The
relative systematic uncertainties compared to the central values of $\SJpsiKS$
and $\CJpsiKS$ are given in brackets. Here, $\SJpsiKS = \num{0.729}$ and
$\CJpsiKS = \num{-0.033}$ are set as reference values.
%
\begin{table}[h]
  \caption{Systematic uncertainties $\delta_{\SJpsiKS}$ and $\delta_{\CJpsiKS}$
  on $\SJpsiKS$ and $\CJpsiKS$. Entries marked with a dash represent studies
  where no significant effect is observed.}
  \label{tab:measurement_of_sin2beta:systematics:summary}
  \begin{tabular}{lrlrl}
    \toprule
    Origin & \multicolumn{2}{c}{$\delta_{\SJpsiKS}$} & \multicolumn{2}{c}{$\delta_{\CJpsiKS}$}    \\
    \midrule
    {Background tagging asymmetry}           &  $\tablenum[table-format = 1.4]{0.0179}$ & ($\tablenum[table-format = 1.1]{2.5}\si{\percent}$) &  $\tablenum[table-format = 1.4]{0.0015}$ & ($\tablenum[table-format = 2.1]{4.5}\si{\percent}$) \\
    {Tagging calibration}                    &  $\tablenum[table-format = 1.4]{0.0062}$ & ($\tablenum[table-format = 1.1]{0.9}\si{\percent}$) &  $\tablenum[table-format = 1.4]{0.0024}$ & ($\tablenum[table-format = 2.1]{7.2}\si{\percent}$) \\
    {$\Delta \Gamma$}                                  &  $\tablenum[table-format = 1.4]{0.0047}$ & ($\tablenum[table-format = 1.1]{0.6}\si{\percent}$) &  \multicolumn{2}{c}{---}               \\
    {Fraction of wrong PV component}         &  $\tablenum[table-format = 1.4]{0.0021}$ & ($\tablenum[table-format = 1.1]{0.3}\si{\percent}$) &  $\tablenum[table-format = 1.4]{0.0011}$ & ($\tablenum[table-format = 2.1]{3.3}\si{\percent}$) \\
    {$z$-scale}                              &  $\tablenum[table-format = 1.4]{0.0012}$ & ($\tablenum[table-format = 1.1]{0.2}\si{\percent}$) &  $\tablenum[table-format = 1.4]{0.0023}$ & ($\tablenum[table-format = 2.1]{7.0}\si{\percent}$) \\
    {$\mathrm{\Delta} m$}                    &  \multicolumn{2}{c}{---}               &  $\tablenum[table-format = 1.4]{0.0034}$ & ($\tablenum[table-format = 2.1]{10.3}\si{\percent}$)\\
    {Upper decay time acceptance}            &  \multicolumn{2}{c}{---}               &  $\tablenum[table-format = 1.4]{0.0012}$ & ($\tablenum[table-format = 2.1]{3.6}\si{\percent}$) \\
    {Correlation between mass and decay time}&  \multicolumn{2}{c}{---}               &  \multicolumn{2}{c}{---}               \\
    {Decay time resolution calibration}      &  \multicolumn{2}{c}{---}               &  \multicolumn{2}{c}{---}               \\
    {Decay time resolution offset}           &  \multicolumn{2}{c}{---}               &  \multicolumn{2}{c}{---}               \\
    {Low decay time acceptance}              &  \multicolumn{2}{c}{---}               &  \multicolumn{2}{c}{---}               \\
    {Production asymmetry}                   &  \multicolumn{2}{c}{---}               &  \multicolumn{2}{c}{---}               \\
    \midrule
    Sum                                            &  $0.020$        & ($\tablenum[table-format = 1.1]{2.7}\si{\percent}$) & $0.005$          & ($\tablenum[table-format = 2.1]{15.2}\si{\percent}$)\\
    \bottomrule
  \end{tabular}
\end{table}
