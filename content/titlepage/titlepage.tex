%!TEX root = ../../common/main.tex

\begin{titlepage}

\vspace*{20ex}
{%
\Huge \sffamily \bfseries 
\begin{center}
Measurement of \CPbfsf violation in \BdToJpsiKSbfsf decays
\end{center} 
}%

\begin{german}
{%
\LARGE \sffamily %\bfseries
\begin{center}
Dissertation zur Erlangung des akademischen Grades\\
\end{center}
}

{%
\LARGE \sffamily %\bfseries
\begin{center}
Dr.~rer.~nat.
\end{center}
}

\vspace{5ex}


{%
\Large \rmfamily
\begin{center}
vorgelegt von \\ [0.8ex]
Christophe Arnold Augustin Cauet \\ [0.8ex]
geboren am 13.02.1984 \\
in Dortmund
\end{center}
}
\vspace{5ex}
{%
\Large \rmfamily
\begin{center}
Fakultät Physik\\
Technische Universität Dortmund
\end{center}
}
\vspace{4ex}
{%
\Large \rmfamily
\begin{center}
Dortmund, im XY 2015
\end{center}
}

\clearpage
\thispagestyle{empty}
\vspace*{\fill}
\noindent Der Fakultät Physik der Technischen Universität Dortmund zur Erlangung
des akademischen Grades eines Doktors der Naturwissenschaften vorgelegte
Dissertation.\\

\parbox{0.90\textwidth}{
  1.~Gutachter: Prof.~Dr.~Bernhard~Spaan \\
  2.~Gutachter: Priv.-Doz.~Dr.~N.N.\\
  Datum der mündlichen Prüfung: 1.~XY 2015\\
  Vorsitzender des Promotionsausschusses: Prof.~Dr.~N.N.
}
\end{german}
\end{titlepage}

\clearpage
\thispagestyle{empty}

\section*{Abstract}
LHCb, measurement, BdToJpsiKS, CPV in interference between osci and decay, CP asymmetry, S and C, beta, dataset, result, best at hadron collider, equally good as babar and belle, consistent with SM expectations and previous measurements


\begin{german}
\section*{Zusammenfassung}
Dies ist eine kurze Zusammenfassung der präsentierten Arbeit

\end{german}

\setcounter{page}{1}

% \linenumbers